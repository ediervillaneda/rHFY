\documentclass[spanish,12pt,a4paper, oneside]{book}
\usepackage[T1]{fontenc}
\usepackage[left=2cm, right=2cm, top=2cm, bottom=2cm]{geometry}
\usepackage{graphicx}
\usepackage{babel}

\setlength{\parskip}{2mm}

\title{Excepto los humanos}
\author{u/SomeOne111Z}
\date{2021}
\begin{document}
    \maketitle
    \chapter*{Excepto los humanos}
    Cuando los drones de exploración de nuestra coalición encontraron el sistema Sol, supimos desde los primeros escaneos que no había mucho de interés allí. La mayoría de los planetas estaban completamente desprovistos de vida de cualquier tipo; solo un puñado de rocas estériles y gigantes gaseosos. El único planeta interesante era el tercer planeta, cuyos escaneos mostraron estar lleno de vida.

    Durante los próximos días, nuestra sonda exploró la Tierra. Aunque no podía ver mucho a través de la inmensa cantidad de escombros espaciales que cubrían el planeta, aparentemente de lanzamientos fallidos al espacio, obtuvo información sobre lo que había en la superficie. Había una diversidad de flora y fauna, una abundancia de agua líquida que sorprendentemente estaba llena de restos de islas estancadas, y una especie sapiente conocida como humanos que parecía ser la causa de las estructuras relativamente comunes y los mencionados desechos espaciales. Los escaneos mostraron que había aproximadamente ocho mil millones de miembros de esta raza.

    La sonda continuó avanzando después de un par de semanas escaneando la superficie. Su misión principal era explorar un sistema recién descubierto que parecía ser rico en recursos, y no se detendría por un planeta en medio de ninguna parte. Claro, los humanos se habían dado cuenta de que podían ir al espacio, pero por la cantidad de escombros que rodeaban su planeta, todos sus intentos habían fallado en su mayoría. A la mayoría de los científicos les parecía que los humanos necesitarían otros cuatrocientos años antes de poder construir naves capaces de abandonar la Tierra, y aún más para descubrir los viajes a velocidades superlumínicas por sí mismos. Nunca habíamos visto una raza tan lentamente progresiva antes. La mayoría de las especies que intentaban llegar al espacio convertían inmediatamente ese objetivo en su principal prioridad, para romper las limitaciones de su planeta natal.

    Excepto los humanos.

    En algún momento, estalló un conflicto entre dos razas miembros de la coalición: los poderosos y belicosos Xanzi y los Dan'ik, que eran más nuevos y mucho más dóciles. Todos sabían que los Dan'ik eran las víctimas en esta guerra, pero sabían que enfrentarían la ira del poderoso ejército de los Xanzi si intervenían. Así que las naves de los Xanzi fácilmente arrasaron y esterilizaron un mundo, capturaron su otro planeta sin esfuerzo y utilizaron su influencia en el Senado para convencer a sus aliados y algunas razas indecisas de expulsar a los Dan'ik de la coalición.

    Nadie intentó defender a los Dan'ik; eran una raza miembro novata, sin representantes en el gobierno, con pocas cualidades destacadas. Solo diez mil millones de miembros, un ejército casi inexistente (lo que puede haber sido parte de la razón por la que los Xanzi invadieron) y solo unas pocas naves capaces de viajar a velocidades superlumínicas. Eran casi tan lamentables como los humanos, como señaló un Senador, y podría ser mejor enviarlos al sistema Sol y poner a prueba a los humanos. Después de todo, sería un desperdicio de recursos enviar una raza realmente importante y una nueva flota para saludar a los humanos; ¿por qué no usar a los Dan'ik?

    Entonces, cincuenta y dos grandes naves coloniales Dan'ik y numerosos cargueros y transportes civiles más pequeños, todos llevando su parte de los siete mil millones que sobrevivieron a la guerra y no fueron esclavizados por los Xanzi, saltaron por su cuenta al sistema Sol, sin apoyo y sin posibilidad de retorno. Sus naves estaban abarrotadas de refugiados, personas sin un hogar al que regresar. Nadie estaría contento de recibirlos en el estado en que se encontraban.

    Excepto, aparentemente, los humanos.

    Pasaron varias semanas antes de que nuestra segunda sonda llegara al sistema Sol, creada específicamente para estudiar cómo interactuaban las razas humanas y Dan'ik. Para cuando llegó, trajo noticias de lo que había sucedido: los gobiernos humanos de la Tierra (y sí, había muchos de ellos) habían aceptado a los Dan'ik en su sociedad. Su planeta estaba lleno de construcción, y aproximadamente la mitad de las naves que llegaron a su destino todavía estaban en órbita. La mayoría de los escombros espaciales, curiosamente, habían desaparecido. Sin embargo, llevaría un tiempo adaptarse por completo. Nadie podía manejar tantos refugiados de una vez.

    Excepto, como descubrimos, los humanos.

    Los pocos investigadores encargados de ver las imágenes de la sonda vieron que los habitantes de la Tierra se adaptaron de manera extraordinariamente rápida a su nueva situación. Construyeron nuevas estructuras a una velocidad sorprendente, utilizando enormes máquinas que simplemente imprimían ciudades enteras en solo unas semanas. Los cargueros Dan'ik que antes transportaban carga ahora se movían en el cinturón de asteroides, extrayendo materiales para construir nuevas estructuras y, sorprendentemente, nuevas naves. Al parecer, los humanos habían aprendido a hacer ingenieria inversa a prácticamente todas las tecnologías que los Dan'ik tenían para ofrecer, desde escudos de partículas hasta motores superlumínicos. Los vehículos que antes tenían ruedas o cadenas se convirtieron para contener nodos de antigravedad; se construyeron nuevas naves únicas, supuestamente de diseño humano, y para nuestra sorpresa demostraron ser aún más rápidas que las naves Dan'ik que las precedieron.

    Nunca habíamos visto tanta industriosidad antes. En la mayoría de los planetas, se tardaba meses en construir una nueva nave, y otro mes si se trataba de una con un motor superlumínico, que era algo raro en la coalición. La mayoría de las naves de la coalición dependían de motores solares, ya que los motores superlumínicos eran demasiado complicados para producir en masa y demasiado costosos para hacerlos de uso estándar; solo los ricos comerciantes entre sistemas o las grandes naves militares los tenían, mientras que la mayoría de las naves más pequeñas, como cruceros y fragatas, no los tenían en absoluto. Y en este planeta previamente ignorado, se construían y lanzaban nuevas naves en una semana. Mientras algunas seguían orbitando la Tierra o la luna local, la mayoría utilizaba de inmediato sus motores superlumínicos y perforaban el espacio-tiempo, viajando a quién sabe dónde.

    En este punto, nuestros líderes estaban creciendo tanto curiosidad como ansiedad ante esta nueva amenaza para su forma de vida. Los generales Xanzi abogaban por apoderarse de lo que hacía que los humanos fueran tan eficientes, y nuestro Senado capituló y permitió a los Xanzi reunir una fuerza de ataque para recuperar lo que consideraban suyo por derecho. Cien naves masivas, incluyendo portaviones, destructores, acorazados, fragatas e incluso un acorazado planetario capaz de destruir un planeta, todas cargadas con suficiente potencia de fuego para simplemente someter a otra raza miembro si así lo deseaban. Miles de cazas pequeños, cientos de naves de transporte de tropas y naves de aterrizaje para la eventual toma de la Tierra. Un solo planeta frente a la fuerza total de una flota Xanzi era una presa fácil, sin importar cuán preparados estuvieran. Y no había forma de que la Tierra estuviera preparada, después de solo un año desde que los Dan'ik aparecieron por primera vez en su puerta. La mayoría de las razas se rendirían de inmediato y esperarían escapar con sus vidas, como los Dan'ik lo hicieron después de solo dos días de combates.

    Excepto, por supuesto, los humanos.

    Tan pronto como la flota Xanzi llegó cerca del gigante gaseoso más grande del sistema para reagruparse y comenzar su ataque, una pequeña flota de una docena de naves de crucero se reunió cerca de la Tierra para defender su hogar. A medida que los Xanzi se acercaban al cinturón de asteroides, la flota humana crecía constantemente hasta que igualaba en tamaño a los Xanzi, ya que más naves saltaban desde aparentemente la nada. Y cuando los Xanzi cruzaron el cinturón, la flota humana se aceleró para encontrarse con ellos en medio, en un pequeño planeta rojo llamado Marte. Cuando ambas fuerzas chocaron, la batalla fue más igualada de lo que cualquiera hubiera pensado que podría ser; aunque las naves Dan'ik originalmente tenían escudos y armamento pobres solo un año antes, los cruceros y acorazados humanos parecían igualar a sus contrapartes Xanzi, intercambiando golpe por golpe devastador. La flota Xanzi se redujo a cincuenta naves, incluyendo su acorazado insignia y algunos portaviones destinados a la subyugación de un planeta, luchando contra sesenta cruceros y destructores humanos, todos decididos a rechazar al enemigo.

    Finalmente, los Xanzi decidieron retirarse. El acorazado se alejó de la Tierra, tratando de saltar de regreso a territorio amistoso antes de ser destruido. Antes de que él y las naves restantes pudieran hacerlo, sin embargo, los tres mayores portaviones humanos liberaron una pequeña oleada de transportes de tropas rápidos, casi quinientos en total, cada uno dirigido a una nave Xanzi específica. Mientras la flota intentaba huir, no logró ver los transportes que se abrían paso hacia los cascos de sus naves y se afianzaban antes de saltar a un sistema aliado cercano, con infantes de marina humanos a bordo. La sonda de exploración observó cómo las naves humanas se mantenían cerca de Marte, como si estuvieran esperando a que las naves Xanzi regresaran y reanudaran su ataque. Pero las naves Xanzi se habían ido para siempre; habían retrocedido. Nunca antes una flota completa de los Xanzi había sido tan maltratada que se vieron obligados a retirarse; ninguna raza militar era capaz de enfrentarlos en una batalla y mantenerse firme.

    Excepto, evidentemente, los humanos.

    Los líderes Xanzi quedaron conmocionados por la derrota. Toda una flota completa y cuatro millones de soldados, reducidos a un cuarto de su fuerza, por el esfuerzo de un solo planeta y lo que eran un grupo de simios torpes hace tan solo cuatrocientos días. Hubo indignación por este giro de los acontecimientos, y los generales Xanzi exigieron que el almirante insignia fuera reprendido, ejecutado e incluso degradado a guardiamarina después. Sin embargo, las comunicaciones con el acorazado insignia nunca obtuvieron respuesta, y se presumió que la nave estaba destruida. Esto fue aún peor; uno de sus buques insignia había sido destruido. Esas naves eran lo mejor de lo mejor, y solo había nueve de ellas en toda la coalición; ¡era incomprensible! Los Xanzi se prepararon para movilizar a toda su fuerza militar (ya que el resto de la coalición no quería intervenir por diversas razones); sus cuatro flotas completas restantes se unieron y se prepararon para encontrarse con los supervivientes de su fuerza expedicionaria.

    A su llegada, los almirantes de la flota tuvieron una sorpresa agradable; su preciado acorazado insignia no había sido, de hecho, destruido. Permanecía en su lugar, rodeado por sus portaviones de apoyo y acorazados, todos con cuatro enormes cañones planetarios cargados y listos. Esto también sorprendió a los almirantes de la flota; era un protocolo estándar nunca mantener los cañones de un acorazado cargados a menos que estuvieran listos para disparar; de lo contrario, la energía de la nave se agotaría, sus escudos flaquearían y sus sistemas de enfriamiento eventualmente no cumplirían su función.

    Fue demasiado tarde cuando los cuatro almirantes Xanzi se dieron cuenta de lo que había sucedido. Cuando dieron la orden de emprender maniobras evasivas, de redirigir toda la energía a los escudos, de cargar sus propios cañones o de abandonar las naves por completo (ya que los cuatro acorazados insignia, en el caos, olvidaron comunicarse entre sí), el acorazado que tenían frente a ellos disparó sus cañones con una precisión perfecta. Cada enorme ráfaga de energía, con el poder de mil cabezas nucleares, impactó directamente en su objetivo. La fuerza de la explosión no fue suficiente para destruir por completo los acorazados, pero eliminó sus escudos, desactivó sus reactores principales y de respaldo, y destruyó la mayoría de las naves de apoyo en las proximidades. Mientras los cuatro acorazados quedaban a oscuras, con todos los sistemas desconectados, los almirantes observaron a través de las ventanas de sus puentes cómo las sesenta naves humanas con las que su colega había luchado saltaban hacia ellos, se volvían hacia los acorazados y lanzaban oleadas de naves de ataque...

    Cuatro días después, los Xanzi firmaron un tratado de rendición incondicional ante la recién formada Unión Terráquea. Entregaron a todos sus esclavos, renunciaron a su tecnología y declararon que ya no poseían los cinco acorazados que ahora ocupaba la fuerza militar conjunta humana/Dan'ik, que se cernía ominosamente sobre la capital del mundo natal de los Xanzi. Nunca antes nadie se había enfrentado a la fuerza más poderosa de la galaxia y había vuelto sus propias armas en su contra. No una sola vez los Xanzi habían sufrido una derrota tan abrumadora, una derrota que nunca les permitiría ser tan poderosos como una vez lo fueron. Parecía que nada volvería a ser tan poderoso como antes.

    Excepto, bueno, los humanos.
\end{document}