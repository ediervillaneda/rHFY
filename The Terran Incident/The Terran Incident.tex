\documentclass[spanish,12pt,a4paper,oneside,titlepage, twocolumn]{article}
\usepackage[T1]{fontenc}
\usepackage[left=2cm, right=2cm, top=2cm, bottom=2cm]{geometry}
\usepackage{amssymb}
\usepackage{graphicx}
\usepackage[spanish]{babel}
\usepackage{incgraph,tikz}
\usepackage{float}

\setlength{\parskip}{2mm}

\title{El Incidente Terrano}
\author{u/MisterNailbrain75}
\date{2021}

\begin{document}
    \maketitle

    \begin{abstract}
        En esta historia, se relata cómo la Supremacía Kachaxiana, una poderosa entidad galáctica, subestima a la humanidad y decide esclavizar a los Terranos después de descubrirlos en un mundo llamado Janus. Sin embargo, los Terranos, a pesar de ser considerados primitivos, sorprenden a la Supremacía con su tecnología y habilidades militares avanzadas. Tras una devastadora batalla en Janus y la amenaza de armas nucleares por parte de los Terranos, la Supremacía Kachaxiana se ve obligada a rendirse y liberar a los Terranos. Este evento marca el declive de la Supremacía Kachaxiana y su eventual colapso, mientras que los mundos anteriormente bajo su control buscan reafirmarse. La historia subraya el temor persistente hacia los Terranos y su capacidad de destrucción, así como la importancia de no subestimar a otras civilizaciones en el espacio.
    \end{abstract}

    \section*{\centering Introducción}

    "Durante muchos siglos, nuestra galaxia disfrutó de una larga paz inquebrantable. A pesar de pequeñas insurrecciones planetarias de escala reducida o disturbios políticos ocasionales, el orden y la serenidad eran mantenidos por la voluntad de la Supremacía Kachaxiana.

    Desde sus humildes comienzos como una raza de comerciantes, los Kachax habían alcanzado prominencia durante los eventos de la Gran Guerra Civil Myari (también conocida por algunos historiadores como "La Guerra de los Príncipes Myari") que tuvo lugar en el año 1765 del calendario Kachaxiano (Nota al pie: Para mayor claridad, me referiré a todas las fechas utilizando el Calendario Kachaxiano, ya que es una de las pocas cronologías adoptadas por múltiples culturas en toda la galaxia). Tras la muerte del Sultán Myari Askesh XX, sus tres hijos legítimos reunieron sus fuerzas para luchar por el derecho a gobernar el planeta.

    Si bien las guerras de sucesión eran raras, no eran desconocidas, y como resultado, la Compañía Myari Kachaxiana (CMK) desplegó su propio ejército privado para asegurar y defender sus activos en la superficie del planeta Myari. Fue durante este tiempo que los Príncipes (después de haber presenciado a las fuerzas de la CMK repeler una fuerza de bandidos más grande que se había unido para aprovechar el conflicto y saquear los asentamientos circundantes) comenzaron a contratar a los Kachax como mercenarios. A lo largo del conflicto, la CMK alquiló sus fuerzas de seguridad como mercenarios al mejor postor, al mismo tiempo que se ocultaba tras una delgada máscara de neutralidad que les permitía seguir comerciando con todas las partes. Después de todo, cada uno de los tres pretendientes era consciente de que atacar directamente a la CMK no solo significaba perder permanentemente sus servicios altamente valorados, sino también arriesgarse a involucrar al gobierno Kachaxiano en la guerra.

    Al final de la guerra civil, la CMK se había establecido prácticamente como una de las potencias dominantes en Myari. Incluso si hubiera deseado hacerlo, el recién coronado Sultán Askesh XXI no estaba en posición de rechazar la ayuda de los Kachax, ya que sus propios recursos y fuerzas militares habían sido diezmados como resultado del conflicto. Intentar destituir a la CMK del poder no solo sería casi imposible, sino que también le negaría suministros vitales ofrecidos por los Kachax. Así que la CMK estaba aquí para quedarse en Myari.

    En los siguientes 50 años estándar, la influencia de la CMK comenzó a crecer. A pesar de los deseos del Sultán, las fuerzas militares de la CMK continuaron aumentando en Myari, ayudando a sofocar facciones rebeldes y proporcionando una mayor seguridad. Mientras tanto, los extraterrestres prácticamente llegaron a dominar los mercados Myari. En el año 1812, los Kachax controlaban Myari en todo menos en nombre. En 1815, el Sultán Tukri II firmaría el Tratado de Bri, convirtiendo a Myari en un planeta vasallo de los Kachax. Algunos historiadores y políticos contemporáneos señalaron lo conveniente que fue la muerte del Sultán Askesh XXII y todos sus hijos como resultado de una explosión de motor de curvatura para los Kachax, ya que Askesh XXII se oponía vehementemente a la intromisión de extranjeros en los asuntos Myari, mientras que su hermano Tukri (un firme defensor de los Kachax) misteriosamente se sintió enfermo el día del viaje y posteriormente fue el único miembro de la familia real que sobrevivió. Los Kachax han negado cualquier implicación en el incidente y, sin evidencia en contrario (como suele ocurrir cuando los motores de curvatura explotan; para una lectura adicional, recomiendo encarecidamente el libro de Graktar Mork "Catástrofes interestelares que sacudieron la galaxia"), lamentablemente debemos coincidir con la investigación que califica la destrucción del crucero real como nada más que un trágico accidente. Después de todo, incluso hoy en día, los Myari no son precisamente conocidos por su construcción de naves (como se aborda en todo el libro de Mork en el que se dedican cuatro capítulos enteros a sus diversos desastres).

    \section*{\centering Expansión de los Kachax}

    Después de su éxito en Myari, los Kachax comenzaron a expandirse a otros sistemas. En sus primeros años, siguieron modelos similares a los de Myari, en los que se volvieron prácticamente indispensables para varios gobiernos. Los mundos aliados con los Kachax pronto se volvieron ricos y poderosos. Fue a través del comercio de mineral de uranio y esclavos con los Kachax que los Caudata reptilianos se convirtieron en la principal potencia galáctica que son hoy en día. Otros mundos que intentaron sancionar a los Kachax por sus acciones, ya fueran militares u otras, se encontraron bloqueados por la Armada Kachaxiana hasta que se sometieron a las demandas de los extraterrestres (a menudo a expensas del mundo bloqueado). Los Sindrin, incluso ahora, 348 años estándar después del fin del bloqueo de su mundo, aún no se han recuperado.

    A medida que su poder crecía, los Kachax comenzaron a expandirse hacia otros mundos e invadir y conquistarlos directamente si no podían resistirse. Como se registra en muchos textos históricos, su primera gran campaña de colonización contra los Yeen insectoides fue un gran éxito después de la toma de la Colmena Nikir, lo que resultó en la captura de la Reina Yeen. Su segunda campaña contra los Oden anfibios se logró famosamente en el lapso de 4 meses estándar en el año 2431, lo que llevó a la dominación total y la esclavitud de los Oden. Su tercera campaña prácticamente aniquiló las flotas de incursión de los Drixit

    que habían aterrorizado el extremo sur de la galaxia durante muchos años. Y así sucesivamente. Los infames regimientos Bluecoat de los Kachax eran respetados y temidos en todas partes.

    Por supuesto, se encontraron con resistencia, ya que muchas razas alienígenas buscaron evitar la dominación completa Kachaxiana. Esto llevó famosamente a la formación de la Alianza de los Bogir, Slek y Druuke para oponerse a la Supremacía Kachaxiana en el año 2702. La Guerra de la Supremacía se libró en los siguientes dos siglos y medio y vio combates en ocho mundos diferentes. A pesar de los valientes esfuerzos de la alianza, el ejército Kachaxiano era muy superior, habiendo perfeccionado el arte de disparar sus mosquetes hasta cuatro veces por minuto y habiendo perfeccionado el arte de formar sus filas para enfrentar perfectamente el ataque enemigo sin importar la situación. Solo la famosa Caballería Druuke era superior a sus contrapartes Kachaxianas, y esta ventaja se perdió para la alianza cuando los Druuke se pasaron a la Supremacía Kachaxiana en 2927 a cambio de plenos derechos en la Supremacía como ciudadanos y sin tener que pagar reparaciones por la guerra (un acuerdo que los Kachax, para su crédito, cumplieron completamente).

    Cuando el último puesto de avanzada Slek cayó en 2951 (como se representa a menudo en varios medios, como la brillante película "Alianza" y la no tan brillante "La Muerte de un Sueño"), la supremacía Kachaxiana quedó prácticamente asegurada. Parecía que, a través del comercio, el ingenio y el poder militar imparable, los Kachax habían pasado de ser un mundo de simples comerciantes a ser líderes del mayor Imperio que jamás había conocido la galaxia."

    \section*{\centering La Flota Expedicionaria 293 y el descubrimiento de los Terranos}

    Durante los siguientes 71 años estándar, la Supremacía Kachaxiana controló la galaxia. Toda la vida conocida en el 40\% de la galaxia se despertó, vivió y murió bajo el dominio Kachaxiano. Sin embargo, los Kachax seguían buscando expandir su influencia y enviaron múltiples expediciones a las regiones desconocidas del espacio.

    A principios del año 3012, la 293ª Flota Expedicionaria tropezó con un mundo de Categoría-9. Aunque en el límite de lo que muchas razas consideraban habitable, este mundo no solo albergaba vida, sino vida consciente. La monitorización inicial de los primitivos sistemas de comunicación del planeta determinó que este probablemente era una de las pocas colonias de una especie que apenas comenzaba a explorar las estrellas. Lo que impulsaría a una especie cuerda a intentar colonizar un mundo de Categoría-9 (solo un escalón por debajo del temido "Mundo de la Muerte" de Categoría-10) con instrumentos tan primitivos estaba más allá de la comprensión. En ese momento no se entendió que los colonos provenían de uno de estos Mundos de la Muerte de Categoría-10 y que para ellos este planeta (llamado "Janus", según se informa, en honor a uno de sus antiguos dioses) era un paraíso verde que de hecho se utilizaba para cultivar varios alimentos (sí, estas criaturas cultivaban y consumían la variada fauna y flora venenosa de un mundo de Categoría-9) para su planeta de origen, "Terra".

    Habiendo pasado una semana fuera del alcance de los primitivos escáneres de los Terranos, el Almirante de Flota Lord Blom ordenó que el mundo fuera conquistado en nombre de la Supremacía Kachaxiana. En un ataque relámpago, el Mundo de Janus fue tomado. Al ser considerados primitivos, los Terranos fueron capturados y esclavizados. De las cinco naves espaciales de la colonia, tres fueron destruidas en el ataque inicial mientras estaban en órbita y la cuarta fue destruida al intentar escapar del mundo. Solo una nave, la UNSS Explorer, escaparía de la campaña y solo porque el sentido del honor militar del Lord Blom sentía que no habría gloria en perseguir una nave en fuga. Si la 293ª Flota Expedicionaria hubiera pasado más tiempo analizando la cultura de los Terranos durante su reconocimiento inicial, es posible que no hubieran permitido que el Explorer regresara a territorio Terrano. Pero el ansia del Almirante por la gloria de una campaña rápida solo le permitió 3 días estándar para la exploración antes de la invasión.

    Los esclavos Terranos capturados fueron llevados de vuelta al espacio Kachaxiano para ser vendidos, y suscitaron mucho interés. El hecho de ser una especie recién descubierta no fue la única razón de esto. Los Terranos eran criaturas parecidas a simios bípedos. Su piel suave variaba en diferentes tonos de blanco, beige, marrón y negro. Aunque eran de estatura baja en comparación con la mayoría de otras especies (con una media de solo 5.5 kiks), pesaban el doble que un macho Kachax adulto, con una fuerza a juego. Esto llevó a los eruditos a especular que los Terranos habían evolucionado en un mundo de alta gravedad, lo que se confirmaría más tarde. Esto los convertía en una opción ideal para trabajos manuales. Además, los Terranos eran de origen mamífero y su cabello era suave y casi lujoso al tacto, lo que, si se recolectaba en cantidades suficientes, podría convertirse en un excelente material. También se supuso que la leche Terrana podría ser una mejor alternativa a la de la bestia bakit (que es prácticamente incomible para hasta 142 especies diferentes). Todos estos factores llevaron a inversores y funcionarios gubernamentales a financiar más expediciones al espacio Terrano para conquistar su raza y ponerlos en granjas de esclavos, como se había hecho antes con otras razas como los Slek.

    A finales de 3014, el Almirante Lord Blom estaba preparando su armada y sus fuerzas de invasión en Janus cuando una Flota Terrana de dieciséis naves emergió del Warp.

    \section*{\centering La Ira de los Terranos}

    Inicialmente, la llegada de los Terranos sorprendió pero no preocupó al Almirante de Flota Lord Blom. Especuló que los Terranos habían enviado todas sus naves para retomar su colonia, lo que haría que la conquista del espacio Terrano fuera más fácil. Además, si lograba capturar una nave Terrana (su único fracaso durante la campaña inicial), podría averiguar la ubicación de otros mundos Terranos. Con entusiasmo, envió sus naves al encuentro y ordenó a su infantería que se preparara para cualquier aterrizaje de Terranos.

    Lord Almirante Blom no consideró dos factores vitales. En primer lugar, debido a la sorpresa del ataque, su invasión inicial no había encontrado resistencia militar Terrana. En segundo lugar, las naves estelares encontradas en la Conquista de Janus eran naves coloniales con poca capacidad de defensa, no los acorazados especialmente diseñados que se acercaban a él.

    La flota Kachaxiana se dirigió hacia los Terranos. Más esbeltas y rápidas, las naves Kachaxianas superaron fácilmente a los mastodontes de los Terranos. Como moscas de sangre que pican, las naves Kachaxianas dispararon sus baterías contra los acorazados. Sin lugar a dudas, los diseños de las naves Kachaxianas eran estéticamente más atractivos. El plan de Blom era hacer que sus naves pasaran junto a los Terranos en línea recta para proporcionar un bombardeo de costado prácticamente continuo. Era una estrategia que había servido a los Kachax durante siglos.

    Entonces las naves Terranas dispararon.

    Aunque eran lentos ladrillos comparados con las naves Supremacy más elegantes, los acorazados Terranos también impactaron con la fuerza de un dios enojado. La primera nave en comprender esto fue la KSS Espíritu Perseverante, que fue destrozada en cuestión de segundos bajo una lluvia de fuego. Tres naves más fueron destruidas en rápida sucesión antes de que Blom ordenara a sus naves romper la formación. La potencia de los cañones Terranos hacía prácticamente imposible acercarse lo suficiente para enfrentarlos. Imperturbables, las naves Terranas comenzaron a aterrizar en la superficie de Janus.

    A medida que los Terranos aterrizaban, el 214º Ejército Kachaxiano Bluecoat formó filas y esperó el asalto de los Terranos. Los alienígenas podían haber tenido un gran éxito usando la fuerza bruta contra la flota Kachaxiana, pero aquí se esperaba que la marea se diera la vuelta. Después de todo, se consideraba que estos Terranos eran salvajes primitivos y se esperaba que corrieran gritando hacia las líneas Bluecoat en el momento en que dispararan sus mosquetes, como había ocurrido en tantas guerras contra otros primitivos. Aquí era donde la legendaria distancia de 120 kiks de las armas de fuego Kachax y sus famosos cuatro disparos por minuto les ganarían la batalla. Tambores sonaron para marchar a las tropas a la batalla mientras las cámaras de noticias transmitían la gloria de esta victoria a toda la Supremacía.

    Sin duda, los Terranos eran (y posiblemente todavía lo son) salvajes. Pero no se sabía hasta qué punto. Si se hubiera sabido, es probable que la Supremacía hubiera ordenado una cuarentena del espacio Terrano. Verás, los Terranos tienen una larga historia guerrera. Estudiando diversas fuentes Terranas, he aprendido que desde los albores de su civilización, las tendencias tribales, la agresión y la naturaleza competitiva de los Terranos significaban que hasta la llegada de los Kachax, los humanos no habían experimentado más de dos décadas estándar sin una guerra. Gran parte de su desarrollo tecnológico se debió a conflictos (las primeras naves espaciales Terranas se diseñaron a partir de la base de las armas Terranas, véase el cohete V2 en los diagramas).

    Cuando la mayoría de las razas inventó el cohete (generalmente en forma de fuegos artificiales o experimentando con polvos químicos), dijeron: "Me pregunto cómo podemos usar esto para ir más rápido".

    Cuando los Terranos inventaron el cohete, su primer pensamiento fue probablemente: "Me pregunto cuántas personas puedo destrozar con esto".

    Como resultado, la tecnología de guerra Terrana se volvió cada vez más avanzada. La cultura Terrana se basa tanto en la guerra que gran parte de sus medios de comunicación cuenta historias de guerra e incluso sus deportes y juegos simulan la guerra, con deportes que fomentan la cohesión de grupo para marcar un gol (véase: Fútbol en referencias), mientras que tienen simulaciones informáticas que replican escenarios de batalla con géneros dedicados a ser un solo soldado (FPS) y liderar un ejército (RTS). Los Kachax pueden haber dominado la guerra, pero los Terranos la estaban revolucionando. Y los Kachax estaban a punto de descubrirlo.

    En lugar de avanzar en hordas sin sentido, la infantería Terrana se movía en pequeños grupos móviles. Esto desconcertó a los oficiales Kachax, pero aún esperaron a que los Terranos entraran en el rango de 120 kiks de sus mosquetes. Luego, los Terranos se detuvieron a 300 kiks y levantaron sus armas. Seguro que no esperaban que sus armas fueran efectivas a esa distancia.

    Luego, los Terranos dispararon sus armas. Disparando no uno, dos, tres o cuatro, sino cientos de rondas por minuto, diezmando a los Bluecoats mientras sus filas apretadas impedían una rápida huida. Peor aún fueron los "tanques" Terranos, monstruosos cañones acorazados que destrozaron la caballería Druuke que cargó contra los Terranos. Además, los Terranos llevaban ropa verde oscuro que los ayudaba a mezclarse con los árboles y los hacía prácticamente invisibles a la extrema distancia a la que sus armas de fuego podían disparar.

    Comprendiblemente, las fuerzas Kachax entraron en pánico y comenzaron a huir. El General Vik Gir y su personal de nobles Kahax esperaron a que los Terranos se acercaran, entendiendo que probablemente serían capturados y rescatados como era costumbre para la nobleza. Cuando un "escuadrón" (nombre dado a un pequeño grupo de soldados Terranos) se acercó al personal del General, Vik Gir y sus hombres fueron brutalmente acribillados a tiros en una lluvia de fuego que hasta una hora antes se creía que solo podía ser producida por miles de soldados disparando a la vez. Solo el camarógrafo de la estación de noticias sobrevivió, transmitiendo sin saberlo todo lo que estaba sucediendo a una galaxia horrorizada.

    En órbita, la situación era mucho peor. La flota del Almirante de Flota Lord Blom yacía hecha jirones, con varios capitanes desertando y llevando sus naves de regreso al espacio Kachaxiano. Sin duda impulsado por la ira, Blom ordenó a su nave insignia KSS Victoriosa que se enfrentara a las naves Terranas, decidido a derribar al menos una.

    Cuando la KSS Victoriosa se acercó, la nave Terrana UNSS Odisea disparó una salva, detonando los motores Kachaxianos y dejando a la Victoriosa a la deriva. Usando un dispositivo parecido a un arpón, los Terranos anclaron las dos naves juntas y enviaron sus "Escuadrones" a bordo. En el lapso de una hora, la nave estaba bajo control Terrano. Varios informes detallan cómo el Almirante de Flota Lord Blom se suicidó como una salida honorable y para evitar ser capturado. Otros dicen que uno de sus esclavos aprovechó esta oportunidad (en un momento de vacilación por parte de Blom) para arrebatarle la pistola al Almirante y dispararla en su cráneo.

    La Batalla de Janus resultó en la pérdida de 20,541 Bluecoats, 34,702 miembros del personal naval y 18 Cruceros de Batalla en el lapso de un solo día. Peor aún, las bajas Terranas ascendieron a menos de 150 miembros del personal y daños extensos en dos de sus acorazados (aunque estos pudieron repararse). Y con la captura de la KSS Victoriosa y su tripulación, los Terranos tenían acceso a todos los datos disponibles sobre la Supremacía. Todo esto se transmitió en directo a la galaxia.

    Sin embargo, aún no habíamos visto el arma más temida de los Terranos.

    \section*{\centering Represalia}

    En toda la galaxia, estalló el pánico y los disturbios a medida que se mostraba la carnicería en Janus. Se desplegaron muchos regimientos Bluecoat para sofocar las multitudes. Mientras tanto, el gobierno Kachaxiano se sentó para discutir la situación. Estaba claro que los Terranos habían sido subestimados. Tal vez, si se movilizaba toda la fuerza de la Supremacía, los Terranos podrían ser abrumados por la pura superioridad numérica. Nunca en la historia de la Supremacía una fuerza había sido derrotada de manera tan contundente.

    Fue entonces cuando las sirenas resonaron sobre el Mundo de Xixis Prime, ya que una flota Terrana apareció en órbita. La Supremacía apenas podía creer su suerte. Xixus Prime era uno de los principales puntos de partida militares de los Kachax. Más de 75 cruceros de batalla estaban estacionados en órbita y un millón de Bluecoats esperaban en la superficie. Mientras tanto, los Terranos tenían solo 20 naves. Si había algún lugar que detuviera a los Terranos, sería este.

    Fue entonces cuando los Terranos abrieron comunicaciones. Un hombre Terrano de mediana edad con ojos azules que parecían atravesar la cámara se enfrentó a los oficiales de la Supremacía. Llevaba lo que parecía ser un traductor automático capturado en su oído.

    "Soy el Almirante Callum Harker de la Armada de las Naciones Unidas. En respuesta al ataque no provocado de su 'Supremacía'", el Terrano dijo esta palabra con un desprecio audible, "las Naciones Unidas han decidido imponerles este ultimátum. Liberen a todos los rehenes humanos y entréguense según nuestros términos. Tienen 24 horas para cumplir." (Apareció una cuenta regresiva en la pantalla con símbolos desconocidos que comenzaron a contar a intervalos regulares). "Si no responden o si percibimos cualquier acto de agresión, nos veremos obligados a tomar medidas drásticas". El mensaje terminó.

    Los Kachaxianos estaban indignados. ¡Cómo se atrevían estos salvajes a dictar los términos de rendición! Inmediatamente, y tal vez de manera imprudente, los Kachaxianos se movieron para interceptar la flota Terrana. Los Terranos respondieron simplemente disparando seis veces.

    Los Kachaxianos han apodado a estas armas como "Fuego Infernal de Brutir", en honor a su dios vengativo de la guerra de tiempos antiguos. Mi propio pueblo, los Feritus, las llamaron "El Fin de Todas las Cosas". Los Terranos simplemente las llaman "Nukes".

    Se dispararon seis veces y pasaron junto a la flota que se acercaba y se estrellaron contra la superficie de Xixis Prime. Utilizando información de los registros de la tripulación capturada y las computadoras de la KSS Victoriosa, los Terranos habían aprendido las posiciones de los objetivos estratégicos más vitales en la superficie de Xixis Prime. Astilleros, ciudades de fábricas, fuertes principales y puntos de partida importantes. En cuestión de segundos, cada uno fue consumido por el fuego nuclear que vaporizó todo a su alrededor en un fuego atómico y dejó la tierra arrasada durante décadas con la taint de la radiación. ¡Los Terranos habían convertido la energía nuclear en un arma! ¡No eran simplemente salvajes, eran dementes!

    Una vez más, el Almirante Terrano apareció en la pantalla. "Esa fue su última advertencia. No nos hagan hacer algo de lo que podríamos arrepentirnos." A medida que la comunicación se cortaba una vez más, abundaron los informes de flotas Terranas que aparecían sobre otros seis mundos a lo largo de la frontera. Todas mostraban signos de llevar las mismas armas que se habían desencadenado en Xixis Prime. Pero esta vez estaban apuntando a objetivos no militares. Los ciudadanos entraron en pánico y comenzaron a sublevarse. Las naves comenzaron a evacuar personas de los mundos y, en muchos casos, regimientos enteros de Bluecoat desertaron o incluso se amotinaron contra sus superiores. Si no hubiera sido por el alcance limitado de las naves Terranas, es probable que se hubieran acercado lo más posible a Kachax. La Supremacía no tenía elección.

    En su histórico discurso al público, el Ministro Alsen anunció que todos los esclavos Terranos debían ser devueltos inmediatamente a Janus. En el siguiente Tratado de Janus de 3015, la Supremacía se vio obligada a respetar la soberanía del espacio Terrano y no se le permitía enviar naves de ningún tipo al espacio Terrano sin una escolta armada. También se prohibió estrictamente el comercio de armas Terranas con extraterrestres.

    Esto resultó ser la campanada de la muerte para la Supremacía. De repente, muchos mundos que habían sufrido bajo su control comenzaron a alzarse, sintiendo debilidad. Los Kachax ya no parecían una fuerza imparable. Se les había hecho sangrar y si algo podía sangrar, podía ser asesinado. Aunque las legiones Kachaxianas habían logrado sofocar la revuelta ocasional en el pasado, sus fuerzas de repente se encontraron bajo una presión creciente a medida que más y más sistemas se rebelaban contra ellos. Para 3031, los Kachax se retiraron a su mundo natal y quedaron destrozados. Una vez gobernantes de la galaxia, fueron derrotados y esclavizados.

    En los años intermedios desde entonces, los mundos anteriormente bajo el control Kachaxiano han estado trabajando para restablecerse. Se han estado formando nuevos lazos y alianzas. Ha habido conflictos, pero también grandes triunfos y avances. Sin embargo, todos nosotros miramos con temor a la frontera occidental.

    Porque aunque los Terranos actualmente parecen contentos colonizando mundos que la mayoría de los seres conscientes cuerdos ni siquiera considerarían, recordamos lo que sucedió cuando decidieron luchar por un mundo. A pesar de que están contentos de comerciar con nosotros, recordamos las armas que mantienen ocultas. A pesar de que no ha habido un conflicto importante entre los Terranos y ninguna otra especie conocida desde la Guerra de Janus, todos recordamos el momento en que los Terranos rompieron el Imperio más antiguo y grande que la galaxia haya conocido en menos de 2 años. Todos recordamos el momento en que los Terranos fueron a la guerra.

    Y rezamos para que no se les dé motivo para hacerlo de nuevo.

\end{document}