\documentclass[spanish,12pt,a4paper,oneside,titlepage]{book}
\usepackage[T1]{fontenc}
\usepackage[left=2cm, right=2cm, top=2cm, bottom=2cm]{geometry}
\usepackage{amssymb}
\usepackage{graphicx}
\usepackage[spanish]{babel}
\usepackage{incgraph,tikz}
\usepackage{float}

\usepackage{titlesec}
\titleformat{\chapter}[display]
{\centering \normalsize \huge \color{black}}{\textbf{Capítulo \thechapter}}{11pt}{}

\setlength{\parskip}{2mm}

\title{Por que los humanos evitan la guerra}
\author{Daniel Pascap}
\begin{document}
    \maketitle
    \tableofcontents
    \chapter{Los cobardes}
    Los humanos se suponía que eran cobardes.

    El registro de especies de la Federación Galáctica los tenía listados como un 2 de 16 en el índice de agresión. Nuestras interacciones con la Unión Terrana hasta este punto respaldaban esas conclusiones. No habían librado guerras entre ellos en siglos y habían formado un gobierno mundial unificado antes de lograr los viajes más rápidos que la luz. Respondieron con entusiasmo en lugar de hostilidad al primer contacto, a diferencia de muchas especies.

    La Tierra había resuelto todas sus disputas a través de la diplomacia y el compromiso desde que se convirtió en miembro oficial de la Federación. Por ejemplo, hace unos años, los expansionistas Xanik reclamaron una colonia minera terrana como su territorio. La Federación se preparó para un conflicto menor, ya que esperaban que los humanos defendieran su puesto avanzado. Pero los humanos simplemente se encogieron de hombros y acordaron ceder el planeta a cambio de una pequeña tarifa anual. En lugar de ir a la guerra, los terranos terminaron siendo importantes socios comerciales de los Xanik.

    También hubo un incidente en el que los paranoicos Hoda'al arrestaron a embajadores terranos bajo cargos de espionaje. Encarcelar a diplomáticos sin evidencia alguna fue una clara provocación a la guerra, pero los humanos no hicieron nada. ¡Ni siquiera atacaron la instalación donde se encontraban sus representantes! Simplemente abrieron negociaciones en un canal clandestino con los Hoda'al y organizaron un intercambio de prisioneros, canjeando a algunos contrabandistas por su gente.

    Las opiniones sobre los humanos variaban según a quién le preguntaras. Algunos en la Federación encontraban su pacifismo loable y apreciaban su diplomacia equilibrada. Otros pensaban que era debilidad lo que los llevaba a evitar la guerra. Yo estaba en el último grupo; la única razón para no responder a insultos evidentes con agresión era que no tenían la inteligencia ni la fuerza para hacerlo.

    Cuando llegaron los Devoradores, las tres especies más militaristas de la galaxia (según el índice de agresión) se unieron para enfrentar su avance. No sabíamos mucho sobre ellos, pero los llamamos Devoradores porque su única misión era drenar la energía de las estrellas. No puedo decirte por qué harían algo así. Sea cual sea su motivo, tomarían un sistema por la fuerza, lo agotarían y pasarían al siguiente.

    Nuestra flota, la mejor que la Federación tenía para ofrecer, sufrió grandes pérdidas cuando chocamos con destructores enemigos. Luchamos con todas nuestras fuerzas, y no importó. Nuestras armas apenas parecían rasguñar sus naves. Fue una decisión difícil, pero ordené lo que quedaba de la flota que se retirara. Por mucho que necesitáramos detenerlos, perderíamos toda la armada si nos quedábamos más tiempo.

    Envié una señal de socorro, transmitiendo nuestra sombría situación y suplicando refuerzos. Había otras especies con fuerzas militares menores, pero aún poderosas dentro de la Federación. Pero mi solicitud fue recibida con silencio. Ni uno solo de esos cobardes se ofreció como voluntario para ayudar. Al enterarse de nuestra derrota, supongo que decidieron huir y arreglárselas por sí mismos.

    Pensé que estábamos solos, hasta que detectamos naves humanas saltando a nuestra posición. Qué irónico, los únicos que vinieron en nuestra ayuda fueron los que se consideraban fácilmente influenciables en la galaxia. Según nuestros sensores, solo había cinco de ellos, lo que no era ni de cerca suficiente para librar una batalla. Un espectáculo patético, pero era más de lo que habían enviado las otras potencias de la Federación.

    "Señor, los Terranos nos están contactando. ¿Qué creen que van a hacer, hablar al enemigo hasta la muerte?" bromeó el Primer Oficial Blez.

    Escuché algunas risas de mi tripulación, pero los callé rápidamente. "Necesitamos toda la ayuda que podamos conseguir. En pantalla."

    Un humano de cabello oscuro apareció en la pantalla. "Nave de la Federación, soy el Comandante Mikhail Rykov de la Unión Terrana. Estamos aquí para ayudar en todo lo posible."

    Incliné la cabeza con gratitud. "Gracias por venir, Comandante Rykov. Soy el General Kilon. Únase a nuestra formación y ayúdenos a cubrir nuestra retirada."

    "¿Retirada?" El comandante humano parpadeó unas cuantas veces, luciendo confundido. "Nuestras intenciones son enfrentar y aniquilar al enemigo."

    "¿Con cinco naves? Con todo el respeto, los Devoradores se cuentan por miles y aplastaron nuestra flota de igual magnitud. No esperaría que una especie pacífica como la suya comprendiera la guerra, pero es de su interés seguir nuestro liderazgo", dije.

    El Comandante Rykov parecía aún más confundido. "¿Creen que los humanos son una especie pacífica? ¿Qué diablos? ¿Por qué pensarían eso?"

    "Bueno... nunca pelean con nadie. Resuelven todo con palabras. Los humanos son la especie con la calificación más baja en el índice de agresión", respondí.

    "Ya veo. La Federación nos ha juzgado mal en ese aspecto. ¿Saben por qué evitamos la guerra, General?"

    "¿Porque no creen que pueden ganar? ¿Miedo?"

    El humano rió con ganas. "No, es porque sabemos quiénes somos. Lo que somos capaces de hacer. Y nadie lo merece aún."

    La idea de que los terranos hicieran amenazas ominosas habría sido una broma para mí hasta ahora, pero algo en el tono de Rykov me dijo que creía lo que decía con convicción. Esto era un claro caso de delirio surgido de la falta de experiencia en la guerra interestelar. Los Devoradores harían tontos de los terranos y los castigarían por su exceso de confianza. Sin embargo, si el Comandante realmente quería enviar a sus hombres a una masacre, no lo detendría.

    "Si insisten en luchar, ciertamente no me opondré. Pero sepan que están solos, nosotros nos retiramos. ¿Cuál es su plan?" pregunté.

    "Trajimos una bomba de nanobots que desarrollamos. Nunca la hemos usado antes, ya que en aproximadamente el cinco por ciento de las simulaciones, no se detienen con entidades localizadas y consumen toda la materia en el universo", el Comandante Rykov dijo esto de manera demasiado casual para mi gusto. "Pero las programamos para autodestruirse después de unos segundos, lo que probablemente funcionará. Alférez Carter, dispara contra el enemigo en cinco segundos."

    Mis ojos se agrandaron de alarma. "Espera, espera, acabas de decir que podría destruirlo todo..."

    La nave insignia terrana disparó un misil antes de que pudiera pronunciar otra palabra para detenerlos. Al principio, pensé que habían fallado su objetivo. El proyectil atravesó la flota Devoradora sin conectar con ninguna nave. Luego, estalló en la retaguardia de la formación, y todo se volvió un caos.

    El espacio mismo pareció estremecerse cuando una explosión arrasó con cualquier cosa en su cercanía. La fuerza fue tan poderosa que nuestros sensores solo proporcionaron un mensaje de error como medida. Al menos un tercio de la flota Devoradora se vaporizó al instante, ya que una cantidad improbable de energía y calor los convirtió en sopa de metal. No había forma de que los ocupantes de esas naves sobrevivieran a eso.

    Las naves enemigas más alejadas del epicentro sobrevivieron a la explosión inicial, aunque muchas de ellas sufrieron daños graves. Pero una fuerza invisible pareció diseccionarlas lentamente; solo pude mirar con incredulidad mientras los poderosos cruceros se desintegraban poco a poco. Supongo que la bomba había liberado una serie de nanobots que atacaron la estructura de las naves a nivel molecular.

    Los Devoradores apenas sabían qué les había golpeado. Para cuando pensaron en contraatacar, ya no quedaba nada con qué contraatacar. Su arsenal se evaporó en cuestión de segundos, y sin duda, su personal sufrió el mismo destino. Donde antes había un ejército imparable, ahora solo quedaba espacio vacío.

    Los humanos habían desatado una ola de destrucción que no tenía rival en toda mi carrera militar, con un solo misil. El horror recorrió mis venas al pensar que algún día podrían dirigir sus monstruosas armas contra la Federación. No había forma de defenderse contra tales creaciones diabólicas.

    El índice de agresión necesitaba una actualización. La especie que inventaría armas como esas no era un 2. Mirando a mi tripulación, vi reacciones asombradas y atónitas que reflejaban las mías. Si alguna vez se volvían hostiles, los humanos representaban una amenaza de nivel máximo. Podrían más que probablemente arrasar toda la galaxia sin despeinarse.

    "Ahora que eso está resuelto. Deberías habernos invitado a la fiesta desde el principio", sonrió el Comandante Rykov. "Te diré algo, General, la próxima vez que nos veamos, nos debes una cerveza."

    Fruncí el ceño. Los humanos podrían pedir mucho más que una bebida si quisieran. "Sí, creo que podemos hacer eso."

    El Comandante Rykov terminó la llamada, y vi cómo las naves terranas saltaban de nuevo al hiperespacio. Todavía estaba tratando de asimilar todo lo sucedido, y me preguntaba cómo iba a ponerlo en palabras para el informe de combate. La Federación no tenía idea de quiénes eran realmente los terranos, pero me aseguraría de que lo supieran.

    Y mientras repasaba los acontecimientos del día en mi mente, comprendí. Finalmente entendí por qué una especie tan poderosa evitaba la guerra.

    Los humanos evitan la guerra porque les sería demasiado fácil ganarla.

    \chapter{Ula POV}

    El Senado de la Federación esperaba lo peor cuando llegó el mensajero.

    Siguiendo las costumbres galácticas, la nave más rápida fue enviada antes de la flota para proporcionar un informe de primera mano de la batalla a los embajadores. La mirada aterrorizada en el rostro del joven alférez Jatari cuando entró en la cámara del Senado parecía confirmar los temores de todos.

    Recordé la transmisión que habíamos recibido hace solo unas horas, detallando la sombría situación de aquellos que se habían enfrentado a los Devoradores. Los números de bajas confirmadas ya habían sido elevados, y sin que ningún miembro de la Federación enviara refuerzos, podríamos estar hablando de una tasa de bajas de hasta el 90%.

    Como Presidente, había intentado persuadir a las especies de agresión intermedia para que ofrecieran ayuda, pero todos se negaron rotundamente. Si tuviera el poder para obligarlos a ir, lo haría. Todos conocíamos la estela de destrucción que dejaban los Devoradores a su paso, pero no teníamos más opción que detenerlos. Nos empujarían al borde de la extinción si les permitíamos avanzar por nuestra galaxia.

    Sin embargo, hubo algunos puntos extraños en el comportamiento del mensajero. A medida que se acercaba al podio, clavó la mirada en la Embajadora Terrana Nikki Johnson y tragó nerviosamente. Noté que le temblaban las manos. Los Jatari eran una raza orgullosa, impulsada por el honor, que había presenciado los horrores de la guerra una y otra vez. Nunca antes había visto a uno de ellos regresar a casa con aspecto de haber visto un fantasma. Y ¿por qué su fijación estaba en los pacíficos humanos, de todas las razas?

    "Uh, hola, S-Senadores. Soy el Alférez Telus". La mirada del heraldo no se apartó de la Embajadora Johnson. "Los Devoradores han sido derrotados. No sobrevivió ni una sola de sus naves".

    Murmuros sorprendidos se extendieron por la asamblea. Yo también estaba perplejo; la correspondencia anterior había pintado un panorama desolador para nuestros hombres. Si realmente se había producido un giro tan drástico de los acontecimientos, necesitábamos saber cómo había sucedido. Cualquier táctica que la flota hubiera empleado podía transmitirse a otros comandantes para futuros encuentros.

    Una rápida mirada por la habitación reveló que la mayoría de los representantes estaban en estado de confusión. Pero la Embajadora Terrana estaba sonriendo, con un brillo depredador en sus ojos. Había algo en su expresión que me perturbaba en lo más profundo de mi subconsciente.

    Me levanté, decidido a restaurar el orden. "¡Silencio! ¿Cómo es posible? Por favor, explíquese".

    "Bueno, Señora Presidenta... fueron los humanos. Solo enviaron unas pocas naves en nuestra ayuda, pero... construyeron algo terrible". La voz del Alférez Telus bajó hasta apenas más que un susurro. "Fue como si hubieran aprovechado una supernova. Nunca en mi vida había visto tanta destrucción".

    El caos se desató cuando exclamaciones de sorpresa se elevaron a un crescendo, y todas las miradas se dirigieron hacia la Embajadora Johnson. Yo también tenía dudas sobre esta versión de la batalla; ¿los humanos, con alguna arma terrible capaz de destruir a los Devoradores? Era de conocimiento común que evitaban la guerra a toda costa.

    El Embajador Xanik Cazil rió y levantó un dedo para hablar. "Con todo respeto, los humanos no son una especie combativa. Astutos, inteligentes, avaros... son todas esas cosas. Pero si tuvieran armas que pudieran exterminar a los Devoradores, serían algo más que charlatanes y diplomáticos. Ya gobernarían la galaxia en este momento".

    Los Xanik estaban en las filas superiores de las especies agresivas, pero también eran el principal socio comercial de la humanidad. La Unión Terrana los había conquistado con su disposición a vender cualquier cosa, a cambio de un precio, y a pesar de sus filosofías divergentes sobre la violencia, los dos poderes se habían convertido en estrechos aliados.

    "Estás equivocado. Lo vi con mis propios ojos", respondió el Alférez Telus. "La verdad sobre la humanidad es que son asesinos. Son peligrosos. El General cree que deberíamos buscar su amistad, pero no estoy seguro de estar de acuerdo. No confío en ellos".

    Dirigí mi mirada a la Embajadora Johnson. "Deberíamos dejar que la representante terrana responda. ¿Es esto cierto?"

    La Embajadora Johnson suspiró con cansancio. "Sí, es cierto. La Tierra tiene muchas armas de último recurso guardadas. Somos muy hábiles en la guerra, pero intentamos encontrar una manera diferente".

    "¿Por qué nos presentaron una imagen falsa de su especie?", le pregunté. "Hablan de paz y, sin embargo, han estado ocultando las armas más poderosas de la galaxia".

    "Nunca deseamos usarlas", dijo. "Su índice de agresión: las especies de alta agresión a menudo son territoriales y buscan conquistar. Si la Federación hubiera examinado nuestra historia, habrían visto que alguna vez fuimos así. Perdimos millones de vidas en guerras entre nuestras facciones, y nos cansamos de tanta derramamiento de sangre.

    La humanidad ha intentado ser mejor. Nuestra naturaleza destructiva e impulsiva todavía está ahí, simplemente la hemos enterrado profundamente. Ve, somos la única especie agresiva que tiene un fuerte sentido de empatía. Luchamos constantemente con esa dualidad. Nos controlamos con reglas y, en su mayor parte, elegimos el bien.

    Pero conocemos las profundidades de la depravación que existen. Sabíamos que algún día, alguien verdaderamente malvado vendría... y tendríamos que ser peores".

    Asimilé sus palabras, mientras mi mente seguía dando vueltas. ¿Una guerra con su propia especie que tenía millones de bajas? Incluso los conflictos más graves en la historia temprana de los Jatari no llegaban a tener 200,000 muertos, ¡y ellos eran un 15 de 16 en la escala de agresión! La guerra más sangrienta que habíamos conocido anteriormente no se acercaba ni de lejos a la historia vil que había descrito la Embajadora Johnson.

    Una especie con tanta propensión a la violencia debería haberse autodestruido. No había forma de que pudieran formar una sociedad funcionante. ¡Y mucho menos pensar que actuaban como pacificadores galácticos! Era difícil conciliar mis experiencias con diplomáticos humanos civilizados y el pasado vil que había descrito la Embajadora.

    Sin importar cuánto los humanos afirmaran poder controlar su salvajismo, no podíamos confiar en ellos. Una especie con un impulso tan fuerte hacia la violencia podía fácilmente apuñalarte por la espalda en un momento de ira y no pensar dos veces en ello.

    Honestamente, si no temiera represalias, habría propuesto en ese momento expulsar a la Unión Terrana de la Federación. Pero, incluso si era jugar con fuego, probablemente era mejor tenerlos de nuestro lado que tener sus cañones dirigidos contra nosotros. Sin embargo, tendríamos que vigilarlos mucho más de cerca.

    Forcé una expresión neutral. "Nos salvaron de un enemigo al que no podíamos vencer por nosotros mismos. Les debemos una gran deuda. Llevará un tiempo para que la Federación considere completamente lo que nos acaban de contar, pero les agradecemos por poner fin a la guerra".

    Los ojos de la Embajadora Johnson se endurecieron. "La guerra no ha terminado, Señora Presidenta. Derrotamos a una flota, pero los Devoradores enviarán más si no son eliminados. Y solo volverían más fuertes. La humanidad no espera su bendición, pero les pedimos perdón por lo que estamos a punto de hacer".

    "¿Qué... qué están a punto de hacer?" pregunté con cautela.

    "Iremos a atacar su mundo natal con bombas de antimateria, sin sobrevivientes. Es una solución permanente. Puede que no sea bonito, pero no vemos otras opciones para poner fin al terror que someten al resto del cúmulo", respondió.

    Incluso las especies más agresivas parecían horrorizadas ante la sugerencia. Noté que los embajadores más cercanos a la humana se alejaban, como si tuvieran miedo de que les mordiera.

    Sacudí la cabeza con fervor. "¡Eso es genocidio! La Federación no puede aceptar la erradicación de toda una especie; por favor, intentemos negociar una tregua. Debemos agotar los medios pacíficos antes de siquiera considerar un ataque como este".

    "No se puede razonar con alguien que solo quiere destruirte. Matar o ser matado". La Embajadora Johnson se levantó de su asiento, recogiendo sus pertenencias. "¿Cuántas especies inocentes ya han perecido a sus manos? Por lo que a nosotros respecta, es mejor que ellos que nosotros".

    La representante terrana salió del edificio, despidiéndose de la Embajadora Cazil mientras se marchaba. No podía comprender cómo cualquier ser pensante podría estar tan tranquilo y despreocupado ante la perspectiva de arrasar un planeta, incluso el de una raza parasitaria como los Devoradores.

    Me pregunté si al menos deberíamos intentar interponernos en el camino de los humanos. Es poco probable que pudiéramos detenerlos, pero al menos podríamos decir que lo intentamos.

    Las cosas eran más sencillas cuando creíamos que eran pacíficos. Parte de mí deseaba que esa mentira hubiera durado un poco más. Ya extrañaba a nuestros amigos pacifistas.

    \chapter{Kilon POV}

    Teme que los humanos pudieran atacar en cuanto nuestras naves entraran en el sistema Sol, pero el hecho de que todavía estuviéramos aquí era una buena señal.

    El Senado de la Federación había votado por poco enfrentarse a los Terranos, con la Presidenta Ula siendo una de las partidarias más fervientes de la moción. Incluso con su influencia política, muchos representantes estaban indecisos acerca de tomar medidas. El destino que había caído sobre los Devoradores fácilmente podría ser el nuestro si provocábamos a los humanos.

    Sinceramente, creo que si fuera su propia especie la que estuviera siendo convocada a la acción, el Senado no habría aprobado la propuesta. Pero como siempre, asumían que los Jatari, los Xanik y los Hoda’al harían su trabajo sucio, mientras ellos se quedaban observando desde la seguridad de sus oficinas.

    No estaba muy emocionado por liderar esta misión. Después de todo, estábamos arriesgando vidas de la Federación para proteger a las mismas personas que habían intentado destruirnos. Si bien la solución de los Terranos era extrema, al menos podía entender de dónde venían. Pero sería deshonroso rechazar una orden directa; lo último que quería era que me etiquetaran como traidor y cobarde.

    Además, si yo comandaba la flota, al menos sería lo suficientemente sensato como para no cargar en batalla contra un ejército superior. No estaba seguro de que mis colegas, que no habían presenciado la acción de las armas humanas de primera mano, fueran tan cautelosos. Especialmente dado que la mayoría de los oficiales Jatari veían la diplomacia como una admisión de debilidad.

    El Primer Oficial Blez miró hacia arriba desde su computadora mientras pasábamos el primero de los planetas exteriores. “Señor, estamos casi dentro del alcance de los misiles de la Tierra. ¿Deberíamos preparar nuestras armas?”

    “Nuestras órdenes son detenerlos, no atacarlos. Si entramos en una lucha directa, estamos condenados”, respondí. “Esperemos que a los humanos todavía les guste hablar. Llama al Comando Terrano.”

    Blez abrió la boca para discutir, pero luego pensó mejor. Ingresó silenciosamente algunas órdenes en su terminal, murmurando para sí mismo. Los pocos momentos que la llamada no fue respondida fueron angustiantes; temía que los humanos simplemente nos ignoraran. El alivio me inundó cuando una cara familiar parpadeó en la pantalla de visualización.

    El Comandante Rykov no se veía bien. Su cabello negro estaba desaliñado, su uniforme estaba arrugado y las ojeras se habían instalado debajo de sus ojos. Esto estaba lejos del hombre radiante y confiado que había venido a rescatarnos ayer. Parecía que debería estar descansando en lugar de estar en el puente de una nave, pero temía que señalar su condición causaría ofensa.

    El oficial humano miró fijamente a la cámara, con una expresión suplicante en su rostro. “General. Le recomendamos encarecidamente que dé la vuelta y se aparte.”

    “No puedo hacer eso. Lo que estás a punto de hacer está mal. La vida inteligente es sagrada y eliminar a una especie entera es un crimen contra la conciencia”, dije.

    “Los Devoradores apenas han demostrado ser conscientes. Me sorprende que, de todas las personas, te apresures a defenderlos”, reflexionó Rykov. “Ni siquiera ha pasado un día completo desde que eliminaron miles de tus naves. Tú y yo sabemos que si no hubiéramos aparecido, te habrían matado a todos sin pensarlo dos veces”.

    Me estremecí. “No me lo recuerdes. A pesar de todo lo que han hecho, no quiero ver a una especie entera masacrada. Eso nos hace igual de malos que ellos. Sus acciones no hacen que las tuyas sean correctas”.

    El Comandante Rykov suspiró. “Bueno, parece que estamos en un punto muerto. Supongo que nos atacarás si no retrocedemos, ¿verdad?”

    “Solo queremos hablar. No tienes por qué hacer esto. ¿Tu especie tiene un código moral, verdad?” Tomé una respiración profunda, tratando de recopilar mis pensamientos. “¿Y si hay personas inocentes, niños y civiles, en su planeta natal?”

    “Mira, no me gusta lo que estamos a punto de hacer, pero tengo mis órdenes. Ni siquiera sabemos si tienen civiles o si pueden mostrar emoción”.

    “Exactamente, no sabemos. ¿Cuál es el daño en esperar y obtener más información? ¿No quieres saber por qué están haciendo esto?”

    “Me gustaría entenderlo”, Rykov inclinó la cabeza, como si estuviera pensando. “Supongo que no haría daño reunir algo de inteligencia. Demonios, podría ser útil en el futuro. ¿Qué sugieres?”

    “¿Crees que pueden capturar una de sus naves? Necesitamos traer a uno de ellos con vida”.

    "Sí, creo que podemos hacerlo, General. ¿Qué te parece unirte a nosotros en persona en nuestro buque insignia? Preferiríamos estar juntos en lugar de enemigos."

    Pesé mis opciones. Esto fácilmente podría ser una especie de astucia de los humanos, atrayendo al oficial de mayor rango de la Federación a su sede solo para ser encarcelado. Sacarme de la imagen interrumpiría el mando de nuestra flota; era natural encontrar su oferta un poco sospechosa.

    Pero supuse que si las intenciones de Rykov hacia nosotros fueran maliciosas, no estaríamos teniendo este diálogo en primer lugar. Los Terranos tenían la capacidad de eliminar toda nuestra flota de un solo golpe, pero no nos habían disparado. En cualquier caso, aún le debía al Comandante una gran deuda por salvar mi vida. Lo menos que podía darle era un poco de confianza.

    “Estaré encantado de unirme a ti, Comandante”, respondí.

    La insinuación de una sonrisa se deslizó en el rostro de Rykov. “Excelente. Esperaremos tu lanzadera. Ven solo y desarmado. Por favor, ordena a tus naves que detengan su avance y nos permitan el paso.”

    La transmisión terminó y el Primer Oficial Blez habló inmediatamente. “Señor, no puedes estar pensando seriamente en ir allí”.

    Le fulminé con la mirada, sin apreciar que se cuestionara mis decisiones. “Tengo que hacerlo. Es nuestra única oportunidad de hablar con los humanos y será la primera vez que alguien haya hablado con el enemigo en persona”.

    Por supuesto, cualquier visión que pudiera obtener sobre la naturaleza de los Devoradores sería invaluable para la Federación. Pero estaría mintiendo si dijera que mi curiosidad no era personal. Me deleitaba con la posibilidad de exigir sus razones yo mismo. El asesinato en masa no era la solución, pero nuestros enemigos debían rendir cuentas por las pérdidas que habían causado.

    \begin{center} {\Huge $\divideontimes$} \end{center}

    Dos soldados terranos esperaban en la esclusa cuando mi lanzadera se acopló. La revisión que me hicieron se sintió un poco... invasiva, pero supongo que solo querían ser minuciosos. Una vez que estuvieron satisfechos de que no tenía armas encima, me condujeron al puente.

    En comparación con las naves de la Federación, la nave insignia terrana era realmente fea por dentro. Los pasillos eran estrechos y los colores eran una mezcla apagada de gris y blanco sucio. Era evidente que los humanos prestaban poca atención a los elementos de diseño, enfocándose en llenar la nave de guerra con la mayor cantidad de armas y estaciones posible. No pude evitar sentirme un poco claustrofóbico mientras navegábamos a través de una serie de pasillos sinuosos y escaleras estrechas.

    El pasillo finalmente se abrió en una cámara más amplia, que estaba alineada con filas de monitores de computadora y una pantalla holográfica en el centro. Mi primer pensamiento fue que nunca antes había visto un centro de mando tan desordenado en mi vida. Docenas de personas estaban ocupadas en el lugar, con tabletas en la mano, gritándose mutuamente. ¿Cómo podían funcionar en medio de tanto ruido y caos?

    El Comandante Rykov estaba en el centro de esta locura, estudiando una proyección de la flota Devoradora. Dos oficiales estaban a su lado; por lo que escuché, parecía que estaban proporcionando estimaciones aproximadas de las capacidades del enemigo y revisando un plan. Hice una mueca y me froté la frente mientras me acercaba a ellos. Un dolor de cabeza ya se estaba desarrollando debido al tumulto.

    “Bienvenido a bordo, General”. Rykov no apartó la vista del mapa de holovisión ni por un segundo, así que no estaba seguro de cómo notó mi acercamiento. "Estaremos saliendo en unos momentos. Confío en que no nos causará problemas. Siéntese y disfrute del espectáculo".

    "¡Muy bien, todos a sus puestos!" La voz de Rykov subió a un tono atronador, sobresaliendo entre el alboroto de fondo. "Establezcan rumbo hacia el Sistema 1964-A. Sistemas de armas en alerta máxima, el equipo de abordaje debe estar listo".

    En un instante, toda conversación cesó y los miembros de la tripulación se apresuraron a sus puestos. Un equipo silencioso y atento reemplazó el caos en un abrir y cerrar de ojos. Me maravillé de lo drástico que era el cambio, observando cómo ejecutaban sus asignaciones con eficiencia entrenada. La dualidad de la humanidad era tan evidente en sus operaciones diarias como en su política marcial.

    Una sensación de hundimiento familiar apretó mi estómago mientras entrábamos en el hiperespacio. Había un ruido extraño de traqueteo que resonaba en las paredes, lo que sugería que la nave estaba presionando los límites superiores de su velocidad de salto. La nave humana volvió al espacio real en cuestión de minutos, en los límites del territorio Devorador.

    "Nuestros sensores están detectando una formación de 16 naves en trayectoria de patrulla, dentro del alcance de las armas, señor", dijo un joven oficial.

    El Comandante Rykov asintió. "Muy bien. Quiero que todas las naves excepto una sean destruidas antes de que se den cuenta de lo que les golpeó. Deshabilitamos la última y la abordamos. Necesitamos sistemas en línea para que los EMP queden descartados, mantengan armas convencionales. ¡Vamos!"

    Miré por la ventana mientras cientos de misiles se dirigían hacia la flota. Un indicador parpadeó en la pantalla rastreando los bloqueos de objetivo; parecía que la computadora estaba pilotando los misiles de forma remota. Las naves de patrulla giraron para enfrentarnos, disparando proyectiles cinéticos en un intento de destruir los proyectiles. Sus balas conectaron con algunos misiles, pero con solo segundos para reaccionar, no había forma de eliminarlos a todos.

    Los explosivos humanos perforaron los cascos metálicos de los Devoradores como si fueran papel. La fuerza de las detonaciones múltiples simultáneas los redujo a sus esqueletos, lanzando metal deformado en todas direcciones. La única nave que quedó fue la rezagada en la parte trasera de la formación.

    Un solo proyectil golpeó al último crucero, abriendo una brecha en su costado. No había forma de que la nave pudiera saltar mientras se ventila la atmósfera. Una nave de transporte humana se acercó a la nave dañada. No estaba claro a lo que se enfrentaría el equipo de abordaje en el interior, pero después del poder desatado que había presenciado de nuevo, tenía confianza en que cualquier resistencia de los Devoradores sería sofocada con poco problema.

    Rykov golpeó el pie impacientemente mientras su equipo exploraba la nave. "Líder del equipo, informe de estado, por favor".

    "Señor, encontramos a dos combatientes enemigos inconscientes a bordo. Parece que se apagó el soporte vital". Una voz varonil y ronca resonó en el altavoz. "No dañamos su ordenador ni su energía. Ellos mismos lo hicieron".

    "¿Qué?! Intentaron suicidarse en lugar de ser capturados..." El Comandante se quedó pensando. "Llévenlos de vuelta a su nave de inmediato. Traten de reanimarlos".

    "Sí, señor. Estamos en eso".

    Fruncí el ceño confundido. ¿Por qué los Devoradores apagarían su soporte vital? Tal vez se trataba de honor, pero no tenía sentido optar por la asfixia lenta en lugar de una simple bala en la cabeza.

    Tenía que esperar que los médicos humanos fueran tan competentes como sus soldados. Había tantas preguntas que hacer, pero los muertos no nos darían respuestas.

    Teme que los humanos pudieran atacar en cuanto nuestras naves entraran en el sistema Sol, pero el hecho de que todavía estuviéramos aquí era una buena señal.

    El Senado de la Federación había votado por poco enfrentarse a los Terranos, con la Presidenta Ula siendo una de las partidarias más fervientes de la moción. Incluso con su influencia política, muchos representantes estaban indecisos acerca de tomar medidas. El destino que había caído sobre los Devoradores fácilmente podría ser el nuestro si provocábamos a los humanos.

    Sinceramente, creo que si fuera su propia especie la que estuviera siendo convocada a la acción, el Senado no habría aprobado la propuesta. Pero como siempre, asumían que los Jatari, los Xanik y los Hoda’al harían su trabajo sucio, mientras ellos se quedaban observando desde la seguridad de sus oficinas.

    No estaba muy emocionado por liderar esta misión. Después de todo, estábamos arriesgando vidas de la Federación para proteger a las mismas personas que habían intentado destruirnos. Si bien la solución de los Terranos era extrema, al menos podía entender de dónde venían. Pero sería deshonroso rechazar una orden directa; lo último que quería era que me etiquetaran como traidor y cobarde.

    Además, si yo comandaba la flota, al menos sería lo suficientemente sensato como para no cargar en batalla contra un ejército superior. No estaba seguro de que mis colegas, que no habían presenciado la acción de las armas humanas de primera mano, fueran tan cautelosos. Especialmente dado que la mayoría de los oficiales Jatari veían la diplomacia como una admisión de debilidad.

    El Primer Oficial Blez miró hacia arriba desde su computadora mientras pasábamos el primero de los planetas exteriores. “Señor, estamos casi dentro del alcance de los misiles de la Tierra. ¿Deberíamos preparar nuestras armas?”

    “Nuestras órdenes son detenerlos, no atacarlos. Si entramos en una lucha directa, estamos condenados”, respondí. “Esperemos que a los humanos todavía les guste hablar. Llama al Comando Terrano.”

    Blez abrió la boca para discutir, pero luego pensó mejor. Ingresó silenciosamente algunas órdenes en su terminal, murmurando para sí mismo. Los pocos momentos que la llamada no fue respondida fueron angustiantes; temía que los humanos simplemente nos ignoraran. El alivio me inundó cuando una cara familiar parpadeó en la pantalla de visualización.

    El Comandante Rykov no se veía bien. Su cabello negro estaba desaliñado, su uniforme estaba arrugado y las ojeras se habían instalado debajo de sus ojos. Esto estaba lejos del hombre radiante y confiado que había venido a rescatarnos ayer. Parecía que debería estar descansando en lugar de estar en el puente de una nave, pero temía que señalar su condición causaría ofensa.

    El oficial humano miró fijamente a la cámara, con una expresión suplicante en su rostro. “General. Le recomendamos encarecidamente que dé la vuelta y se aparte.”

    “No puedo hacer eso. Lo que estás a punto de hacer está mal. La vida inteligente es sagrada y eliminar a una especie entera es un crimen contra la conciencia”, dije.

    “Los Devoradores apenas han demostrado ser conscientes. Me sorprende que, de todas las personas, te apresures a defenderlos”, reflexionó Rykov. “Ni siquiera ha pasado un día completo desde que eliminaron miles de tus naves. Tú y yo sabemos que si no hubiéramos aparecido, te habrían matado a todos sin pensarlo dos veces”.

    Me estremecí. “No me lo recuerdes. A pesar de todo lo que han hecho, no quiero ver a una especie entera masacrada. Eso nos hace igual de malos que ellos. Sus acciones no hacen que las tuyas sean correctas”.

    El Comandante Rykov suspiró. “Bueno, parece que estamos en un punto muerto. Supongo que nos atacarás si no retrocedemos, ¿verdad?”

    “Solo queremos hablar. No tienes por qué hacer esto. ¿Tu especie tiene un código moral, verdad?” Tomé una respiración profunda, tratando de recopilar mis pensamientos. “¿Y si hay personas inocentes, niños y civiles, en su planeta natal?”

    “Mira, no me gusta lo que estamos a punto de hacer, pero tengo mis órdenes. Ni siquiera sabemos si tienen civiles o si pueden mostrar emoción”.

    “Exactamente, no sabemos. ¿Cuál es el daño en esperar y obtener más información? ¿No quieres saber por qué están haciendo esto?”

    “Me gustaría entenderlo”, Rykov inclinó la cabeza, como si estuviera pensando. “Supongo que no haría daño reunir algo de inteligencia. Demonios, podría ser útil en el futuro. ¿Qué sugieres?”

    “¿Crees que pueden capturar una de sus naves? Necesitamos traer a uno de ellos con vida”.

    "Sí, creo que podemos hacerlo, General. ¿Qué te parece unirte a nosotros en persona en nuestro buque insignia? Preferiríamos estar juntos en lugar de enemigos."

    Pesé mis opciones. Esto fácilmente podría ser una especie de astucia de los humanos, atrayendo al oficial de mayor rango de la Federación a su sede solo para ser encarcelado. Sacarme de la imagen interrumpiría el mando de nuestra flota; era natural encontrar su oferta un poco sospechosa.

    Pero supuse que si las intenciones de Rykov hacia nosotros fueran maliciosas, no estaríamos teniendo este diálogo en primer lugar. Los Terranos tenían la capacidad de eliminar toda nuestra flota de un solo golpe, pero no nos habían disparado. En cualquier caso, aún le debía al Comandante una gran deuda por salvar mi vida. Lo menos que podía darle era un poco de confianza.

    “Estaré encantado de unirme a ti, Comandante”, respondí.

    La insinuación de una sonrisa se deslizó en el rostro de Rykov. “Excelente. Esperaremos tu lanzadera. Ven solo y desarmado. Por favor, ordena a tus naves que detengan su avance y nos permitan el paso.”

    La transmisión terminó y el Primer Oficial Blez habló inmediatamente. “Señor, no puedes estar pensando seriamente en ir allí”.

    Le fulminé con la mirada, sin apreciar que se cuestionara mis decisiones. “Tengo que hacerlo. Es nuestra única oportunidad de hablar con los humanos y será la primera vez que alguien haya hablado con el enemigo en persona”.

    Por supuesto, cualquier visión que pudiera obtener sobre la naturaleza de los Devoradores sería invaluable para la Federación. Pero estaría mintiendo si dijera que mi curiosidad no era personal. Me deleitaba con la posibilidad de exigir sus razones yo mismo. El asesinato en masa no era la solución, pero nuestros enemigos debían rendir cuentas por las pérdidas que habían causado.

    \begin{center} {\Huge $\divideontimes$} \end{center}

    Dos soldados terranos esperaban en la esclusa cuando mi lanzadera se acopló. La revisión que me hicieron se sintió un poco... invasiva, pero supongo que solo querían ser minuciosos. Una vez que estuvieron satisfechos de que no tenía armas encima, me condujeron al puente.

    En comparación con las naves de la Federación, la nave insignia terrana era realmente fea por dentro. Los pasillos eran estrechos y los colores eran una mezcla apagada de gris y blanco sucio. Era evidente que los humanos prestaban poca atención a los elementos de diseño, enfocándose en llenar la nave de guerra con la mayor cantidad de armas y estaciones posible. No pude evitar sentirme un poco claustrofóbico mientras navegábamos a través de una serie de pasillos sinuosos y escaleras estrechas.

    El pasillo finalmente se abrió en una cámara más amplia, que estaba alineada con filas de monitores de computadora y una pantalla holográfica en el centro. Mi primer pensamiento fue que nunca antes había visto un centro de mando tan desordenado en mi vida. Docenas de personas estaban ocupadas en el lugar, con tabletas en la mano, gritándose mutuamente. ¿Cómo podían funcionar en medio de tanto ruido y caos?

    El Comandante Rykov estaba en el centro de esta locura, estudiando una proyección de la flota Devoradora. Dos oficiales estaban a su lado; por lo que escuché, parecía que estaban proporcionando estimaciones aproximadas de las capacidades del enemigo y revisando un plan. Hice una mueca y me froté la frente mientras me acercaba a ellos. Un dolor de cabeza ya se estaba desarrollando debido al tumulto.

    “Bienvenido a bordo, General”. Rykov no apartó la vista del mapa de holovisión ni por un segundo, así que no estaba seguro de cómo notó mi acercamiento. "Estaremos saliendo en unos momentos. Confío en que no nos causará problemas. Siéntese y disfrute del espectáculo".

    "¡Muy bien, todos a sus puestos!" La voz de Rykov subió a un tono atronador, sobresaliendo entre el alboroto de fondo. "Establezcan rumbo hacia el Sistema 1964-A. Sistemas de armas en alerta máxima, el equipo de abordaje debe estar listo".

    En un instante, toda conversación cesó y los miembros de la tripulación se apresuraron a sus puestos. Un equipo silencioso y atento reemplazó el caos en un abrir y cerrar de ojos. Me maravillé de lo drástico que era el cambio, observando cómo ejecutaban sus asignaciones con eficiencia entrenada. La dualidad de la humanidad era tan evidente en sus operaciones diarias como en su política marcial.

    Una sensación de hundimiento familiar apretó mi estómago mientras entrábamos en el hiperespacio. Había un ruido extraño de traqueteo que resonaba en las paredes, lo que sugería que la nave estaba presionando los límites superiores de su velocidad de salto. La nave humana volvió al espacio real en cuestión de minutos, en los límites del territorio Devorador.

    "Nuestros sensores están detectando una formación de 16 naves en trayectoria de patrulla, dentro del alcance de las armas, señor", dijo un joven oficial.

    El Comandante Rykov asintió. "Muy bien. Quiero que todas las naves excepto una sean destruidas antes de que se den cuenta de lo que les golpeó. Deshabilitamos la última y la abordamos. Necesitamos sistemas en línea para que los EMP queden descartados, mantengan armas convencionales. ¡Vamos!"

    Miré por la ventana mientras cientos de misiles se dirigían hacia la flota. Un indicador parpadeó en la pantalla rastreando los bloqueos de objetivo; parecía que la computadora estaba pilotando los misiles de forma remota. Las naves de patrulla giraron para enfrentarnos, disparando proyectiles cinéticos en un intento de destruir los proyectiles. Sus balas conectaron con algunos misiles, pero con solo segundos para reaccionar, no había forma de eliminarlos a todos.

    Los explosivos humanos perforaron los cascos metálicos de los Devoradores como si fueran papel. La fuerza de las detonaciones múltiples simultáneas los redujo a sus esqueletos, lanzando metal deformado en todas direcciones. La única nave que quedó fue la rezagada en la parte trasera de la formación.

    Un solo proyectil golpeó al último crucero, abriendo una brecha en su costado. No había forma de que la nave pudiera saltar mientras se ventila la atmósfera. Una nave de transporte humana se acercó a la nave dañada. No estaba claro a lo que se enfrentaría el equipo de abordaje en el interior, pero después del poder desatado que había presenciado de nuevo, tenía confianza en que cualquier resistencia de los Devoradores sería sofocada con poco problema.

    Rykov golpeó el pie impacientemente mientras su equipo exploraba la nave. "Líder del equipo, informe de estado, por favor".

    "Señor, encontramos a dos combatientes enemigos inconscientes a bordo. Parece que se apagó el soporte vital". Una voz varonil y ronca resonó en el altavoz. "No dañamos su ordenador ni su energía. Ellos mismos lo hicieron".

    "¿Qué?! Intentaron suicidarse en lugar de ser capturados..." El Comandante se quedó pensando. "Llévenlos de vuelta a su nave de inmediato. Traten de reanimarlos".

    "Sí, señor. Estamos en eso".

    Fruncí el ceño confundido. ¿Por qué los Devoradores apagarían su soporte vital? Tal vez se trataba de honor, pero no tenía sentido optar por la asfixia lenta en lugar de una simple bala en la cabeza.

    Tenía que esperar que los médicos humanos fueran tan competentes como sus soldados. Había tantas preguntas que hacer, pero los muertos no nos darían respuestas.

\end{document}