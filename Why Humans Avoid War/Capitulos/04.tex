\chapter{El Enemigo en Casa}\label{sec:el-enemigo-en-casa}

Los Devoradores no parecían tan temibles en persona. Eran bípedos bajos y fornidos que no parecían nada fuera de lo común en comparación con la mayoría de las razas de la Federación. Su altura solo los elevaba a la altura de los hombros de un humano promedio, y su piel era de un tono lavanda pálido. No tenía dudas de que los delgados y musculosos soldados terranos podrían lanzarlos por todos lados si quisieran.

Si el grupo de abordaje hubiera tomado la nave enemiga unos minutos más tarde, nos habríamos quedado con las manos vacías. De hecho, los humanos solo habían podido reanimar a uno de los dos ocupantes. Nuestro prisionero fue entonces transportado de vuelta a la nave insignia y trasladado al ala médica, donde se le restableció su condición estable. Lo mantuvieron atado y lo vigilaron las veinticuatro horas del día centinelas vigilantes.

Acompañé al comandante Rykov mientras se dirigía a la enfermería. Sería interesante presenciar tácticas de interrogatorio humanas. Después de ver el placer cruel en sus ojos durante la batalla, me pregunté si torturarían al prisionero para obtener información. Ciertamente, eso estaba dentro del ámbito de lo posible.

Mientras caminábamos, un asistente le entregó al comandante una taza llena de un líquido marrón humeante. Cuando le pregunté qué era, me explicó que se llamaba “café” y que era un estimulante suave. Me limité a asentir, sin querer ofender a mi anfitrión. Sin embargo, en mi interior pensaba que era de muy mal gusto que un oficial consumiera drogas durante el servicio. Era un mal ejemplo para sus subordinados.

El prisionero se estaba despertando justo cuando llegamos a nuestro destino. Parecía un poco desorientado, pero curiosamente no luchaba contra las ataduras. Había una computadora portátil junto a su cama, con una captura de audio ejecutándose en la pantalla.

“¿Funcionará nuestro software de traducción?”, le susurré a Rykov.

El humano se encogió de hombros en respuesta. “Debería. Nuestro programa ha revisado todas las transmisiones que tenemos registradas y esperamos que haya podido descifrar su idioma a partir de eso”.

El enemigo cautivo pronunció unas cuantas sílabas en un galimatías y, un segundo después, la computadora habló en el idioma galáctico. Las dos palabras me dejaron helado. Decía: «Ayúdennos».

El comandante Rykov parpadeó confundido. “¿Te ayudo? Bien, retrocede. Primero que nada, ¿cuál es tu nombre y rango?”

Hubo una pausa mientras la computadora traducía la pregunta y luego otra mientras procesaba la respuesta. “Mi nombre es Byem. No sé qué es ese “rango” del que hablas”.

“¿No tenéis algún tipo de jerarquía?”, pregunté.

“El Maestro está a cargo de todo. Obedecemos o sufrimos las consecuencias. No hay escapatoria.”

Rykov dio un paso hacia delante, indeciso. —¿Quién es el Maestro? ¿Por qué nos atacaste?

El prisionero emitió una extraña vibración, que la computadora identificó como una risa. “La pregunta más precisa es qué es el Maestro. Ahora veo que no sabes nada. Simplemente supuse que la gente con tu tecnología estaría al tanto de nuestra historia.

En su día fuimos una gran especie. Cuando era joven, recuerdo que me maravillaba la tecnología que inventamos. Puedo decir con seguridad que fuimos los mejores constructores de nuestra galaxia. La ironía es que fue nuestra astucia la que nos destruyó.

Creamos una inteligencia artificial con una sola directiva: crear un mundo sin escasez. Se le dio autoridad para gobernar nuestros recursos y abastecer de energía a nuestras ciudades. Creímos que podíamos crear una utopía que pusiera fin a toda la necesidad, el trabajo y el sufrimiento. Era demasiado bueno para ser verdad.

La máquina reflexionó sobre el problema. Supusimos que crearía una nueva forma de energía o que optimizaría la minería de asteroides. Pero encontró una solución diferente. La única forma de evitar la escasez era controlar todos los recursos del universo. Los tomaría por la fuerza y nos usaría como su ejército.

Intentar imaginar a los Devoradores como una especie pacífica de inventores era difícil. Durante años, la Inteligencia de la Federación los había visto destruir a cualquier especie que se atreviera a defender su planeta natal. Rodeaban estrellas con paneles absorbentes y saqueaban planetas, sin pensar dos veces en las formas de vida que extinguían.

Nos dijeron que no se podía razonar con el enemigo y que su codicia no tenía parangón. Pero si lo que decía Byem era cierto, entonces habían participado involuntariamente todo el tiempo. Su comportamiento mecánico y sin sentido tenía mucho más sentido si estaban bajo la dirección de una IA rebelde.

Yo creía en su historia; la pregunta era si Rykov también lo creía. La revelación podría alejar a la Unión Terrana de la ruta del genocidio, pero el Comandante tenía que ser quien transmitiera el mensaje. Dudaba que los humanos creyeran cualquier información que viniera de nosotros.

El comandante Rykov tomó un sorbo de café y se tomó un momento para procesar lo que se había dicho. “¿Por qué nadie se defendería? ¿O intentaría destruirlo?”

—Por supuesto que la gente lo hizo. Pero ahora todos están muertos. El Maestro había anulado su función de apagado de emergencia. Ninguna de nuestras salvaguardas funcionó. Controlaba todo, militar e industrial, así que ¿con qué podíamos luchar?

Su única utilidad es la de servirnos de recurso. Si lo desafiamos, si fracasamos, ya no seremos útiles… y ya verás lo que pasa. Una vez que tome el control de todo, no tengo ninguna duda de que nos matará a todos de todos modos, pero eso llevará tiempo. Si nos sometemos, ganaremos unas cuantas generaciones más.

Como ya he dicho, no hay salida para nosotros. Debe cumplir su misión. No entiende otra cosa”.

—Ya veo —murmuró el comandante Rykov—. Respóndeme una cosa más. ¿Tus armas también son inventos tuyos?

—No, nuestra flota fue ideada por el Maestro. Su tecnología está más allá de cualquier cosa que los seres biológicos pudieran conjurar, o eso creíamos. Después de todo, ¿qué podría ser mejor para matar que una computadora?

Eres el primero en derrotarlo y lo hiciste con facilidad. Tal vez debería tenerte miedo... pero eres nuestra única esperanza.

El comandante frunció el ceño. “Gracias por hablar con nosotros, Byem. Eso es todo por ahora. General, por favor, acompáñeme de regreso al puente”.

Esperé hasta que estuvimos fuera del alcance auditivo del prisionero y luego me volví hacia Rykov. “¿Qué piensas?”

—Es una historia inquietante —respondió el humano—. Si no fuera por el intento de suicidio, no le creería tanto. No tiene sentido sin una fuerza externa. Necesito compartir nuestros hallazgos con mi gobierno de inmediato. Esto lo cambia todo.

“¿Les aconsejaría que suspendan el bombardeo?”, pregunté.

El comandante Rykov suspiró. “Lo haré. Al menos tenemos que intentar ayudar”.

``¿Pero?"

—Pero la única forma de asegurarnos de destruir esa cosa es destruir todo lo que hay en ese planeta. Si tratamos de evacuar a la gente, simplemente los matará. Si no hacemos nada, podría estudiar nuestra tecnología y replicarla. Entonces estaríamos en serios problemas. No estoy seguro de que tengamos otra opción, general.

Las palabras del Comandante tenían sentido, por mucho que odiara oírlas. No podíamos arriesgarnos a que el armamento terrestre cayera en manos de una IA asesina. Alguien tenía que idear un plan sólido en poco tiempo, antes de que pasara el momento de actuar.

Pero había algo más que me molestaba. Era un punto que Byem había mencionado y que me quedó grabado en la mente: el hecho de que los terranos habían creado mejores herramientas para la guerra que una computadora, una máquina con el poder puro del cálculo de su lado.

Decía mucho sobre su especie y sobre lo natural que resultaba matar para la humanidad. Sentí que debía ser más cauteloso, pero no pude evitar sentirme encantado por ellos. Por alguna razón, mi instinto me decía que podía confiar en ellos.

Quizás deberíamos temer a los humanos, pero en este punto, ellos eran la única esperanza de la galaxia.
