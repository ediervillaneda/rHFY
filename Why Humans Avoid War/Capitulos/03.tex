\chapter{Kilon POV: La Encrucijada de la Diplomacia}

Teme que los humanos pudieran atacar en cuanto nuestras naves entraran en el sistema Sol, pero el hecho de que todavía estuviéramos aquí era una buena señal.

El Senado de la Federación había votado por poco enfrentarse a los Terranos, con la Presidenta Ula siendo una de las partidarias más fervientes de la moción. Incluso con su influencia política, muchos representantes estaban indecisos acerca de tomar medidas. El destino que había caído sobre los Devoradores fácilmente podría ser el nuestro si provocábamos a los humanos.

Sinceramente, creo que si fuera su propia especie la que estuviera siendo convocada a la acción, el Senado no habría aprobado la propuesta. Pero como siempre, asumían que los Jatari, los Xanik y los Hoda’al harían su trabajo sucio, mientras ellos se quedaban observando desde la seguridad de sus oficinas.

No estaba muy emocionado por liderar esta misión. Después de todo, estábamos arriesgando vidas de la Federación para proteger a las mismas personas que habían intentado destruirnos. Si bien la solución de los Terranos era extrema, al menos podía entender de dónde venían. Pero sería deshonroso rechazar una orden directa; lo último que quería era que me etiquetaran como traidor y cobarde.

Además, si yo comandaba la flota, al menos sería lo suficientemente sensato como para no cargar en batalla contra un ejército superior. No estaba seguro de que mis colegas, que no habían presenciado la acción de las armas humanas de primera mano, fueran tan cautelosos. Especialmente dado que la mayoría de los oficiales Jatari veían la diplomacia como una admisión de debilidad.

El Primer Oficial Blez miró hacia arriba desde su computadora mientras pasábamos el primero de los planetas exteriores. “Señor, estamos casi dentro del alcance de los misiles de la Tierra. ¿Deberíamos preparar nuestras armas?”

“Nuestras órdenes son detenerlos, no atacarlos. Si entramos en una lucha directa, estamos condenados”, respondí. “Esperemos que a los humanos todavía les guste hablar. Llama al Comando Terrano.”

Blez abrió la boca para discutir, pero luego pensó mejor. Ingresó silenciosamente algunas órdenes en su terminal, murmurando para sí mismo. Los pocos momentos que la llamada no fue respondida fueron angustiantes; temía que los humanos simplemente nos ignoraran. El alivio me inundó cuando una cara familiar parpadeó en la pantalla de visualización.

El Comandante Rykov no se veía bien. Su cabello negro estaba desaliñado, su uniforme estaba arrugado y las ojeras se habían instalado debajo de sus ojos. Esto estaba lejos del hombre radiante y confiado que había venido a rescatarnos ayer. Parecía que debería estar descansando en lugar de estar en el puente de una nave, pero temía que señalar su condición causaría ofensa.

El oficial humano miró fijamente a la cámara, con una expresión suplicante en su rostro. “General. Le recomendamos encarecidamente que dé la vuelta y se aparte.”

“No puedo hacer eso. Lo que estás a punto de hacer está mal. La vida inteligente es sagrada y eliminar a una especie entera es un crimen contra la conciencia”, dije.

“Los Devoradores apenas han demostrado ser conscientes. Me sorprende que, de todas las personas, te apresures a defenderlos”, reflexionó Rykov. “Ni siquiera ha pasado un día completo desde que eliminaron miles de tus naves. Tú y yo sabemos que si no hubiéramos aparecido, te habrían matado a todos sin pensarlo dos veces”.

Me estremecí. “No me lo recuerdes. A pesar de todo lo que han hecho, no quiero ver a una especie entera masacrada. Eso nos hace igual de malos que ellos. Sus acciones no hacen que las tuyas sean correctas”.

El Comandante Rykov suspiró. “Bueno, parece que estamos en un punto muerto. Supongo que nos atacarás si no retrocedemos, ¿verdad?”

“Solo queremos hablar. No tienes por qué hacer esto. ¿Tu especie tiene un código moral, verdad?” Tomé una respiración profunda, tratando de recopilar mis pensamientos. “¿Y si hay personas inocentes, niños y civiles, en su planeta natal?”

“Mira, no me gusta lo que estamos a punto de hacer, pero tengo mis órdenes. Ni siquiera sabemos si tienen civiles o si pueden mostrar emoción”.

“Exactamente, no sabemos. ¿Cuál es el daño en esperar y obtener más información? ¿No quieres saber por qué están haciendo esto?”

“Me gustaría entenderlo”, Rykov inclinó la cabeza, como si estuviera pensando. “Supongo que no haría daño reunir algo de inteligencia. Demonios, podría ser útil en el futuro. ¿Qué sugieres?”

“¿Crees que pueden capturar una de sus naves? Necesitamos traer a uno de ellos con vida”.

``Sí, creo que podemos hacerlo, General. ¿Qué te parece unirte a nosotros en persona en nuestro buque insignia? Preferiríamos estar juntos en lugar de enemigos.''

Pesé mis opciones. Esto fácilmente podría ser una especie de astucia de los humanos, atrayendo al oficial de mayor rango de la Federación a su sede solo para ser encarcelado. Sacarme de la imagen interrumpiría el mando de nuestra flota; era natural encontrar su oferta un poco sospechosa.

Pero supuse que si las intenciones de Rykov hacia nosotros fueran maliciosas, no estaríamos teniendo este diálogo en primer lugar. Los Terranos tenían la capacidad de eliminar toda nuestra flota de un solo golpe, pero no nos habían disparado. En cualquier caso, aún le debía al Comandante una gran deuda por salvar mi vida. Lo menos que podía darle era un poco de confianza.

“Estaré encantado de unirme a ti, Comandante”, respondí.

La insinuación de una sonrisa se deslizó en el rostro de Rykov. “Excelente. Esperaremos tu lanzadera. Ven solo y desarmado. Por favor, ordena a tus naves que detengan su avance y nos permitan el paso.”

La transmisión terminó y el Primer Oficial Blez habló inmediatamente. “Señor, no puedes estar pensando seriamente en ir allí”.

Le fulminé con la mirada, sin apreciar que se cuestionara mis decisiones. “Tengo que hacerlo. Es nuestra única oportunidad de hablar con los humanos y será la primera vez que alguien haya hablado con el enemigo en persona”.

Por supuesto, cualquier visión que pudiera obtener sobre la naturaleza de los Devoradores sería invaluable para la Federación. Pero estaría mintiendo si dijera que mi curiosidad no era personal. Me deleitaba con la posibilidad de exigir sus razones yo mismo. El asesinato en masa no era la solución, pero nuestros enemigos debían rendir cuentas por las pérdidas que habían causado.