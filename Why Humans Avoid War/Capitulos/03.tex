\chapter{El Dilema de la Federación}\label{sec:el-dilema-de-la-federacion}

Había temido que los humanos pudieran atacar tan pronto como nuestras naves entraran en el sistema Solar, pero el hecho de que todavía estuviéramos aquí era una buena señal.

El Senado de la Federación había votado por un estrecho margen a favor de enfrentarse a los terranos, y la Portavoz Ula era una de las defensoras más fervientes de la moción. Incluso con su influencia política, muchos representantes estaban indecisos sobre si tomar medidas. El destino que había caído sobre los Devoradores fácilmente podría ser el nuestro también si provocábamos a los humanos.

Sinceramente, creo que si fuera su propia especie la que tuviera que actuar, el Senado no habría aprobado la propuesta. Pero, como siempre, asumieron que los Jatari, los Xanik y los Hoda'al harían el trabajo sucio, mientras ellos se quedaban de brazos cruzados y observaban desde la seguridad de sus oficinas.

No me entusiasmaba demasiado liderar esta misión. Después de todo, estábamos arriesgando vidas de miembros de la Federación para proteger a las mismas personas que habían intentado destruirnos. Aunque la solución de los terrícolas era extrema, al menos podía entender sus intenciones. Pero sería deshonroso rechazar una orden directa; lo último que quería era que me tildaran de traidor y cobarde.

Además, si yo comandaba la flota, al menos tendría la suficiente sensatez como para no lanzarme a la batalla contra un ejército superior. No estaba seguro de que mis compañeros, que no habían presenciado de primera mano el uso de armas humanas en acción, fueran tan cautelosos. Sobre todo teniendo en cuenta que la mayoría de los oficiales Jatari consideraban la diplomacia como una admisión de debilidad.

El primer oficial Blez levantó la vista de su computadora cuando pasamos por el primero de los planetas exteriores. “Señor, estamos casi dentro del alcance de los misiles de la Tierra. ¿Deberíamos preparar nuestras armas?”

—Nuestras órdenes son detenerlos, no atacarlos. Si nos enzarzamos en una pelea directa, estaremos perdidos —respondí—. Esperemos que a los humanos todavía les guste hablar. Saludos al Comando Terran.

Blez abrió la boca para discutir, pero luego lo pensó mejor. Ingresó en silencio algunos comandos en su terminal, murmurando en voz baja. Los pocos momentos en que la llamada no fue respondida fueron angustiosos; temí que los humanos simplemente nos ignoraran. Sentí alivio cuando una cara familiar apareció en la pantalla.

El comandante Rykov no tenía buen aspecto. Su pelo negro estaba despeinado, su uniforme estaba arrugado y tenía ojeras bajo los ojos. No tenía nada que ver con el hombre radiante y seguro de sí mismo que había acudido a nuestro rescate ayer. Parecía que debería estar descansando en lugar de en el puente de un barco, pero temí que señalar su estado pudiera ofender a alguien.

El oficial humano miró fijamente a la cámara con expresión suplicante. “General, le recomendamos encarecidamente que dé la vuelta a sus naves y se haga a un lado”.

—No puedo hacer eso. Lo que estás a punto de hacer está mal. La vida inteligente es sagrada y matar a una especie entera es un crimen contra la sensibilidad —dije.

—Los Devoradores apenas han demostrado ser inteligentes. Me sorprende que tú, de entre todas las personas, te hayas apresurado a defenderlos —reflexionó Rykov—. Ni siquiera ha pasado un día completo desde que destruyeron miles de sus naves. Tú y yo sabemos que, si no hubiéramos aparecido, los habrían matado a todos sin pensarlo dos veces.

Me estremecí. —No me lo recuerdes. A pesar de todo lo que han hecho, no quiero ver a una especie entera masacrada. Eso nos hace tan malos como ellos. Sus acciones no justifican las tuyas.

El comandante Rykov suspiró. “Bueno, parece que estamos en un punto muerto. Supongo que nos atacarán si no nos retiramos, ¿no?”

—Solo queremos hablar. No tienes por qué hacer esto. Tu especie tiene un código moral, ¿no? —Respiré profundamente, intentando ordenar mis pensamientos—. ¿Y si hay gente inocente, niños y civiles, en su mundo natal?

“Mira, no me gusta lo que vamos a hacer, pero tengo mis órdenes. Ni siquiera sabemos si tienen civiles o si pueden mostrar emociones”.

—Exactamente, no lo sabemos. ¿Qué daño hay en esperar y obtener más información? ¿No quieres saber por qué están haciendo esto?

—Me gustaría entenderlo. —Rykov inclinó la cabeza, como si estuviera pensando—. Supongo que no estaría de más reunir algo de información. Diablos, podría resultar útil en el futuro. ¿Qué sugieres?

“¿Crees que puedes capturar uno de sus barcos? Necesitamos traer a uno de ellos con vida”.

—Sí, creo que podemos hacerlo, general. ¿Qué le parecería unirse a nosotros en persona en nuestro buque insignia? Preferiríamos estar juntos que como enemigos.

Consideré mis opciones. Fácilmente podría tratarse de algún tipo de truco humano, que atraía al oficial de más alto rango de la Federación a su cuartel general solo para encarcelarlo. Sacarme de la escena perturbaría el mando de nuestra flota; era natural que su oferta me pareciera un poco sospechosa.

Pero pensé que si las intenciones de Rykov hacia nosotros eran maliciosas, no estaríamos teniendo este diálogo en primer lugar. Los terranos tenían la capacidad de destruir toda nuestra flota de un solo golpe, pero no nos habían disparado. En cualquier caso, todavía le debía mucho al comandante por salvarme la vida. Lo mínimo que podía darle era un poco de confianza.

—Me encantaría unirme a usted, comandante —respondí.

En el rostro de Rykov se dibujó una leve sonrisa. —Excelente. Esperaremos su lanzadera. Venga solo y desarmado. Por favor, ordene a sus naves que detengan su avance y nos permitan el paso.

La transmisión terminó y el primer oficial Blez intervino de inmediato: “Señor, no puede estar pensando seriamente en ir allí”.

Le fruncí el ceño, sin apreciar que se cuestionaran mis decisiones. —Tengo que hacerlo. Es nuestra única oportunidad de convencer a los humanos y será la primera vez que alguien hable directamente con el enemigo.

Por supuesto, cualquier información que pudiera obtener sobre la naturaleza del Devorador sería inestimable para la Federación, pero mentiría si dijera que mi curiosidad no era personal. Me encantaba la posibilidad de exigirles sus razones yo mismo. El asesinato en masa no era la solución, pero nuestros enemigos debían rendir cuentas por las pérdidas que habían infligido.

---

Dos soldados terranos me esperaban en la esclusa de aire mientras mi transbordador atracaba. El cacheo que me hicieron me pareció un poco… invasivo, pero supongo que solo querían ser minuciosos. Una vez que estuvieron seguros de que no llevaba armas, me guiaron hasta el puente.

En comparación con las naves de la Federación, la nave insignia de los terranos era francamente fea por dentro. Los pasillos eran estrechos y los colores eran una mezcla monótona de gris y blanquecino. Era evidente que los humanos no prestaban demasiada atención a los elementos de diseño y se concentraban en llenar la nave de guerra con tantas armas y estaciones como fuera posible. No pude evitar sentirme un poco claustrofóbico mientras navegábamos por una serie de pasillos tortuosos y escaleras estrechas.

El pasillo finalmente se abrió a una cámara más amplia, que estaba llena de hileras de monitores de computadora y una pantalla holográfica en el centro. Mi primer pensamiento fue que nunca había visto un centro de comando tan desordenado en mi vida. Docenas de personal se movían ajetreados por el lugar, tabletas en mano, gritándose unos a otros. ¿Cómo podrían siquiera funcionar en medio de tanto ruido y caos?

El comandante Rykov estaba en el centro de esta locura, estudiando una proyección de la flota Devoradora. Dos oficiales estaban a su lado; por lo que escuché, parecía que estaban brindando estimaciones aproximadas de las capacidades del enemigo y revisando un plan. Hice una mueca y me froté la frente mientras caminaba hacia ellos. Ya me estaba empezando a doler la cabeza por la conmoción.

—Bienvenido a bordo, general. —Rykov no apartó la vista del holomapa ni un segundo, así que no estaba muy seguro de cómo había visto mi llegada—. Nos marcharemos en unos minutos. Confío en que no nos cause ningún problema. Siéntese y disfrute del espectáculo.

—¡Muy bien, todos a sus puestos! —La voz de Rykov se elevó hasta convertirse en un grito atronador que se oía por encima del parloteo de fondo—. Pongan rumbo al Sistema 1964-A. Sistemas de armas en alerta máxima, equipo de abordaje preparado.

En un instante, cesó toda conversación y los miembros de la tripulación se apresuraron a ocupar sus puestos. Un equipo silencioso y atento reemplazó el caos en un instante. Me maravillé de lo drástico del cambio, al observar cómo ejecutaban sus tareas con una eficiencia entrenada. La dualidad de la humanidad era tan evidente en sus operaciones diarias como en su política marcial.

Una familiar sensación de hundimiento se apoderó de mi estómago mientras nos deslizábamos hacia el hiperespacio. Se oía un extraño ruido de traqueteo que resonaba en las paredes, lo que sugería que la nave estaba superando los límites superiores de su velocidad warp. La nave humana saltó de nuevo al espacio real en cuestión de minutos, en los límites del territorio de los Devoradores.

“Nuestros sensores están detectando una formación de 16 barcos en trayectoria de patrulla, dentro del alcance de las armas, señor”, gritó un joven oficial.

El comandante Rykov asintió. —Muy bien. Quiero que todas las naves, menos una, sean destruidas antes de que sepan qué las golpeó. Desactivaremos la última y la abordaremos. Necesitamos sistemas en línea para que los pulsos electromagnéticos no sean una opción y nos quedemos con las armas convencionales. Vámonos.

Observé por la ventana cómo cientos de misiles se dirigían hacia la flota. Un indicador parpadeó en la pantalla para seguir los objetivos fijados; parecía que la computadora estaba pilotando las armas de forma remota. Las naves de patrulla giraron para enfrentarse a nosotros y dispararon proyectiles cinéticos en un intento de destruir los proyectiles. Sus balas impactaron en algunos misiles, pero con solo unos segundos para reaccionar, no había forma de eliminarlos a todos.

Los explosivos humanos atravesaron los cascos metálicos de los Devoradores como si fueran de papel. La fuerza de múltiples detonaciones simultáneas los destrozó hasta los huesos, arrojando metal deformado en todas direcciones. La única nave que quedó fue la rezagada en la parte trasera de la formación.

Un único proyectil alcanzó al último crucero y le abrió un corte en el costado. No había forma de que la nave pudiera saltar y alejarse mientras expulsaba la atmósfera. Un transporte humano se acercó a la nave averiada. No estaba claro a qué se enfrentaría el grupo de abordaje en el interior, pero después del poder desenfrenado que había presenciado de nuevo, tenía confianza en que cualquier resistencia de los Devoradores sería aplastada sin mayores problemas.

Rykov golpeó el suelo con el pie con impaciencia mientras sus hombres inspeccionaban la nave. “Líder del equipo, informe de situación, por favor”.

—Señor, encontramos a dos combatientes enemigos inconscientes a bordo. Parece que se ha cortado el soporte vital —dijo una voz masculina ronca por el altavoz—. No hemos dañado su ordenador ni su suministro eléctrico. Se lo han provocado ellos mismos.

—¡¿Qué?! ¿Intentar suicidarse en lugar de ser capturado?... —El comandante se quedó en silencio—. Llévalos de vuelta a tu nave de inmediato. Intenta resucitarlos.

“Sí, señor. Estamos en ello”.

Fruncí el ceño, confundida. ¿Por qué los Devoradores desconectarían su soporte vital? Tal vez fuera una cuestión de honor, pero no tenía sentido optar por una asfixia lenta en lugar de una simple bala en el cerebro.

Tenía la esperanza de que los médicos humanos fueran tan competentes como sus soldados. Había muchas preguntas que hacer, pero los hombres muertos no nos darían ninguna respuesta.