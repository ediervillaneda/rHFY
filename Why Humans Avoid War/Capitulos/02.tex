\chapter{Ula POV: La Verdadera Cara de la Humanidad}\label{sec:ula-pov-la-verdadera-cara-de-la-humanidad}

El Senado de la Federación esperaba lo peor cuando llegó el mensajero.

Siguiendo las costumbres galácticas, la nave más rápida fue enviada antes de la flota para proporcionar un informe de primera mano de la batalla a los embajadores. La mirada aterrorizada en el rostro del joven alférez Jatari cuando entró en la cámara del Senado parecía confirmar los temores de todos.

Recordé la transmisión que habíamos recibido hace solo unas horas, detallando la sombría situación de aquellos que se habían enfrentado a los Devoradores. Los números de bajas confirmadas ya habían sido elevados, y sin que ningún miembro de la Federación enviara refuerzos, podríamos estar hablando de una tasa de bajas de hasta el 90%.

Como Presidente, había intentado persuadir a las especies de agresión intermedia para que ofrecieran ayuda, pero todos se negaron rotundamente. Si tuviera el poder para obligarlos a ir, lo haría. Todos conocíamos la estela de destrucción que dejaban los Devoradores a su paso, pero no teníamos más opción que detenerlos. Nos empujarían al borde de la extinción si les permitíamos avanzar por nuestra galaxia.

Sin embargo, hubo algunos puntos extraños en el comportamiento del mensajero. A medida que se acercaba al podio, clavó la mirada en la Embajadora Terrana Nikki Johnson y tragó nerviosamente. Noté que le temblaban las manos. Los Jatari eran una raza orgullosa, impulsada por el honor, que había presenciado los horrores de la guerra una y otra vez. Nunca antes había visto a uno de ellos regresar a casa con aspecto de haber visto un fantasma. Y ¿por qué su fijación estaba en los pacíficos humanos, de todas las razas?

``Uh, hola, S-Senadores. Soy el Alférez Telus''. La mirada del heraldo no se apartó de la Embajadora Johnson. ``Los Devoradores han sido derrotados. No sobrevivió ni una sola de sus naves''.

Murmuros sorprendidos se extendieron por la asamblea. Yo también estaba perplejo; la correspondencia anterior había pintado un panorama desolador para nuestros hombres. Si realmente se había producido un giro tan drástico de los acontecimientos, necesitábamos saber cómo había sucedido. Cualquier táctica que la flota hubiera empleado podía transmitirse a otros comandantes para futuros encuentros.

Una rápida mirada por la habitación reveló que la mayoría de los representantes estaban en estado de confusión. Pero la Embajadora Terrana estaba sonriendo, con un brillo depredador en sus ojos. Había algo en su expresión que me perturbaba en lo más profundo de mi subconsciente.

Me levanté, decidido a restaurar el orden. ``¡Silencio! ¿Cómo es posible? Por favor, explíquese''.

``Bueno, Señora Presidenta... fueron los humanos. Solo enviaron unas pocas naves en nuestra ayuda, pero... construyeron algo terrible''. La voz del Alférez Telus bajó hasta apenas más que un susurro. ``Fue como si hubieran aprovechado una supernova. Nunca en mi vida había visto tanta destrucción''.

El caos se desató cuando exclamaciones de sorpresa se elevaron a un crescendo, y todas las miradas se dirigieron hacia la Embajadora Johnson. Yo también tenía dudas sobre esta versión de la batalla; ¿los humanos, con alguna arma terrible capaz de destruir a los Devoradores? Era de conocimiento común que evitaban la guerra a toda costa.

El Embajador Xanik Cazil rió y levantó un dedo para hablar. ``Con todo respeto, los humanos no son una especie combativa. Astutos, inteligentes, avaros... son todas esas cosas. Pero si tuvieran armas que pudieran exterminar a los Devoradores, serían algo más que charlatanes y diplomáticos. Ya gobernarían la galaxia en este momento''.

Los Xanik estaban en las filas superiores de las especies agresivas, pero también eran el principal socio comercial de la humanidad. La Unión Terrana los había conquistado con su disposición a vender cualquier cosa, a cambio de un precio, y a pesar de sus filosofías divergentes sobre la violencia, los dos poderes se habían convertido en estrechos aliados.

``Estás equivocado. Lo vi con mis propios ojos'', respondió el Alférez Telus. ``La verdad sobre la humanidad es que son asesinos. Son peligrosos. El General cree que deberíamos buscar su amistad, pero no estoy seguro de estar de acuerdo. No confío en ellos''.

Dirigí mi mirada a la Embajadora Johnson. ``Deberíamos dejar que la representante terrana responda. ¿Es esto cierto?''

La Embajadora Johnson suspiró con cansancio. ``Sí, es cierto. La Tierra tiene muchas armas de último recurso guardadas. Somos muy hábiles en la guerra, pero intentamos encontrar una manera diferente''.

``¿Por qué nos presentaron una imagen falsa de su especie?'', le pregunté. ``Hablan de paz y, sin embargo, han estado ocultando las armas más poderosas de la galaxia''.

``Nunca deseamos usarlas'', dijo. ``Su índice de agresión: las especies de alta agresión a menudo son territoriales y buscan conquistar. Si la Federación hubiera examinado nuestra historia, habrían visto que alguna vez fuimos así. Perdimos millones de vidas en guerras entre nuestras facciones, y nos cansamos de tanta derramamiento de sangre.

La humanidad ha intentado ser mejor. Nuestra naturaleza destructiva e impulsiva todavía está ahí, simplemente la hemos enterrado profundamente. Ve, somos la única especie agresiva que tiene un fuerte sentido de empatía. Luchamos constantemente con esa dualidad. Nos controlamos con reglas y, en su mayor parte, elegimos el bien.

Pero conocemos las profundidades de la depravación que existen. Sabíamos que algún día, alguien verdaderamente malvado vendría... y tendríamos que ser peores''.

Asimilé sus palabras, mientras mi mente seguía dando vueltas. ¿Una guerra con su propia especie que tenía millones de bajas? Incluso los conflictos más graves en la historia temprana de los Jatari no llegaban a tener 200,000 muertos, ¡y ellos eran un 15 de 16 en la escala de agresión! La guerra más sangrienta que habíamos conocido anteriormente no se acercaba ni de lejos a la historia vil que había descrito la Embajadora Johnson.

Una especie con tanta propensión a la violencia debería haberse autodestruido. No había forma de que pudieran formar una sociedad funcionante. ¡Y mucho menos pensar que actuaban como pacificadores galácticos! Era difícil conciliar mis experiencias con diplomáticos humanos civilizados y el pasado vil que había descrito la Embajadora.

Sin importar cuánto los humanos afirmaran poder controlar su salvajismo, no podíamos confiar en ellos. Una especie con un impulso tan fuerte hacia la violencia podía fácilmente apuñalarte por la espalda en un momento de ira y no pensar dos veces en ello.

Honestamente, si no temiera represalias, habría propuesto en ese momento expulsar a la Unión Terrana de la Federación. Pero, incluso si era jugar con fuego, probablemente era mejor tenerlos de nuestro lado que tener sus cañones dirigidos contra nosotros. Sin embargo, tendríamos que vigilarlos mucho más de cerca.

Forcé una expresión neutral. ``Nos salvaron de un enemigo al que no podíamos vencer por nosotros mismos. Les debemos una gran deuda. Llevará un tiempo para que la Federación considere completamente lo que nos acaban de contar, pero les agradecemos por poner fin a la guerra''.

Los ojos de la Embajadora Johnson se endurecieron. ``La guerra no ha terminado, Señora Presidenta. Derrotamos a una flota, pero los Devoradores enviarán más si no son eliminados. Y solo volverían más fuertes. La humanidad no espera su bendición, pero les pedimos perdón por lo que estamos a punto de hacer''.

``¿Qué... qué están a punto de hacer?'' pregunté con cautela.

``Iremos a atacar su mundo natal con bombas de antimateria, sin sobrevivientes. Es una solución permanente. Puede que no sea bonito, pero no vemos otras opciones para poner fin al terror que someten al resto del cúmulo'', respondió.

Incluso las especies más agresivas parecían horrorizadas ante la sugerencia. Noté que los embajadores más cercanos a la humana se alejaban, como si tuvieran miedo de que les mordiera.

Sacudí la cabeza con fervor. ``¡Eso es genocidio! La Federación no puede aceptar la erradicación de toda una especie; por favor, intentemos negociar una tregua. Debemos agotar los medios pacíficos antes de siquiera considerar un ataque como este''.

``No se puede razonar con alguien que solo quiere destruirte. Matar o ser matado''. La Embajadora Johnson se levantó de su asiento, recogiendo sus pertenencias. ``¿Cuántas especies inocentes ya han perecido a sus manos? Por lo que a nosotros respecta, es mejor que ellos que nosotros''.

La representante terrana salió del edificio, despidiéndose de la Embajadora Cazil mientras se marchaba. No podía comprender cómo cualquier ser pensante podría estar tan tranquilo y despreocupado ante la perspectiva de arrasar un planeta, incluso el de una raza parasitaria como los Devoradores.

Me pregunté si al menos deberíamos intentar interponernos en el camino de los humanos. Es poco probable que pudiéramos detenerlos, pero al menos podríamos decir que lo intentamos.

Las cosas eran más sencillas cuando creíamos que eran pacíficos. Parte de mí deseaba que esa mentira hubiera durado un poco más. Ya extrañaba a nuestros amigos pacifistas.