\chapter{El Último Sacrificio}\label{sec:el-ultimo-sacrificio}

El destructor funcionó como cebo, en cierto modo, obligando a la flota humana a extenderse demasiado. Atrajo a los acorazados terrestres más grandes y feroces hacia sus inmediaciones y los mantuvo alejados de la refriega más grande. Si bien nuestros grandes cañones atacaron la nave del fin del mundo con fuego láser y misiles, eran blancos fáciles para el enemigo. Si esos tipos no podían atravesar los escudos del destructor, ninguno de nuestros barcos podría hacerlo. Decidí que nuestra mejor jugada era cubrirlos y ganar un poco más de tiempo.

Noté que el enemigo se acercaba en masa a un acorazado humano en particular, un gigante de clase titán. Parecía haber sufrido graves daños, pero seguía avanzando lentamente hacia el destructor de planetas. ¿Quizás estaban planeando un bombardeo a quemarropa? Sea como fuere, nuestros enemigos claramente no querían que llegara a su destino, o de lo contrario no habrían desviado 15 naves para atacar a una sola nave.

Eché un vistazo a los datos de los sensores de nuestra computadora y me quedé pensando dos veces. Las lecturas de energía de ese acorazado terrestre estaban fuera de serie; el motor warp parecía estar desestabilizando. Mi primer pensamiento fue que un disparo enemigo había alcanzado el reactor, pero la computadora solo detectó daños menores en ese compartimento. Más bien, parecía que los humanos habían apagado el sistema de enfriamiento, como si estuvieran tratando de causar una fusión cuántica a propósito.

—¿Qué están haciendo? Su reactor va a entrar en estado crítico —susurré.

Carl miró su propia pantalla y se rió entre dientes. ``Maldito. Están usando la nave como bomba''.

``Eso es suicidio.''

—No, eso es sacrificio. Hay una diferencia. ¿Por qué no les damos una mano a esos muchachos?

Preparé las armas cinéticas sin hacer más comentarios; no era el momento de pensar en el coraje humano. Nos lanzamos a la batalla y disparamos al enemigo desde su punto ciego. Algunas de las naves se fijaron en nosotros, lo que dio a nuestros aliados un pequeño respiro. Su primer instinto probablemente fue el de acabar con nosotros con proyectiles de plasma, pero Carl nos había acercado peligrosamente a su posición. Cualquier disparo que nos hicieran correría el riesgo de alcanzar a sus propias naves.

No es que eso fuera a hacer ninguna diferencia. Nuestros oponentes tenían armas de precisión y nos atacarían sin pestañear. Cuando dos misiles antimateria nos apuntaron, me preparé para el destello final y esperé que hubiéramos hecho lo suficiente. En el estado debilitado de nuestra nave, no habría maniobras evasivas...

—¡Despliega los últimos minimisiles, Byem! —gritó Carl.

Mis dedos obedecieron antes de que mi cerebro procesara su pedido. Si su propósito original era interceptar, tal vez podrían detener los proyectiles que se aproximaban. Observé con la respiración contenida cómo las diminutas ojivas trazaban un arco a través del espacio y se colocaban en el camino de los misiles. La explosión de antimateria se activó antes de alcanzarnos, pero sabía que no tendríamos tanta suerte la próxima vez.

Nuestra nave se desplazó hacia atrás antes de que pudieran volver a fijarnos como objetivo y comenzó a dar vueltas para pasar de nuevo. Podía sentir cómo la nave se estremecía cada vez que girábamos; en su estado actual, estábamos poniendo a prueba sus capacidades. El panel de luces de advertencia parecía arte abstracto, con más indicadores coloridos que parpadeaban a cada segundo.

La mayoría de las naves enemigas habían vuelto a centrarse en el acorazado suicida; les resultó bastante fácil deshacerse de cualquier tirador terrestre que acudiera en ayuda de su aliado. Sin embargo, uno de sus cazas había seguido nuestro movimiento y se había desviado del grupo. Se dirigía hacia nosotros a toda velocidad, con las armas de plasma preparadas para matarnos. Lo apunté en la mira, resignado a otro tiroteo. En ese momento, se trataba más de llevarnos a tantos como fuera posible, en lugar de sobrevivir.

Mi resignación se transformó en puro terror cuando vi una sonrisa depredadora y enloquecida en el rostro de Carl. Los humanos solían tener esa expresión justo antes de hacer algo insano.

—Juguemos un poco al juego de la gallina, ¿vale? —gruñó.

Un grito agudo escapó de mi garganta cuando nuestra nave aceleró a toda velocidad y me arrojó hacia atrás en mi asiento. Las tablas del piso temblaron bajo mis pies y noté que aparecían nuevas advertencias en la columna del motor. ¿Qué era exactamente ese juego de ''gallinas'' al que se refería el humano? Estaba bastante seguro de que no me gustaba, ya que parecía implicar jugar con nuestras vidas.

Nos desviamos hacia el camino de nuestro adversario, lo que nos puso a pocos minutos de una colisión frontal. Todos los pensamientos coherentes se evaporaron cuando el caza enemigo nos miró fijamente. A pesar de nuestra inminente muerte, Carl no hizo ningún intento de reducir la velocidad ni de darse la vuelta. Mis instintos me gritaban que me lanzara hacia adelante, agarrara la columna de control y nos alejara del peligro, pero no podía moverme.

En el último segundo posible, nuestro oponente se desvió de su curso. Exhalé un suspiro de alivio al darme cuenta de que no había respirado demasiado tiempo. No era de extrañar que estuviera tan mareado. Todo mi cuerpo temblaba, abrumado por los químicos del miedo que bombeaban por mis venas.

Mientras tanto, Carl aplaudió y sonrió de oreja a oreja. ``¡Cobardes! Anímate, Byem, ellos parpadearon, no nosotros''.

Me masajeé las sienes y gemí. ``Carl… no tengo ningún problema con morir, pero no tenemos por qué alentarlo activamente''.

—Tonterías. Como decimos en la Tierra, prefiero morir con una explosión que con un gemido —dijo—. Nuestros amigos embestirán a esos cabrones en treinta segundos, como máximo. Así que les daremos con todo lo que tengamos.

La ruta de regreso al costado del acorazado terrestre quedó despejada después de nuestro pequeño enfrentamiento. Pasamos volando junto a sus asaltantes y descargué toda la munición que llevábamos, incluso las que sabía que no funcionarían. Ninguno de nuestros impactos causó daños graves, pero afortunadamente, había llegado ayuda. La mayoría de las naves humanas restantes se habían unido a la misión suicida, protegiendo el acorazado a toda costa.

Vi a algunos lanzarse a la línea de fuego, usando a sus débiles escaramuzadores como escudos. Otros embistieron contra las naves enemigas, empujándolas lejos del acorazado. Los humanos se mostraron inflexibles ante la muerte, sin inmutarse cuando sus aliados cayeron a su alrededor. Cuando se vieron presionados contra una pared, su sed de sangre rozó la locura.

Agoté las últimas municiones que nos quedaban y observé cómo el acorazado cruzaba el tramo final. Hundió el morro en el costado del destructor y el impacto fue suficiente para que su reactor desestabilizado se desplomara. La fusión liberó cantidades asombrosas de energía en el interior del destructor de planetas y, para mi asombro, estalló por las costuras.

El destructor, que antes era intocable, se fracturó en varios pedazos que comenzaron a desmoronarse. Nuestros sensores detectaron fluctuaciones magnéticas que indicaban que su propio reactor estaba perdiendo el control. Se tambaleó al borde del orden y la destrucción durante unos tensos momentos, antes de desbordarse en el caos. Los restos de la poderosa nave se fusionaron en una monstruosidad deformada, que se dobló como una bola de papel.

Nuestras comunicaciones cobraron vida y me sobresalté. Lo que sea que había estado bloqueando nuestras transmisiones debía estar a bordo del destructor. La voz pertenecía al mismo oficial que había intentado detenernos antes. ``A todas las naves terrestres, se les ordena retirarse. Abandonen el sistema, repito, abandonen el sistema. Envíen un mensaje a la Tierra a toda costa''.

Me sorprendí al escuchar la palabra ``retiro''
 en nuestro feed; no estaba seguro de que la palabra existiera en el léxico humano.

Carl se inclinó hacia delante con el ceño fruncido. —Señor, ¿qué pasa con los refugiados?

—Oh, eres tú —el tono del oficial transmitía una mezcla de desdén y decepción—. Nos habéis dado algo de tiempo. Estamos enviando transportes ahora. No sé si podremos salir antes de que nos derriben.

``Los Devoradores estarán sobre ti en segundos. Están en todas partes. Necesitas una escolta…''

``Probablemente vamos a morir. Lo sé. Si te quedas, tú también morirás, muchacho bonito''.

Mi amigo sacudió la cabeza y apagó la radio con un suspiro de frustración. Me miró con compasión. Sin duda era consciente de nuestra falta de munición y de los daños que había sufrido nuestra nave; no había nada más que pudiéramos lograr. Habíamos hecho todo lo que podíamos. Cualquier ser razonable se marcharía y viviría para luchar otro día.

—Tiene razón, Byem —dijo finalmente Carl—. Si pensara que haría alguna diferencia, me quedaría. Pero con esta nave...

Traté de reprimir la culpa que nublaba mi mente. No sería justo pedirle a Carl que muriera de una manera inútil, después de todo, solo porque tenía la muerte de mi hijo en mi conciencia. Pero supe en ese momento que abandonar a mi gente me perseguiría por el resto de mi vida. Que incluso si los humanos ganaban en el futuro, yo sería el único sobreviviente de una especie extinta.

—Por supuesto. Lo sé. —Me sorprendió lo brusco que fue mi tono, plano y sin emociones. Ni siquiera sonaba como mi voz—. A la Tierra, entonces.

Mi amigo introdujo las coordenadas a toda prisa, ya que no teníamos ni un minuto que perder. Un grupo de naves enemigas se acercaba a nuestra posición y nos rodeaba por todos lados. La mayoría de nuestros aliados supervivientes ya habían saltado, lo que convertía a los rezagados en blancos fáciles. Esperaba que nuestro motor warp siguiera funcionando, porque si no, estábamos a punto de convertirnos en polvo.

Nuestro entorno titiló mientras nos deslizábamos hacia el hiperespacio y el campo de batalla se desvaneció. Mi mente se desvió hacia pensamientos de venganza, la ira ardía en mi pecho. Alguien, o algo, tenía que pagar por lo que había sucedido hoy.