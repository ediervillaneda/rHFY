\chapter{POV Kilon: Despertar de una Pesadilla}

Una mano humana se abatió sobre mi hombro, sacándome de golpe de mi sueño.

—Despierta, General —la voz era firme e insistente—. Necesitamos hablar.

Gruñí, echando un vistazo al reloj. Apenas había pasado una hora desde que me quedé dormido y necesitaba desesperadamente el descanso. Por mucho que me gustara el Comandante Rykov, esperaba que tuviera una buena razón para despertarme.

—¿No puede esperar hasta más tarde? Es que...

—No. Tenemos un problema. Uno muy grave.

¿Era... miedo en su voz? Mi cerebro se puso en alerta en un instante, y me levanté de la cama sin más protestas. Si eso asustaba al Comandante, solo podía significar una de dos cosas. O los humanos habían hecho algo terrible, o había una amenaza apocalíptica en el horizonte. No estaba seguro de cuál me preocupaba más.

Rykov me guió fuera de mis aposentos, abriéndose paso por los sinuosos corredores del buque insignia. Su silencio era inquietante y bastante fuera de lo común. Por tentador que fuera exigir respuestas, tenía la sensación de que las obtendría pronto.

Entramos en una sala de conferencias, donde nos esperaban dos individuos. Reconocí al refugiado Devorador de nuestras aventuras anteriores, pero no al rubio humano sentado junto a él. El humano estaba encorvado en su silla, luciendo derrotado y exhausto. Byem tenía una mirada distante en el rostro, sin reaccionar siquiera ante nuestra presencia.

El Comandante carraspeó, frunciendo el ceño con disgusto.

—Capitán Larsson, por favor repita lo que me dijo. Incluyendo la parte sobre abandonar a un campamento de refugiados a su suerte.

El Capitán Larsson dio un respingo como si lo hubieran abofeteado.

—Con todo respeto, señor, nuestro caza estaba averiado y sin munición. Habría sido un ejercicio inútil...

—No quiero excusas. Empiece desde el principio —interrumpió Rykov.

—Bueno, para resumir, establecimos un campamento de refugiados en una base militar, que los Devoradores atacaron sin previo aviso. Mejoraron sus escudos. Ninguna de nuestras armas principales funcionaba, y la mayoría de nuestra flota fue aniquilada en acción —el capitán rubio hizo una pausa, evaluando mi reacción. Logré mantener mi expresión neutral, pero un nudo de miedo se formaba en el fondo de mi estómago—. Nuestro comandante nos ordenó retirarnos del sistema. Creemos que estarán en camino hacia la Tierra en breve.

Apenas habían pasado días desde la victoria decisiva de los Terrícolas sobre los Devoradores, ¿y ya habían vuelto ineficaces sus armas? Me estremecí al recordar lo fácilmente que los humanos habían atravesado las defensas de la capital; el arsenal de la Federación parecía juguetes en comparación. Si ninguna de sus potencias de fuego funcionaba... no teníamos ninguna oportunidad.

No pude evitar sentirme responsable, pues yo fui quien persuadió a Rykov de no arrasar su mundo. En retrospectiva, tal vez los Devoradores deberían haber sido eliminados cuando la oportunidad lo permitió. Habría sido la opción pragmática, aunque no la moral.

Por lo que recordaba, la Embajadora Johnson dijo algo así como "Prefiero que ellos mueran a que nosotros lo hagamos", en su infame discurso ante el Senado. Ahora, con toda la galaxia enfrentando la extinción, tal vez la Federación entendería esas palabras.

—¿Tenía esta base armas nanitas? —pregunté.

El Capitán Larsson suspiró.

—Por supuesto. Todas nuestras bases tienen.

Ni siquiera quería abordar las implicaciones de ese comentario, pues significaba que los humanos poseían miles de esos misiles que doblan la realidad. Un arsenal de ese tamaño... ¿estaban los terrícolas planeando nivelar una pequeña galaxia o algo así?

—Tiene que haber alguna debilidad. Algún punto débil —reflexioné.

—Armas más pequeñas. No exactamente poderosas, aunque —Larsson tamborileó los dedos sobre la mesa, con los ojos perdidos en sus pensamientos—. Son vulnerables a las tácticas de embestida. Aunque no lo intentaría primero.

—Esas opciones son menos que ideales —convine—. ¿Qué tan pronto llegarán a la Tierra?

—Mi mejor suposición, unas pocas horas.

Bien. Estábamos condenados.

Miré al Comandante Rykov.

—Por favor, dime que tienes uno de tus terribles planes.

—Bueno... —Ni siquiera una sonrisa de parte del humano de cabello oscuro. Evitaba el contacto visual, lo cual no era una buena señal—. Creo que tenemos que sacar las armas del juicio final.

Casi caí en shock, mientras trataba de procesar sus palabras. ¿Esas bombas nanitas, que podían desintegrar toda una flota, no eran sus armas de último recurso? Según sus simulaciones, esas tenían un cinco por ciento de posibilidad de destruir el universo; lo que Rykov mencionaba debía ser verdaderamente espantoso.

—Estoy seguro de que esto me encantará —mascullé—. ¿Qué exactamente son?

El Comandante miró al silencioso Byem, su ceño frunciéndose aún más.

—Les llamamos bombas gravitatorias. Se pueden usar para provocar una explosión estelar. Algunos en la Federación nos verán como monstruos, ya lo hacen. Pero necesitamos terminar el trabajo.

La humanidad tenía el poder de apagar estrellas. Podían convertir sistemas enteros en inhabitables en un instante, condenando especies con un simple movimiento de mano. Para entonces, estaba bien consciente de que eran una especie peligrosa para hacer amistades, pero su capacidad de violencia nunca dejaba de sorprenderme. ¿Por qué incluso soñarían con algo así durante siglos de tiempo de paz?

Sin duda, los líderes de la Federación harían la misma pregunta, si este plan funcionaba, y quizás con razón. Al menos estarían allí para plantear esas preocupaciones. No veía otra manera adelante; si no se eliminaba la IA Devoradora, serían nuestros mundos los reducidos a polvo en su lugar.

El Capitán Larsson se inclinó, dando un golpecito en el hombro a Byem.

—¿Qué piensas?

—Hazlo —el refugiado Devorador se movió ligeramente, con el rostro desprovisto de emoción—. Hay destinos peores que la muerte. Hemos sufrido lo suficiente.

—¿Y usted, General? ¿Tengo su bendición? —preguntó Rykov.

—Siempre. Pero una cosa... ¿los Devoradores no nos verán venir? ¿Intentarán detenernos?

—Estoy seguro de que han dejado algunas naves atrás. Tendremos que abrirnos camino a través de ellas.

—¿Y qué hay de los que se dirigen hacia la Tierra?

—Según Byem, están programados con un "interruptor de muerte" si pierden contacto con el mundo natal. Eliminamos la IA, matamos dos pájaros de un tiro.

Hice una mueca. Lo que estábamos a punto de hacer equivalía a genocidio, y no puedo decir que me sentía bien con mi conciencia. Millones de Devoradores esclavizados perecerían junto con la IA, y no habría oportunidad para que los rescatáramos a ninguno de ellos. A pesar de sus palabras tranquilas, la tensión en el rostro del Comandante me decía que él sentía la misma carga.

—Tiene que hacerse. Deberíamos partir lo antes posible —dije.

Rykov asintió.

—Muy bien. Larsson, Byem, desembarquen de inmediato. Mis guardias los escoltarán.

—Pero señor, queremos unirnos a usted —protestó el Capitán.

—Esa fue una orden. Ambos lucen

como el infierno —el Comandante se volvió hacia mí, y noté los círculos oscuros bajo sus ojos. Él mismo no se veía tan bien—. General, sígame al puente. Tengo un asunto privado que discutir en el camino.

Lo seguí, encontrando mucho más fácil mantener el paso con sus zancadas rápidas que antes. La terapia con nanites debió haber tenido un impacto positivo en mi condición física; ciertamente me sentía más fuerte, incluso más joven. Eso me recordó que necesitaba negociar para llevar esa tecnología a casa. Podría mejorar la calidad de vida, revolucionar el campo de la medicina y fortalecer nuestras fuerzas. Con las innovaciones humanas en nuestro bolsillo, podríamos traer la nueva edad dorada de los Jatari.

—Quizás no sea el momento adecuado, pero... —vacilé, tratando de encontrar las palabras adecuadas—. Si sobrevivimos a todo esto, estaba pensando que los Jatari podrían ayudar con su investigación de nanites. Si quieren socios, claro.

—Sí, eso es lo que quería hablar contigo también —el Comandante se pasó una mano por el cabello, suspirando—. Te prometí no más tonterías, así que aquí está la verdad. Entiendes lo peligrosa que puede ser la nanotecnología. Lo último que la Tierra quiere es que caiga en manos equivocadas, así que no quieren que salga de nuestro territorio bajo ninguna circunstancia.

—Ya veo. Eso no incluiría a mí, ¿verdad? —el humano guardó silencio—. Vaya. ¿Soy un prisionero?

—No haría eso contigo. Pero... otros lo harán. Me ha gustado tenerte a bordo, y pensé que tal vez podrías quedarte. Como mi primer oficial.

—¿Estás sugiriendo que me una al ejército Terrano?

—Sí.

Un puesto de mando sonaba mejor que una celda oscura, eso seguro. Una estancia permanente con los humanos no era exactamente lo que tenía en mente, pero no parecía que tuviera elección en el asunto. Claramente, Rykov había sabido lo que implicaría exponerme a los nanites, pero no podía culparlo, ya que la alternativa habría sido dejarme morir.

Sin embargo, la idea de no volver a ver mi hogar era casi demasiado para soportar. Y algo me decía que los humanos harían un desfile público de mi traslado, utilizándolo para fines de relaciones públicas. Mi legado sería destruido, mi nombre maldecido y difamado entre mi gente. Sería recordado como el general que abandonó el barco al primer signo de problemas, y eso, realmente no podría vivirlo.

—No es mi primera opción —dije.

—Lo sé. Lo siento.

—¿Sabes lo que hacen mi pueblo a los traidores, Comandante? Porque así es como verán la deserción. A los que capturan, los desollan de la cabeza a los pies, luego vierten ácido dentro de sus cuerpos. Y tu linaje se considera manchado durante cien generaciones, tu familia burlada y golpeada. Cualquier hazaña que hayas logrado, quemada de las páginas de la historia, tu nombre solo se pronuncia con los peores insultos. Sería mejor morir.

—Entonces les decimos que has muerto. Con honor. Ellos no saben que estás vivo.

—Eso... podría ser aceptable. Lo pensaré.

Mis lealtades primarias siempre estarían con los Jatari, pero tal vez podría encontrar una manera de proteger sus intereses desde lejos. Del mismo modo en que los humanos vigilaban a sus vecinos. Si dependiera de mí, nuestras especies trabajarían juntas. Pero si la Unión Terrana estaba tan decidida a mantener su progreso para sí misma, se necesitaba un enfoque diferente.

Ninguna de esas preocupaciones sería relevante si fracasábamos en nuestra misión, por supuesto. El destino de la galaxia dependía de las próximas horas, y sería agradable si solo una vez, todo saliera según lo planeado.