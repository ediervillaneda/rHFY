\chapter{El Último Enfrentamiento}\label{sec:el-ultimo-enfrentamiento}

Mientras nos precipitábamos hacia una colisión inminente, los sonidos de la tos llenaron el puente. Un escalofrío involuntario recorrió mi columna vertebral; aunque sabía que era algo bastante común en su especie, prácticamente podía sentir el traqueteo de sus pulmones.

Las condiciones cada vez más deterioradas en el interior del buque insignia también me afectaban a mí. Me ardían los ojos por las columnas de humo que se formaban en el aire y mi piel ardía por la temperatura, que había aumentado hasta un nivel incómodo. Eché un vistazo a mi alrededor y vi que muchos humanos se tiraban del cuello de las camisas. Ninguno de ellos parecía demasiado contento con el estado medioambiental. Si nos quedábamos a bordo de esta nave demasiado tiempo más, todos nos asfixiaríamos.

Con los escudos bajados y el interior en llamas, esperaba que la nave se partiera en dos con el impacto. Pero los terranos, fieles a sus tendencias suicidas, siguieron adelante de todos modos. Los cinco cruceros Devoradores no se movieron cuando nos acercamos a ellos; solo parecieron apiñarse más cerca. Como última línea de defensa, tenían que detenernos, incluso a costa de lo máximo.

—¿No protesta, general? —Rykov tenía la nariz hinchada hasta el tamaño de una fruta pequeña, pero de todos modos sonreía—. Esperaba la palabra «basta» al menos una vez.

Negué con la cabeza. ``¿Por qué molestarme? No quiero desperdiciar mis últimos segundos de existencia''.

—Eres bastante cínico, ¿sabes? Diablos, podríamos sobrevivir a esto —dijo riendo.

Mis antenas se movieron confusas. ``¿Se supone que eso es tranquilizador?''

El amplio casco de nuestra nave se hundió en la línea enemiga. Incluso con los amortiguadores de inercia en funcionamiento, una sacudida me impulsó varios largos hacia adelante. Me encontré boca abajo en el suelo una vez más, haciendo muecas por otro aterrizaje brusco. Los nanocitos en mis venas solo podían reparar una cierta cantidad de daño tisular a la vez; imaginé que mañana estaría negro y verde.

Antes de que pudiera recuperar el equilibrio, los humanos hicieron que la nave se desviara bruscamente hacia un lado. Se me escapó un grito indigno cuando me deslicé de hombros hacia una estación de trabajo. ¿Qué estaban haciendo? ¿Se habían vuelto locos sus sistemas de dirección?

Sentí que el buque insignia empezaba a inclinarse en la otra dirección y me agarré a la base del escritorio con todas mis fuerzas. Un chirrido espantoso resonó en las paredes de nuestra nave, probablemente porque nuestra dura piel se estaba desprendiendo. Pero no hubo ninguna señal de preocupación por parte de los pilotos ni de la computadora, por lo que la rotación errática tenía que ser intencional.

¿Por qué los humanos querrían dar volteretas justo después de una colisión?

Reflexioné sobre la pregunta durante un momento y, de repente, se me ocurrió una idea. Los terrícolas estaban retorciendo su nave como si fuera un cuchillo, abriendo un tajo más profundo. Nos estábamos abriendo paso a través de las naves enemigas y destrozando sus mecanismos internos a medida que avanzábamos.

Un crujido resonó en la pared exterior del puente y atrajo mi atención. Vi una grieta que serpenteaba desde el suelo hasta el techo, con varios tentáculos que se ramificaban desde la raíz. Claro, nuestros oponentes habían quedado fuera de combate, pero nuestra propia nave pendía de un hilo.

``Se han reportado daños graves en todos los sectores. La integridad estructural está comprometida''. La voz de la computadora cortó el aire muerto. ``Recomienden apagar la nave de inmediato''.

Rykov se aclaró la garganta. —Ordenador, ¿ya no recomiendas la evacuación?

``Negativo. Eso ya no es posible'', fue la respuesta.

``¿Por qué no, exactamente?''

``Los hangares y las cápsulas de escape han sufrido graves daños por el fuego. No funcionan. Se recomienda enviar una señal de socorro y esperar el rescate''.

El fuego se había propagado rápidamente y, si continuaba al ritmo actual, alcanzaría el puente en cuestión de minutos. Nos habíamos detenido en seco tras la colisión y ahora estábamos encajados de morro contra las naves enemigas. Cualquier otro impacto fracturaría el exterior de la nave insignia, por lo que atravesar el otro lado no era una opción.

El comandante se rascó la cabeza y frunció el ceño. —Bien, ya has oído lo que ha dicho la computadora. Toda la potencia a los propulsores y, una vez que estemos fuera de este naufragio, nos dirigiremos a toda velocidad hacia la estrella del Devorador. Debería ser pan comido.

Quizás no fui el mejor escuchando pero… eso definitivamente no fue lo que dijo la computadora.

Nuestros motores, que ya estaban a punto de licuarse, cobraron vida. Avanzamos lentamente, abriéndonos paso a través de los últimos blindajes del enemigo. El techo que teníamos encima empezó a ceder a medida que su soporte flaqueaba, y supuse que caería sobre nosotros. Lo que más me dolió fue saber lo cerca que habíamos estado de completar la misión y que, de todos modos, no habíamos logrado completarla.

Entonces, con un terrible temblor, el buque insignia quedó libre de los restos. Había cortado cinco naves enemigas por la mitad, dejando atrás solo escombros inactivos. Nuestro armazón se estremeció, todavía decidiendo si colapsar o no. Por algún milagro, se mantenía unido, por ahora.

Mientras los aplausos estallaban en el puente, miré al comandante. El hombre había encontrado un asiento detrás de una consola de armas, desplazando a uno de sus alféreces. Fue cuestión de segundos hasta que pasamos por la estrella de los Devoradores, y él se estaba asegurando de que la bomba gravitacional estuviera preparada para su despliegue. La terrible tarea del genocidio aún estaba por delante de nosotros.

No había en su rostro el menor asomo de sonrisa, a pesar de nuestra dura victoria. Tenía la mirada fija en la ventana, hacia la serena esfera azul y blanca que colgaba a lo lejos. Una lágrima le rodó por el rostro y sospeché que no tenía nada que ver con sus heridas físicas.

La compasión humana era tan misteriosa y tan antitética a sus instintos violentos. La carga de masacrar a una especie entera devoraría a Rykov por dentro; él mismo lo había admitido. ¿Por qué un hombre que despreciaba tan claramente esta misión tomaría el timón para ejecutarla?

La única respuesta que obtuve fue que deseaba salvar a sus subordinados de la culpa. Noble hasta el final. Su discurso anterior sobre el sacrificio todavía resonaba en mis oídos. No se vuelve siendo la misma persona, dijo.

No quería que cambiara. No quería que se pareciera en nada a esos crueles humanos que mutilaban por deporte y disfrutaban de la matanza.

Sin pensarlo, agarré a Rykov por los hombros y lo tiré al suelo sin contemplaciones. Los miembros de la tripulación más cercanos se pusieron tensos ante mi ataque no provocado a su líder, pero los ignoré. Mis dedos actuaron en piloto automático. El primer paso fue confirmar el objetivo, luego...

—¿Qué...? —El comandante se puso de pie, luciendo más confundido que enojado. Notó que algunas pistolas me apuntaban y les hizo una señal a sus hombres para que se retiraran—. Realmente no deberían atacar a alguien en su propia nave. Háganse a un lado, ahora.

—No —murmuré.

La incredulidad brilló en sus ojos. ``¿Qué quieres decir con ' no'?''

—No dejaré que te hagas esto. Te mereces algo mejor. —Mi estómago se revolvió mientras nos deslizábamos hacia la posición, frente a la estrella naranja vibrante. Era una sensación terrible tener el poder de la aniquilación al alcance de la mano—. Que esto caiga sobre mi conciencia.

``Aprecio la idea, pero soy yo quien da las órdenes. De cualquier manera, es mi conciencia quien debe decidirlo, general'', respondió.

—El impacto es distinto cuando se aprieta el gatillo. Supongo que un soldado debería saberlo. —Mi mano temblaba y mi voz apenas era más que un susurro. Presioné el mecanismo de disparo antes de poder convencerme de que no lo era—. Por cierto, ya no se trata de «general». Para ti, es primer oficial o, simplemente, Kilon.

Me quedé desconcertado cuando el comandante Rykov me abrazó. ¿Por qué los humanos demostraban afecto intentando asfixiar a la gente? Apreté los dientes y resistí el impulso de soltarme de su agarre. Se acabaría pronto, ¿no?

—Eres un buen amigo, Kilon —dijo.

Suspiré. ``Sí, pero hueles a vinagre. Quítate de encima''.

El humano se retiró, dando un paso atrás. ``Eso no es muy agradable''.

—Lo sé. Es parte de mi encanto —respondí.

—Está bien, dejen de mirarme, todos. Es hora de ponerse en marcha. —El comandante echó una mirada severa al puente y los miembros de la tripulación volvieron a trabajar—. Queremos que esta estrella se vaya hace tiempo cuando explote. Esperemos que tengamos suficiente combustible para llegar a casa.

Deslizarse al hiperespacio siempre fue duro, pero esta vez superó todas las experiencias anteriores. La presión que se condensaba en mis canales auditivos me quitó la audición y sentí como si mi estómago estuviera hecho un nudo. Me sacaron cada partícula de aire de los pulmones y mis pies estaban pegados al suelo.

Me sentía totalmente impotente, atrapada en un cuerpo sin vida. Quería gritar y luego vomitar, tal vez las dos cosas a la vez, pero tenía los labios apretados. Los minutos se hicieron más largos y parecieron horas, y mi desesperación se convirtió en un frenesí.

La parálisis se disipó cuando salimos, dentro del alcance de la Tierra. Caí de rodillas y esparcí mis entrañas por el suelo. La vergüenza corrió por mis venas, hasta que me di cuenta de que la mayoría de los humanos también estaban doblados por la mitad. Había una razón por la que se suponía que las naves con problemas de regulación ambiental no debían ingresar al hiperespacio. Si el sistema no hubiera estado a punto de convertirse en una nova, no creo que ni siquiera su especie trastornada lo hubiera intentado.

Cuando recuperé el sentido, la preocupación más importante cruzó por mi mente. ¿Habíamos dejado caer nuestra carga a tiempo para salvar la Tierra? No había señales de la nave espacial Devorador en los planetas exteriores, lo que significaba una de dos cosas: o nunca habían llegado al sistema de los terrícolas o ya lo habían tomado.

``Nave desconocida, estás invadiendo el espacio terrestre. Tu transpondedor está fuera de línea, lo que viola las leyes planetarias. Identifícate de inmediato'', dijo una voz masculina por los altavoces.

Rykov se tambaleó hasta la pantalla holográfica. —Comando Orbital, espero que reconozcáis vuestra propia nave insignia.

—Dios mío —el tono del hombre se elevó bruscamente, probablemente por la sorpresa—. ¿Rykov? ¿Qué demonios te ha pasado? Parece que tu pájaro ha sido mordido y escupido.

``Uh, lo estrellamos a toda velocidad, pero ese no es el punto. Solicitamos un aterrizaje de emergencia en la Estación Marte, necesitamos vehículos de emergencia en el lugar''.

``Entendido. Tienes autorización para atracar en cualquier puerto abierto''.

Una extraña mezcla de alivio y tristeza nubló mi mente mientras nos deslizábamos hacia nuestro destino. Saber que nuestros esfuerzos habían salvado a la Federación de una muerte segura era motivo de orgullo, pero deseaba que hubiera habido otra manera. Si no fuera por nuestra arrogancia, si hubiéramos terminado lo que comenzamos en esa misión de rescate hace poco tiempo... tal vez podría haberla habido.

La humanidad tendría que rendir cuentas ante sus vecinos galácticos cuando supieran lo que habíamos hecho hoy. Tal vez, sólo tal vez, las dos partes podrían aprender de sus errores pasados y encontrar un camino hacia la reconciliación. Si la Federación insistía en convertir a los humanos en enemigos, podrían obtener más de lo que esperaban.

Esta guerra había terminado, pero la siguiente…esa era la que realmente temía.