\chapter{La Mentalidad De Un Degenerado}\label{sec:la-mentalidad-de-un-degenerado}

¿Los humanos estaban recibiendo realmente una recompensa por el genocidio de una especie entera? La Federación bien podría eliminar las palabras ``paz'' e ``igualdad'' de su declaración de misión, si el Portavoz estaba poniendo a esos salvajes a cargo del ejército.

Una raza abominable como la suya no debería ocupar puestos de poder, y eso nunca habría sucedido bajo mi liderazgo. Hice todo lo que pude para exponer su naturaleza, y justo cuando pensaba que estaba llegando a alguna parte, la narrativa cambió. Los humanos me acusaron de ser el villano, y la Federación se lo creyó a pies juntillas. En un momento, yo era su amado Portavoz, campeón de la democracia, protector de los inocentes; al siguiente, me echaron a patadas. Todo sobre la base de unos cuantos memorandos desagradables de hace años.

¿En qué me equivoqué? ¿Había algo que podría haber hecho de manera diferente?

Tendría mucho tiempo para pensar después de limpiar mi oficina. Rebusqué en el último cajón de mi escritorio, buscando algo que valiera la pena llevar a casa. Debajo de una pila de documentos, había una sola foto enmarcada. Estaba boca abajo y cubierta por una gruesa capa de polvo.

Era evidente que hacía años que no tocaba esta foto. Sentí curiosidad en el pecho y le di la vuelta.

Una versión más joven de mí estaba de pie al lado del embajador Johnson, sosteniendo un documento. Recordé ese día; habíamos estado en la firma de un tratado sobre crímenes de guerra, que los humanos patrocinaron. Dijeron que querían mitigar el sufrimiento y habían parecido muy genuinos en su compromiso con la paz. En ese momento, yo quería ser como ellos.

Un gruñido retumbó en mi pecho y arrojé la foto al suelo. El marco se rompió y los cristales se esparcieron por todas partes.

``Ten cuidado. No querrás pisar eso mientras sales de esta oficina, por última vez''.

Se me escapó una maldición cuando miré hacia atrás y vi a la embajadora Johnson apoyada contra el marco de la puerta. Era la última persona con la que quería hablar. No tenía idea de cuánto tiempo había estado allí, pero su sonrisa sugería que ya había visto suficiente.

—¿Vienes a regodearte? Ya conseguiste lo que querías, déjame en paz —espeté.

La humana desapareció y, por un breve momento, pensé que tal vez me dejaría en paz. En cambio, regresó con una escoba y un recogedor y comenzó a limpiar los fragmentos de vidrio.

Apreté los dientes, asqueado por su proximidad. ``Tú y tu… asquerosa especie arruinaron todo. Mi vida, mi gobierno, mi trabajo...''


—No es tan malo. Hay muchos puestos vacantes —dijo alegremente—. Sabes, he oído que están contratando en Galaxymart. Me lo imagino, con un bonito chaleco verde reponiendo estanterías. Te quedaría muy bien.

—¡Vete a la mierda! ¡SAL DE AQUÍ! —grité.

El embajador Johnson se rió entre dientes y, por fin, salió de mi despacho. Recogí mi caja del suelo y miré a mi alrededor por última vez. Se suponía que ésta iba a ser mi vida, pero, de alguna manera, me la habían arrebatado.

Pero no importaba. Me ganaría el apoyo de los ciudadanos caprichosos difundiendo la verdad sobre los humanos a cualquiera que quisiera escucharme. Tal vez podría crear un blog o aparecer como invitado en los programas de entrevistas. Lo que fuera necesario para que el mensaje llegara a todos.

Llamaron a la puerta y me erizó la piel de fastidio. Era un gesto humano para solicitar la entrada, lo que significaba que el embajador Johnson había regresado. Maravilloso.

—Estúpido humano. Ya te has divertido lo suficiente, me voy —murmuré.

Una voz masculina, fría como el hielo, respondió: ``No soy humano y tú no irás a ninguna parte''.

Algo suave y metálico se presionó en mi nuca, lo cual parecía una pistola. El miedo se apoderó de mi cuerpo cuando me di cuenta de lo que estaba sucediendo. Los humanos habían enviado a un asesino a buscarme, ¿no? No podía decir que me sorprendiera, pero no podía entender por qué no apretó el gatillo.

Me volví para enfrentar a mi agresor con movimientos lentos y no amenazantes. Para mi sorpresa, estaba diciendo la verdad: no era humano. Tampoco reconocí su especie. No había ningún bípedo de piel morada y nariz chata en el registro de la Federación, que yo supiera.

``¿Quién eres tú?'', balbuceé.

El hombre extraño señaló con la cabeza hacia mi silla de escritorio. ``¿Por qué no tomas asiento?''


Me aparté un poco, siguiendo sus instrucciones. —Deberíamos hablar de esto. Lo que sea que te hayan pagado los humanos… puedo darte más. El doble, incluso.

``Los humanos no tienen nada que ver con esto. Estoy aquí porque todo lo que he conocido, todas las personas que me han importado se han ido. No tienes idea de lo que es el verdadero dolor''.

``Mira, si estás deprimida, hay formas de obtener ayuda. No tienes que hacerme daño a mí ni a nadie, ¿de acuerdo?''


—Ahí es donde te equivocas. Soy el último superviviente de mi especie. Alguien tiene que pagar por ello.

Algo hizo clic en mi cabeza, aunque sonara ridículo. ¿Cómo podría alguien haber sobrevivido a una supernova?

—Eres un Devorador —susurré.

Sus músculos faciales se contrajeron. ``No me gusta esa palabra. Mi nombre es Byem''.

—Está bien, entonces, Byem, escúchame. Los humanos estaban decididos a cometer un genocidio desde el principio. No había nada que yo pudiera hacer. —Forcé una expresión comprensiva en mi rostro. Era difícil pensar con un arma apuntándome a la cabeza, pero sabía que necesitaba redirigir su ira—. El comandante Rykov es a quien buscas. Está aquí, en el edificio. Él mató a tu gente, no a mí.

—No. Necesitaba entender por qué había sucedido esto... y todo te lleva de nuevo a ti. Rykov intentó rescatarnos, pero tú saboteaste las naves furtivas. De hecho, con tu sabotaje, intentaste obligar a los humanos a matarnos —dijo, con la voz temblorosa por la ira—. Podrían haber terminado de evacuar mi planeta, si no hubieran necesitado un desvío para lidiar contigo. Si no hubieras iniciado una guerra civil, y si no hubieran tenido que reparar su nave después. En última instancia, tú eres el responsable.

—No fue así. ¡No lo entiendes! Necesitaba que cometieran un error para que todo el universo pudiera ver su verdadero rostro, en toda su fealdad. Los humanos han estado estafando a la galaxia durante siglos, mientras conspiran y construyen un arsenal para matarnos a todos. Fue un riesgo calculado para salvar a la Federación de un mal que no puedes comprender.

``¿Un riesgo calculado? ¿Acaso una extinción planetaria es un sacrificio menor para ti? Tú eres el malvado. No mereces el nitrógeno que respiras. Dile adiós a tu miserable existencia''.

Parecía que convencer a Byem no era una opción. El Devorador tenía una mirada enloquecida en su rostro; sus venas estaban a punto de estallarle por el cuello. Su dedo estaba suspendido sobre el gatillo, mientras intentaba mantener el arma firme. Quería pedir ayuda, pero me eliminaría tan pronto como alcanzara mi holopad.

Salvo un milagro, parecía que este sería el final.

La puerta se abrió y el embajador Johnson entró tranquilamente, mirando un trozo de papel. ``¡Ula, viejo amigo! Tenía algo de tiempo libre, así que redacté un currículum para ti. Échale un vistazo''.

La humana levantó la vista y se puso pálida al ver el arma. No estaba segura de si debía pedirle ayuda o convencer a Byem de que le disparara.

—Vete. Ahora. Esto no tiene nada que ver contigo —susurró.

Johnson levantó las manos en un gesto de apaciguamiento. ``He leído sobre ti en los informes de la misión. Byem, ¿verdad? Baja el arma. No quieres hacer esto''.

—Sí, lo hago. Quiero que esta alimaña muera. —Una lágrima rodó por su mejilla y se la secó rápidamente—. Intenté salvar a mi gente y fracasé. La venganza es lo único que me queda.

—No es tu culpa. Por favor, no dejes que tu dolor y tu odio te definan, Byem. Eres mejor que eso. La venganza no te ayudará a largo plazo.

``No me importa si sirve de algo. ¿Por qué debería vivir ella, cuando millones de personas no verán otro día por su culpa? ¿Cómo es eso justo?''


—No lo es. La gente como Ula es terrible. Créeme, estoy de acuerdo contigo. Pero si nos ponemos a su nivel, ellos ganan. Y yo soy demasiado mezquina para dejarles ganar.

—Ahí es donde diferimos, humano. Estoy más que feliz de dejar que Ula obtenga su victoria.

El Devorador revisó la mira una última vez, sonriéndome. Byem estaba a punto de apretar el gatillo cuando la embajadora Johnson se abalanzó sobre él y lo agarró del brazo dominante. Tiró de él hacia abajo mientras el arma se disparaba, abriendo un agujero en el suelo. El dúo cayó al suelo, luchando por controlar el arma de fuego.

Mi mente daba vueltas mientras observaba la escaramuza. ¿Por qué la embajadora intentaba ayudar? Esta era su oportunidad de deshacerse de mí, sin que el gobierno terrestre se manchara de sangre. Se suponía que los humanos disfrutaban de la carnicería de todos modos. No había ninguna razón para que ella se pusiera en la línea de fuego por un enemigo jurado.

El embajador Johnson giró la muñeca, intentando soltar la pistola. Byem extendió el brazo libre y recogió un trozo de cristal del suelo. Con un movimiento rápido, se lo clavó en el muslo.

La humana gritó de dolor y su falta de concentración permitió que el Devorador se liberara. Se soltó de su agarre y comenzó a arrastrarse. Se me ocurrió correr o unirme a la pelea, pero estaba paralizado. Algo en mi cerebro se había apagado y no podía volver a encenderlo.

Byem se puso de pie con dificultad, apoyándose en el escritorio. Apuntó con el arma a la embajadora Johnson, que se estaba curando la pierna herida. La sangre carmesí había empapado sus pantalones azul marino, tornándolos morados. No sabía mucho de anatomía humana, pero debía haber recibido un golpe en algún tipo de vaso sanguíneo.

—Quédate abajo. No quiero hacerte daño —le suplicó.

El arma giró hacia mí y me preparé para lo inevitable. No tenía sentido pedir perdón a ninguno de los dos, porque no lo sentía. Claro, mis métodos no habían sido perfectos, pero yo era el único lo suficientemente valiente para enfrentar a esos miserables humanos y hacer sacrificios por el bienestar de la Federación.

Sentí un dolor punzante en la frente. Me desplomé en el asiento y vi cómo el mundo se volvía borroso. Todo estaba tan borroso, tan desenfocado...

En lo más recóndito de mi conciencia, era vagamente consciente de que Byem huía de la escena del crimen. Mis oídos registraron las palabras de la embajadora Johnson, que pedía ayuda a través de su holopad. Pero yo estaba demasiado ido para procesar nada más que la sensación de frío que me invadía el cuerpo.

La nada se apoderó de mis sentidos y me hundí en los brazos del vacío.