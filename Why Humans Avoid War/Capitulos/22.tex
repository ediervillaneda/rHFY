\chapter{La Guerra y la Humanidad}\label{sec:la-guerra-y-la-humanidad}

En el último combate oficial en el que vi a los humanos participar, desplegaron un total de cinco naves. Eso era todo lo que necesitaban.

Esta vez, habían formado una armada de verdad, reuniendo miles de sus mejores naves. Se había enviado un mensaje a la Federación informándoles del desastre en el campo de refugiados y advirtiéndoles que se prepararan para una incursión de los Devoradores en caso de que fracasáramos. Sin embargo, los terranos optaron por no solicitar ninguna ayuda militar ni divulgar los detalles de su plan. Sin duda, el sabotaje del Portavoz todavía estaba impreso en su memoria.

La repentina movilización de la Tierra, a pesar de las razones que dieron, sin duda despertó algunas alarmas en la Federación. Las dos partes desconfiaban profundamente entre sí, a la luz de los acontecimientos recientes, y yo dudaba que esa desconfianza se disipara alguna vez. Diablos, incluso yo tenía miedo de la humanidad a veces.

Pero ahora, mientras nos dirigíamos hacia nuestro enfrentamiento final, solo me preocupaban los humanos. Especialmente Rykov. Tal vez no comprendía el lenguaje corporal humano, pero su postura me indicaba dolor. Estaba acurrucado sobre su holopantalla, con la cabeza inclinada. Sus hombros se hundían como si hubieran colocado un peso sobre ellos. ¿Estaba realmente preparado para hacer lo que era necesario?

Me acerqué a él y le di un codazo en el brazo. ``¿Estás bien? Parece que estás dudando''.

—Claro. No te preocupes, no lo dudaré. —El humano suspiró y sacudió la cabeza—. La gente de casa no entenderá lo que sacrificamos. Lo que damos de nosotros mismos. Quizá tú tampoco, general.

``¿Morir? ¿Casi morir? ¿Recibir un disparo?'', sugerí.

Rykov se rió amargamente. ``Más allá de eso, no regresas siendo la misma persona. Tienes que vivir con las cosas terribles que has hecho y simplemente intentas convencerte de que fue lo correcto''.

``Es correcto. Se salvarán billones de vidas. Creo que eso es un honor'', respondí.

``Tal vez. Hoy en día no hay buenas opciones'', dijo. ``Pero créeme, no disfruto…''

La nave se sacudió violentamente cuando salimos del hiperespacio, en los límites del sistema Devorador. Me deslicé por el suelo y gruñí al chocar con una silla de metal. No parecía una emergencia controlada. Parecía que nos habíamos topado con un campo que alteraba la curvatura; el enemigo debía haber instalado nuevas defensas mientras estábamos fuera.

Oí que alguien maldecía a unos cuantos metros de distancia y vi al comandante Rykov de rodillas. Parecía que se había chocado de cabeza contra una estación de trabajo. De sus fosas nasales manaba sangre roja y su nariz sobresalía de forma antinatural hacia un lado. Definitivamente estaba rota y, en mi opinión, eso implicaba una cantidad considerable de dolor.

Me arrastré hasta el humano. ``Necesita atención médica. Puedo supervisar…''

El comandante se levantó y apretó los dientes. —No es más que un rasguño. Soldados, quiero que las armas se enciendan para ayer. Disparad a todo lo que se mueva y no sea de los nuestros.

Lo seguí cojeando, mirándolo con incredulidad. ¿De verdad iba a seguir como si nada hubiera pasado? Los humanos eran más duros de lo que parecían.

—Señor, tenemos otro problema —gritó un alférez—. Parece que el enemigo ha desprendido una cápsula de ataque. Un grupo de abordaje se ha infiltrado en la nave.

Mis antenas temblaron confundidas. ``¿Desde cuándo abordan barcos?''

—Creo que sienten un odio especial por este. —El comandante frunció el ceño y se secó un hilo de sangre del labio—. General, nos faltan soldados de a pie. Me gustaría que se reuniera con mi equipo de seguridad en el hangar principal y se encargara del comité de bienvenida. Manténgalos con vida, si es posible.

A pesar de haber hecho varias visitas a la nave insignia, no estaba seguro de dónde estaba el hangar principal, pero pensé que podría averiguarlo. Lo último que quería era parecer incompetente, como el futuro primer oficial de la nave. Le hice un gesto a Rykov y saqué un rifle de plasma del carro de armas. El arma era mucho más ligera de lo que recordaba, lo que atribuí a las mejoras de nanitos.

Salí del puente rápidamente y esperaba que siguiendo mis instintos llegara a mi destino. Pero después de doblar algunas esquinas, me di cuenta de que iba en la dirección equivocada. Los armarios llenos de frascos de pastillas y jeringas sugerían que se trataba de la enfermería. No había ningún médico a la vista; los no combatientes probablemente estaban encerrados hasta que se asegurara la nave.

Inspeccioné las paredes en busca de un mapa. Tenía que haber algún tipo de indicaciones, en algún lugar, al menos para los procedimientos de evacuación. Mis oídos se pusieron alerta cuando se escucharon pasos en el pasillo y suspiré aliviado. Tal vez estos humanos podrían guiarme por el camino correcto.

Estaba a punto de gritar, pero entonces me di cuenta de que algo no iba bien. No eran los fuertes golpes de las botas terrestres, sino un sonido de golpeteo, como gotas de lluvia que caen sobre un tejado. Me metí en el consultorio de un médico y miré por detrás de una pared. Había cinco soldados Devoradores, según mis cálculos, escabulléndose por los pasillos. Parecían estar revisando el lugar en busca de rezagados.

Apreté mi arma contra mi pecho, tratando de estabilizar mi respiración. Mi única esperanza era tenderles una emboscada y eliminar a varios antes de que supieran qué los había golpeado. Las sombras se extendían más allá del marco de la puerta, lo que sugería que pasarían por la oficina en cuestión de segundos.

Mi dedo encontró el gatillo en cuanto cruzaron mi campo de visión. Una bala de plasma atravesó la frente del soldado líder y se desplomó en el suelo. Impulsado por el instinto de supervivencia, cargué contra los otros cuatro en un frenesí animal. Antes de que pudieran devolver el fuego, había derribado a otro tipo y lo había puesto encima de mí para que sirviera de escudo.

Sus amigos abrieron fuego y sentí que su cuerpo se contraía mientras lo acribillaban a balazos. Hubo una pausa mientras sus armas se enfriaban, y ese fue todo el tiempo que necesité. Saliendo de debajo del cadáver, maté a uno de ellos con un tiro en el pecho y luego apunté con mi rifle al segundo. Una descarga le atravesó el cuello y se desplomó en un montón de sangre.

Cuatro abajo, falta uno. En teoría, el último tipo debería haber sido el más fácil de eliminar, pero en la experiencia parecía suceder lo contrario. Este me estaba observando con ojos agudos y se agachó justo cuando encontré mi objetivo. Mi tiro pasó zumbando sobre su cabeza y se enterró sin hacer daño en la pared. Para empeorar las cosas, mi rifle vibró en mis manos, lo que indicaba que estaba en un tiempo de recuperación de cinco segundos. Bueno, mierda.

En ese momento, su arma estaba casi recargada, lo que significaba que tenía que actuar de inmediato. Acorté la distancia entre nosotros a grandes zancadas y lo golpeé en el estómago con la culata de mi rifle. Con una exhalación brusca, se tambaleó hacia atrás y dejó caer su arma al suelo.

Extendí la mano para coger el arma suelta, pero mi acción no pasó desapercibida. La pateó y se dirigió hacia mí. El pánico me impulsó a blandirle el rifle como si fuera un garrote. Golpearlo en la cabeza no parecía la peor idea; bueno, al menos hasta que lo intenté. Sus manos se levantaron como un borrón y atraparon el arma por el cañón. Antes de que me diera cuenta, el arma había desaparecido de mi alcance.

Cuando cayó al suelo, me di cuenta de que estaba en problemas. El combate cuerpo a cuerpo nunca fue mi fuerte, ya que no era algo en lo que se entrenara a los militares Jatari. Con los reflejos divinos de este tipo, sin duda tenía la ventaja.

Apenas levanté los puños a tiempo para bloquear una serie de golpes. Mientras intentaba parar sus golpes, me hizo un barrido con los pies. Sentí un dolor que me recorrió la columna vertebral cuando choqué contra el frío metal. Se desplomó a mi altura en un instante y me hizo una llave de cabeza.

Se retorcía y se arañaba la cara; nada de eso parecía servir de nada. Su agarre alrededor de mis vías respiratorias se hizo más fuerte y pude sentir que mi conciencia se debilitaba. Una sensación de ardor latía a través de mis pulmones, mientras mi cuerpo gritaba por oxígeno. Cada vez me resultaba más difícil formar pensamientos coherentes. Un velo de oscuridad se estaba infiltrando en mi percepción; en unos momentos, me deslizaría hacia su abrazo.

Se escuchó un crujido espantoso, que apenas registré. El agarre de mi agresor se aflojó y me solté, jadeando en busca de aire. Un dolor persistente en mi garganta donde había estado su brazo, y supuse que tendría algunos moretones que mostrar de nuestro encuentro. Pero sin intervención... podría haber sido mucho peor.

—¿En qué estabas pensando? ¿En enfrentarte a un grupo de ellos tú solo? —Miré hacia arriba y vi a Mac, el oficial de seguridad de Rykov, acompañado por un equipo de otras doce personas—. No apareciste y ahora entiendo por qué.

—Me perdí —balbuceé—. Este barco… es un maldito laberinto.

Mac hizo una pausa. —Ya veo. Bueno, le di a tu amigo un buen golpe en la cabeza. Cuando se despierte, estará en el calabozo. Ven conmigo, si puedes caminar.

—Uh, gracias. Te debo una. —Me puse de pie con dificultad y seguí al hombre corpulento—. ¿Adónde vamos?

—El puente. Supongo que el jefe quiere que vuelvas.

``¿Qué? Seguro que hay más, tenemos que…''

—Había. Tiempo pasado —gruñó—. Sólo... Ah, mira el paisaje mientras caminamos.

Una serie de preguntas flotaban en mi mente, que pensé que era mejor dejar sin respuesta. ¿Estaba diciendo que habían despejado toda la nave? No podían haber pasado más de 15 minutos desde que partí siguiendo las órdenes de Rykov.

El significado de su declaración se hizo evidente cuando entramos en el pasillo principal del puente. Había cuerpos esparcidos por el pasillo, con sangre y materia cerebral salpicada en las paredes. Uno de los cadáveres tenía un brazo amputado metido en la garganta, mientras que otro estaba cortado por la mitad. Ni siquiera quería saberlo... la brutalidad era repugnante.

—¿Paisaje? —Miré a mi alrededor, estupefacto—. ¡¿Qué carajo, Mac?! Estos tipos son esclavos . ¿Tuviste que descuartizarlos? Rykov dijo vivos y tú...

—Tranquilos. No son esclavos. ¿Veis las marcas que llevan en el cuello? —El humano bajó la voz—. Vuestro amigo Byem nos advirtió sobre ellos. Forman parte de un culto apocalíptico que adora a la IA y quiere ayudarla a provocar el fin de los días. Siempre están en primera línea cuando destruyen un mundo.

Fruncí el ceño. ``Aun así, ¿no crees que esto es… un poco excesivo?''

``En absoluto. De hecho, me gustaría que sufrieran más''.

Me quedé en silencio. Obtener placer del dolor de otro ser parecía cruel, pero Mac hablaba como si fuera lo más normal del mundo. Era en los atisbos de sus peores impulsos que la humanidad me aterrorizaba; siempre estaban a un paso de convertirse en los monstruos que despreciaban.