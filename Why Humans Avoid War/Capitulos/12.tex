\chapter{La Fortaleza de la Capital}\label{sec:la-fortaleza-de-la-capital}

El sistema de defensa planetaria del mundo capital de la Federación estaba diseñado para protegerse de un bombardeo orbital y consistía en armamento de última generación. Este era el planeta más fuertemente protegido de la galaxia, dada su importancia política y simbólica. Con toda su potencia de fuego dirigida contra una sola nave, no había forma de que los escudos convencionales pudieran resistir la explosión. Temía que los humanos hubieran mordido más de lo que podían masticar.

¿Cómo respondería el gobierno terrano a la destrucción de la joya de la corona de su flota? No estaba claro si se apegarían al concepto de una respuesta proporcional. Una declaración de guerra abierta podría ser inminente, especialmente si la Federación disparaba los primeros tiros.

Desde mi punto de vista, parecía de sentido común que provocar a los humanos no era lo mejor para nosotros. La Tierra había firmado tratados que prohibían los ataques a civiles y, en muchos casos, los había fundado. Pero, si se los empujaba al límite, ¿quién sabía de qué eran capaces? Con solo una bomba de nanocitos arrojada contra una población metropolitana, las víctimas se contarían por millones.

No es que yo estuviera aquí para preocuparme por las consecuencias. En unos momentos, me evaporaría, junto con todos los demás ocupantes de esta nave.

Al mirar por la ventana de la nave insignia, vi un resplandor azul que se extendía por la superficie lunar. Eso indicaba que el láser orbital se estaba cargando; era capaz de emitir la misma cantidad de energía que una estrella de tamaño mediano, al menos durante unos segundos. Un impacto tan poderoso atravesaría nuestros escudos como si no existieran.

Cualquier esperanza que tenía de escapar con vida de la situación se esfumó. Pensé que dispararían los cañones de plasma de la estación lunar, o sus misiles guiados, como era el protocolo habitual para una intrusión espacial. El láser orbital era la última línea de defensa de la capital, lo que parecía demasiado para una sola nave. ¿Había alguna manera de convencer al comandante Rykov de que diera la vuelta?

—¡Esto es un suicidio! Debemos retirarnos o todos moriremos. —Odiaba el tono de desesperación en mi voz, que se hacía cada vez más agudo a medida que hablaba—. Puedes hablar con la Federación más tarde y resolver algo...

El Comandante se enderezó, con un destello de amargura en sus ojos. ``El tiempo de hablar ha terminado. Hemos intentado hablar durante cientos de años, y mira qué bien nos ha ido. La Federación necesita aprender una lección de humildad''.

—Mira, estoy de acuerdo en que esto es un acto de guerra. Si yo estuviera en tu lugar, también respondería de la misma manera. Pero necesitas más naves y un plan sólido. Nuestras muertes no lograrán nada —supliqué.

Hizo un gesto con la mano con desdén. —No pienso morir hoy, general. Estaremos bien.

Se me ocurrió que Rykov o bien no había comprendido la gravedad de la amenaza, o bien sus recientes escaramuzas le habían hecho confiar demasiado. Fueran cuales fuesen las fortificaciones que tuviese el buque insignia, no había forma de que estuviesen diseñadas para soportar fuerzas tan extremas.

``Atención. Se ha detectado un objetivo fijado contra esta nave. Tiempo estimado de impacto: cinco segundos'', dijo una voz computarizada.

Cerré los ojos con fuerza, esperando que la oscuridad permanente me invadiera. El sonido de las alarmas sonó en mis oídos y me pregunté si sería lo último que oiría. No había miedo en mi mente, solo un odio ardiente hacia los tontos que dirigían el gobierno federal. Esta pérdida de vidas podría haberse evitado, si tan solo el Portavoz se hubiera comportado con sensatez.

Cinco, cuatro, tres, dos, uno…

Abrí los ojos de golpe cuando una sacudida atravesó la nave y casi perdí el equilibrio. Las luces parpadearon en lo alto, probablemente porque la energía se redirigió a los escudos, pero ese fue el único efecto secundario de la explosión que noté. No se produjeron incendios en el puente ni se desconectó ningún sistema.

``Escudos al 96 \%. Integridad estructural intacta. Se recomienda tomar represalias'', entonó la computadora.

No podía entender cómo la nave insignia seguía en una sola pieza. Ese láser orbital estaba diseñado para dominar a una formación entera, pero apenas había hecho daño a la nave terrestre. ¡Lo único que había logrado era reducir ligeramente la capacidad de su escudo!

La mansedumbre de la humanidad y el motivo por el que habían ocultado su verdadera naturaleza durante tanto tiempo resultaban más desconcertantes que nunca. Sus naves eran casi invencibles y su tecnología superaba en siglos al armamento de la Federación. ¿Qué era lo que diferenciaba a los humanos de otras especies agresivas? Podían gobernar la galaxia si así lo deseaban, pero en lugar de eso moralizaban y mediaban.

El comandante Rykov se aclaró la garganta. —¿Esa era su arma más poderosa, general? Ustedes nos necesitan más de lo que creen.

—Supongo que ya nada debería sorprenderme —refunfuñé—. ¿Y ahora qué? Volverán a disparar una vez que se recargue el láser.

El humano inclinó la cabeza, como si sopesara sus opciones. —¿Esa estación se opera de forma remota? Un escaneo de sensores de la luna no detectó rastros de vida.

``Sí, que yo sepa'', respondí.

—Bien. En ese caso, nos aseguraremos de que no tengan la oportunidad de volver a disparar. —Juntó las manos y esbozó una sonrisa depredadora—. Alférez Carter, prepare los misiles antimateria. Quiero que entierre esa estación.

``Entendido, señor'', fue la respuesta desde el puesto de armas.

Me sentí extrañamente distante al ver cómo un trío de ojivas acortaba la distancia entre nosotros y la estación. Mi juramento como soldado era proteger y defender a la Federación contra todos los enemigos, y parecía que los terranos ahora podían ser clasificados como un grupo hostil. Después de todo, sin el sistema de defensa planetaria, la capital quedaría vulnerable a los ataques. Al no tomar ninguna medida durante este enfrentamiento, me estaba poniendo del lado de los humanos, ¿no?

Los misiles se estrellaron contra la superficie lunar con un destello radiante, levantando columnas de polvo y escombros. Donde antes había un complejo de defensa sofisticado, solo quedaron tres cráteres. Las consecuencias recordaron más al impacto de un asteroide que a un misil, dada la magnitud de su profundidad. Al contemplar el corte reciente en el suelo pedregoso, me pregunté una vez más por qué la Federación estaba empeñada en enfadar a los humanos. A las criaturas con tal maestría en la destrucción se las debería apaciguar, no enfadar, a menos que el objetivo fuera la erradicación de la civilización.

El zumbido bajo de la maquinaria sonó detrás de mí y me estremecí por instinto. Me di la vuelta y vi un carro robótico cargado de armas de fuego que cruzaba el puente y al personal terrestre preparándose a su paso. El comandante Rykov sacó un rifle de plasma con mira telescópica y, sin decir palabra, me lo entregó. Su peso era mucho mayor de lo que esperaba y mis hombros se hundieron cuando lo acepté. O bien los humanos lucían algún tipo de armadura de poder o su fuerza física superaba con creces la de mi especie.

—General... creo que tendremos que abrirnos paso hasta la embajada una vez que atraquemos. Entre los manifestantes y las fuerzas de seguridad de la Federación, nos superarán en número —dijo el comandante—. Usted conoce la distribución de la capital mejor que nosotros. ¿Alguna sugerencia?

Hice una pausa y me di cuenta de algunas ideas. —Bueno, una distracción sería de ayuda. ¿Tienen armas químicas en el buque insignia? Supongo que los manifestantes están apiñados fuera de la embajada, así que podrían atacarlos con algún tipo de gas. Cuando los aerodeslizadores de emergencia lleguen a la escena, podrán usarlos como escudo contra las fuerzas especiales.

Esperaba que el humano apreciara la ingeniosidad de mi plan, pero en lugar de eso, me miraba como si me hubiera crecido un cuarto ojo. Su boca se abrió y se cerró varias veces, como si estuviera luchando por encontrar las palabras.

—No vamos a hacer eso —respondió finalmente—. Por favor, olvídate de que te lo he pedido.

Cualquiera que fuera el problema que tenía con mi sugerencia, no lo entendía. No sólo minimizaría las bajas humanas, sino que también proporcionaría cobertura en un paisaje urbano denso. Una avenida amplia no era exactamente ideal para el combate terrestre y las maniobras.

La nave insignia inició un rápido descenso a través de la atmósfera del planeta, pasando a toda velocidad entre nubes plateadas. La pantalla de la computadora indicó que había fijado la señal de aterrizaje y me preparé para enfrentar lo que me esperaba.

A medida que nos acercábamos a nuestro destino, finalmente vislumbré el suelo en la pantalla. Sabía que probablemente tendríamos que luchar para salir del puerto espacial antes de poder dirigirnos a la embajada. No me sorprendió, entonces, cuando vi al contingente de soldados entrando en el hangar con las armas preparadas.

Sin embargo, no esperaba que fueran cientos y que estuvieran compuestos únicamente por militares de Xanik. No se trataba de una fuerza de seguridad común y corriente, y eso podría hacer que nuestra misión fuera más difícil de lo esperado.

Solo esperaba que el comandante Rykov tuviera un plan, porque contra una unidad de ese tamaño, no tenía la menor idea de cómo escapar con vida del puerto espacial.