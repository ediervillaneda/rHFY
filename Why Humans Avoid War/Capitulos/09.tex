\chapter{Una Traición Inesperada}

La elegante silueta de la nave insignia terrestre se alzaba en la pantalla. Nuestros sensores no detectaron ningún daño ni desviación del funcionamiento normal, ni siquiera a corta distancia. Pero nuestras llamadas a las naves terrestres, solicitando actualizaciones de estado, habían quedado sin respuesta durante horas.

El silencio de radio de su flota no era habitual en ellos y yo había empezado a temer lo peor. Se envió una lanzadera para restablecer el contacto con los humanos y prestar ayuda si era necesario. Opté por acompañar al equipo, a pesar del posible riesgo para mi seguridad. Era poco común que un oficial de alto rango como yo participara en una misión de rescate, pero sentía que les debía mi presencia. Supuse que si los papeles se invertían, el comandante Rykov no estaría observando desde la barrera; estaría ayudando en todo lo que pudiera.

¿Qué pudo haber causado que los terrícolas interrumpieran todas las comunicaciones sin ninguna explicación? No era totalmente inesperado que algo hubiera salido mal con el plan, por supuesto. La logística de evacuar a miles de civiles en minutos era poco práctica, si no imposible.

Recordé mi breve misión en la colonia lunar de Jatari, cuando un asteroide que se acercaba obligó a una evacuación obligatoria. A pesar de las advertencias del gobierno, muchas personas se mostraron reacias a abandonar sus hogares. Los que se marcharon en la avalancha inicial se concentraron en un único puerto espacial, lo que provocó congestiones y retrasos. Habían tardado días en desalojar a todos los habitantes, y nos apresuramos a buscar a los rezagados hasta el último momento.

Según mi experiencia, la única esperanza de completar la misión en el tiempo asignado era la intervención divina. Sin embargo, el comandante Rykov parecía tan confiado que había desestimado mis objeciones como si fueran triviales. ¿Cómo podía presionarlo más, cuando los humanos habían cumplido una y otra vez sus promesas imposibles?

Si hubiera hecho caso a mi instinto, tal vez se hubiera podido evitar la situación actual. Ahora nos encaminábamos hacia un posible peligro, sin la menor idea de lo que podíamos estar afrontando.

La piloto, una joven llamada Daari, se aclaró la garganta. ``Señor, nuestros sensores detectaron dos lecturas que coinciden con las de las balas de plasma. Se acercan rápidamente''.

Mis antenas se movieron de la sorpresa. ¿Quién nos estaba disparando? Las únicas naves que había en las cercanías eran las de los terranos, pero nunca habían mostrado ninguna inclinación a atacar. Tenía que haber otra explicación.

``¡Realicen maniobras evasivas!'', grité.

—No hay tiempo. —Apretó unos cuantos botones, probablemente desviando toda la energía a los escudos—. Prepárense para el impacto.

En el mejor de los casos, el transbordador sufriría graves daños; este pequeño cacharro no estaba diseñado para recibir impactos directos de un cañón de riel. En el peor de los casos... bueno, todos estaríamos muertos.

Los siguientes momentos se prolongaron durante una eternidad y, a medida que mi ansiedad crecía, me esforcé por mantener la compostura. Mis instintos me gritaban que hiciera algo, aunque fuera inútil. No había nada peor que esperar, sin poder evitar la muerte.

``Señor… las balas no nos alcanzaron por poco'', dijo Daari. ``Si el ángulo hubiera sido más bajo, habrían golpeado los escudos''.

Sentí un alivio que me recorrió las venas y luego confusión. ``¿Qué? No me quejo, pero no deberíamos haber sido un objetivo difícil. ¿De dónde vinieron los disparos?''

—La nave insignia terrestre —respondió ella.

—No, no, eso es imposible. —Las únicas naves en la zona eran humanas, pero aún no podía creer que nos dispararan—. Deben haber sido hackeadas por la IA. O tal vez fueron secuestradas. Tenemos que ayudarlas de inmediato.

Daari se movió, incómoda. —Con todo respeto, dudo que ese sea el caso. No hubo señal de socorro, ni señales de una brecha, ni cambios en las funciones de la computadora.

``Yo…yo no entiendo.''

—Yo tampoco sé por qué, señor. Pero quizá el presidente tenía razón.

``¿Tienes razón en qué?''

``Una especie tan agresiva atacará sólo por el gusto de hacerlo. Nunca debimos haber confiado en los humanos''.

Mi mente daba vueltas. Las pruebas apuntaban a que los terrícolas actuaban por voluntad propia. Daari tenía razón. Nunca me sumaría a las ideas de la Portavoz sobre la agresión, dado que las había dirigido contra mi propia especie en el pasado. Pero sus advertencias de que los humanos se volverían contra nosotros eran proféticas a la luz de la situación actual.

¿Quizás me estaban usando solo para obtener acceso a las naves furtivas todo el tiempo? Una vez que cumplí ese propósito, ya no tenían motivos para fingir ser aliados.

Si bien esa explicación tenía sentido dadas las circunstancias, no me sonaba a verdad. Eran personas que ayer mismo consideraba mis amigos, a quienes les habría confiado mi vida. No sé si era solo por terquedad, pero aun así me encontré buscando otra respuesta.

—¿Cómo responderemos, señor? —Daari rompió el silencio—. Nos dispararon, lo que es una declaración de guerra. Según las reglas de combate, estamos autorizados...

``No fallan'', afirmé.

Parecía desconcertada por mi comentario. ``Está claro que fallaron, señor. No por mucho. Para todo hay una primera vez''.

—No creo que hayan tenido la intención de atacarnos —entrecerré los ojos—. Si nos hubieran querido muertos, ya estaríamos muertos. Saludad a las naves terrestres de nuevo.

-Pero señor, no creo que...

—Tomo nota de tu objeción, Daari. Ahora cumple mis órdenes de inmediato.

Si los humanos no respondían esta vez, no estaba seguro de qué hacer. Era obvio que algo había cambiado durante el transcurso de su misión. Necesitaba saber qué había sucedido antes de que termináramos en un combate aéreo con la principal potencia militar de la galaxia.

El comandante Rykov apareció en la pantalla. Tenía los brazos cruzados y los ojos entrecerrados. A juzgar por su expresión, si nos hubiésemos reunido en persona, habría intentado golpearme en la cara. No tenía idea de qué había hecho para merecer tal hostilidad.

Intenté tranquilizar al humano con una sonrisa amistosa. ``Debe haber algún tipo de malentendido, porque estoy bastante seguro de que acabas de dispararnos''.

Su ceño se profundizó. —Ese fue un disparo de advertencia. El próximo se va a enterrar en tu casco, al diablo con las órdenes. Vete, ahora.

—Solo vinimos a ayudar —protesté—. Estábamos preocupados por ti.

—¿En serio? ¿Por eso saboteaste las naves? —se burló Rykov.

``¡Eso es ridículo! No hice tal cosa''.

``Supongo que pensaste que no recuperaríamos las naves después de que las derribaran. Es claramente obvio, a partir de los registros de la computadora, que hubo una anulación remota de su protocolo de sigilo. Por alguien con autorización de Nivel 9, que solo posee el general de más alto rango de la Federación''.

Me quedé atónito por lo que estaba escuchando. ¿Los humanos creían que yo, su más ferviente partidario, había inutilizado las naves Vortex? Jamás haría algo así, pero si tenían pruebas para respaldar esa conclusión, no estaba seguro de cómo persuadirlos. Esto tenía que ser algún tipo de trampa.

—Escucha, no fui yo. Podemos resolver esto juntos —le supliqué.

El comandante negó con la cabeza. —No te lo pienses. Tres de mis hombres murieron por tu culpa. Deberíamos matarte, pero que derribemos una nave de la Federación es justo lo que Ula necesita para echarnos.

Se me escapó un jadeo cuando me di cuenta: ``Ula… ella también tiene autorización de nivel 9''.

—Espera, ¿lo hace? Así es, el Portavoz también es tu Comandante en Jefe —murmuró—. Ella tiene muchos más motivos que tú. No pude entender por qué lo hiciste.

—Sé que ella odia a los de tu especie, pero no puedo creer que haya llegado a tal punto.

``Es una fanática. Conozco a ese tipo de personas. Cree que está haciendo lo correcto y eso la hace peligrosa''.

Era evidente que la Portavoz había querido enfrentar a la flota y a los humanos entre sí, y, por muy aterrador que fuera, casi funcionó. Sus acciones pusieron en peligro no solo la seguridad de los terranos, sino también la de sus propias fuerzas y la de los civiles Devoradores. La ira ardía en mi interior ante la idea de enfrentarme a ella. No estaba seguro de lo que haríamos, pero hacerla responsable era ahora mi máxima prioridad.

``Ella es peligrosa para todos nosotros mientras esté al frente de la Federación. Por favor, a menos que todavía quieras que me vaya, déjanos ayudar. Ula tiene que pagar''.

``Aceptaríamos gentilmente su ayuda y le daríamos la bienvenida a bordo del buque insignia''.

Nuestras diferencias con los humanos se habían suavizado, lo cual era un alivio. Pero no podía decir que la facilidad con la que Rykov me acusaba de traición no me dolía, especialmente después de las recientes pruebas que habíamos enfrentado juntos. Claramente, la confianza que deposité en él no era un sentimiento mutuo.

``Atracamos de inmediato'', respondí. "Pero primero... ¿realmente creíste que te sabotearía? ¿Haría que mataran a tu gente?''

El comandante sonrió tristemente. "Si algo he aprendido en mi etapa como comandante es que nunca se conoce realmente a nadie. Pero por si sirve de algo, lo siento, general. Nunca debí haberlo acusado''.

Mi amargura se disipó cuando vi el brillo húmedo en sus ojos. Una punzada de compasión atravesó mi corazón cuando me di cuenta de que el pobre hombre no confiaba en nadie. Esperar su confianza después de conocernos durante unos días era quizás demasiado pedir.

``Disculpa aceptada''.