\chapter{El Último Asalto}\label{sec:el-ultimo-asalto}

Me apresuré a regresar a mi puesto junto a la pantalla holográfica, ansioso por alejarme lo más posible de los cuerpos destrozados.

A través de la lente del visor, vislumbré la batalla que se libraba entre las estrellas. Mientras yo estaba preocupado por los soldados rasos que se infiltraban en nuestra nave, los compañeros de tripulación en el puente se habían estado abriendo paso a través de un mar de enemigos. Unos cientos de naves aliadas estaban escondidas a nuestro lado, tratando de resistir una andanada de fuego de plasma.

Para su crédito, los terranos estaban lanzando una cantidad igual de munición a los hostiles. Los humanos esperaban encontrar cualquier debilidad que explotar, cualquier grieta en su armadura, pero fue en vano. Los escudos de los Devoradores absorbían todo lo que les lanzábamos. Nuestro avance se había ralentizado y estábamos a punto de ser empujados hacia atrás.

—General —el comandante Rykov asintió con la cabeza, con la voz tensa por el dolor—. Me alegro de verte en una sola pieza.

—Sí. Otros no compartieron mi buena suerte —murmuré.

El humano entrecerró los ojos. ``¿Qué significa eso? ¿Sufrimos bajas?''


Aparté la mirada, incapaz de mirarlo directamente. —No que yo sepa. Quiero decir que tus muchachos descuartizaron a los alienígenas. Si no lo supiera, pensaría que un animal salvaje los atacó.

—Ah, ya veo —suspiró Rykov, mientras se masajeaba la frente—. Las emociones humanas pueden ser muy intensas. Mis órdenes fueron claras... pero supongo que se dejaron llevar un poco. Lamento que hayas tenido que ver eso.

Me incliné sobre la consola de mando, sin nada más que decir al respecto. Lo importante ahora era completar la misión y, para ello, necesitábamos encontrar una forma de atravesar la sólida línea defensiva del enemigo. Pero eso parecía casi imposible ahora que conocían el manual de estrategias de los terranos.

Cada vez había más naves humanas que desaparecían del radar. Teníamos que tomar medidas drásticas antes de perder una flota entera. ¿Qué podíamos hacer que fuera completamente impredecible y que pillara desprevenidos a los Devoradores?

Una especie de misil antimateria impactó contra nuestro casco, provocando ondas en el armazón de la nave insignia. Las luces del techo se apagaron y fueron reemplazadas por un tenue resplandor naranja que emanaba de las tablas del suelo. Los sistemas informáticos se desconectaron por un momento antes de volver a funcionar. Algunos de los humanos miraron a su alrededor con nerviosismo.

``Fuente de energía primaria inoperante. Generador de emergencia activado'', dijo una voz mecánica. ``Escudos al 12%''.

Sacudí la cabeza con incredulidad. ``¿Esta es la misma nave que atravesó las defensas orbitales de la Federación sin sufrir un rasguño?''

—No es el primer golpe que recibimos. Es sólo la gota que colmó el vaso —gruñó Rykov.

No entendí el lenguaje humano, pero creí entender lo esencial. ``Entonces, nos estamos quedando sin tiempo. No parece que podamos aguantar muchos más golpes. ¿Qué hacemos ahora?''

—La Tierra depende de nosotros, general. Seguimos adelante. Lucharemos hasta el final. No hay nada más que hacer —respondió.

``Eso no será suficiente. Están un paso por delante de nosotros en todo momento''.

—Bueno, ¿qué propones exactamente? ¿Que invoquemos a Dios para que los castigue?

``Nos retiramos.''


—Tienes que estar bromeando, general. Prefiero morir que vivir como un cobarde. No voy a dejar que mi planeta se queme...

—No soy ningún cobarde, lo sabes. Romper su formación es imposible. Tenemos que atraerlos hacia nosotros. Cuando nos persigan, los flanquearemos.

Rykov frunció los labios mientras pensaba en mi propuesta. A pesar de la convicción en mi voz, no estaba convencido de que mi estratagema fuera efectiva. Pero con el conteo de naves humanas disminuyendo, nuestros escudos disminuyendo y nuestra trayectoria de avance estancada, era una apuesta necesaria. Solo esperaba que el Comandante llegara a la misma conclusión.

—Está bien, entonces retrocedamos. ¿Qué te hace pensar que nos seguirán? —preguntó después de un largo silencio.

Intenté imitar un encogimiento de hombros humano. ``Es solo una corazonada. Son arrogantes y, por lo que he visto, no les gusta dejar sobrevivientes''.

—Una corazonada, dices. —Frunció el ceño y echó una última mirada a la pantalla holográfica—. Muy bien. Enviaré un mensaje cifrado a la flota. Por el bien de todos, espero que funcione.

El comandante dio una serie de órdenes y puso en marcha el nuevo plan. La nave insignia se inclinó bruscamente y cambió de rumbo en cuestión de segundos. Aceleramos a toda velocidad; cuanto antes saliéramos del alcance del plasma, mejor. Nuestros aliados nos siguieron de cerca y los Devoradores se quedaron sin ninguna nave a la que enfrentarse.

El enemigo permaneció inmóvil por un momento. Sin duda estaban desconcertados por la repentina partida de los humanos. Una retirada coordinada, mientras la lucha aún estaba en juego, debía levantar sospechas, después de todo. Pero la oportunidad de acabar con nosotros, antes de que pudiéramos reagruparnos o saltar a la curvatura, debió haber sido demasiado tentadora. Nuestros oponentes se lanzaron tras las naves que se retiraban, disparando sus armas.

Di un suspiro de alivio, aunque ese era solo el primer paso del plan. La siguiente parte era la más difícil y sacaría a los humanos de su zona de confort: actuar como presas. Los Devoradores necesitaban oler la debilidad. Los terranos redujeron su ritmo vertiginoso, permitiendo que los perseguidores acortaran la distancia. Una avalancha de misiles precedió a la llegada del enemigo, todos con destino a nuestros acorazados más grandes. Conté al menos veinte ojivas que se dirigían hacia el buque insignia y esperé que pudiéramos capear la tormenta una vez más.

El primer impacto fue casi imperceptible y parecía que los daños no serían tan graves. Luego, varios misiles cayeron en rápida sucesión y se desató el infierno.

En su estado debilitado, nuestros escudos se esforzaron por mantener el ritmo de tal fuerza concentrada. El suelo tembló bajo mis pies, con tanta fuerza que pude sentir cómo me castañeteaban los dientes. Un terrible crujido se escuchó desde el techo, como si se hubiera reventado una tubería. Las chispas saltaron por los cables de las paredes, dejando un olor acre en el aire. Supuse que los sistemas de refrigeración se habían llevado la peor parte.

``Se ha informado de un incendio en la sala de máquinas. Los escudos funcionan a un nivel insignificante''. La voz de la computadora no transmitía emoción, como siempre. ``Se recomienda la evacuación de todo el personal''.

—No va a pasar. —Rykov juntó las manos y sonrió como un loco. Su nave estaba literalmente en llamas y él estaba… ¡emocionado! —Ahora nos toca a nosotros, muchachos. ¡Enciéndanlos!

Los Devoradores se dirigieron rápidamente hacia nuestra posición; claramente, habían dejado de lado toda precaución. Las naves terrestres se separaron a su alrededor cuando llegaron, lo que les permitió acceder al corazón de la formación. Los estábamos tragando como un bocado sabroso y no creo que se dieran cuenta. Bueno, no hasta que ya estaban rodeados por todos lados.

Después de acorralar al enemigo, los humanos los atacaron con medidas precisas e implacables. De manera similar a cómo se habían derrumbado los escudos de la nave insignia, las defensas de los Devoradores simplemente no estaban diseñadas para un bombardeo de 360 grados. Su lógica IA maestra nunca habría planeado tal escenario, porque no había ninguna razón válida para que la flota quedara acorralada de esa manera.

Los humanos ablandaron primero los escudos del enemigo, acribillándolos con fuego láser constante desde todos los lados. A continuación, desplegaron misiles tan pronto como sus defensas comenzaron a debilitarse. Los primates parecían tener un suministro infinito de explosivos que acabarían con el mundo, como de costumbre.

Tanto los enormes acorazados como los ágiles cazas desaparecieron de repente, sin poder competir con el poder combinado de las armas de fisión y antimateria. En sus intentos desesperados por escapar, algunas de las naves Devoradoras chocaron entre sí. Esas naves también se convirtieron en poco más que mil fragmentos esparcidos por la oscuridad de la noche.

—Ponga rumbo a su estrella a toda velocidad —gritó Rykov—. No hay tiempo que perder.

Una alférez se aclaró la garganta nerviosamente. ``Señor, nuestro motor se está sobrecalentando…''


El comandante se cruzó de brazos y frunció el ceño. ``¿Tartamudeé?''


La nave insignia atravesó los restos de la flota y finalmente encontró una ruta hacia el sistema Devorador. Solo esperaba que pudiéramos completar nuestra misión antes de que su ofensiva llegara a la Tierra; esta era una carrera que no podíamos permitirnos perder. El destino de la humanidad estaba en juego.

Unas cuantas naves enemigas se habían quedado atrás mientras sus compañeras caían en nuestra trampa, aunque estas naves eran la minoría. Un grupo de cinco acorazados bloqueaba nuestro camino, ya que eso era todo lo que podían hacer. No hubo nuevas órdenes en el puente, ni siquiera un reconocimiento de su presencia. El buque insignia siguió avanzando a toda velocidad.

Nuestro armazón expuesto fue quemado con fuego de plasma, que quemó algunos agujeros en nuestra piel metálica. A medida que la atmósfera comenzó a ventilarse, los procesos automatizados sellaron los sectores dañados de la nave. Iban a hacer falta más que unas cuantas marcas de perforación para acabar con esta monstruosidad terrestre.

Hice una mueca al darme cuenta de que no estábamos reduciendo la velocidad en absoluto y que estábamos a unos segundos de embestir al quinteto defensivo. Las naves Devoradoras eran vulnerables a las tácticas de embestida, claro, pero ¿realmente no había otras opciones? Si los humanos querían desperdiciar un billón de créditos, podrían haberlo hecho sin destrozar su mejor nave.

¡Diablos! Olvídense del costo económico. Con la situación de fuego en la sala de máquinas, una sacudida podría ser suficiente para convertirnos en un desastre humeante. Incluso en circunstancias normales, no había garantía de que saliéramos de allí en una sola pieza.

Sabía que era inútil razonar con los humanos, ya que los roces con la muerte parecían ser un incentivo para ellos. Todo lo que podía hacer era encontrar algo a lo que aferrarme y enviar una oración silenciosa al universo. Incluso si sobrevivíamos, este sería un viaje lleno de baches.