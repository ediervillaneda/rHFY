\chapter{Pesadillas y Redenciones}\label{sec:pesadillas-y-redenciones}

A veces me preguntaba por qué nos necesitaba. De todos modos, las máquinas se desempeñaban mucho mejor en el trabajo manual. Mi mejor suposición era que había un límite en la cantidad de tareas en las que podían concentrarse a la vez.

Una vez más, me encontré odiando al Maestro, detestando mi existencia. Una niebla de agotamiento me oprimía como siempre; habían pasado días desde la última vez que dormí. Una parte de mí quería acurrucarse en el suelo y dejar que todo se desvaneciera en la oscuridad. Tener paz por fin.

Respiraba entrecortadamente y el sudor me perlaba la frente. El aislamiento del traje de vacío me impedía entrar en contacto con el frío gélido del planeta helado, pero también me impedía gastar calor.

Miré a mi hijo, Kel, que me ayudaba a empujar un cubo de mineral de hierro hacia el elevador del túnel minero. Podía luchar contra el dolor y el cansancio por él. El pobre muchacho, nacido en un mundo implacable, no conocería nada más que servidumbre. Necesitaba a su padre cerca, aunque solo fuera para sentir un poco de amor y calidez.

Una extraña sensación de déjà vu se apoderó de mi mente un momento antes de que ocurriera. Sin previo aviso, el suelo tembló bajo mis pies y comenzaron a llover estalactitas desde arriba. Debió haber sido algún tipo de actividad tectónica. La mayoría de los planetas no experimentaban el fenómeno, pero sabíamos que, en los que sí lo hacían, podía causar estragos en las estructuras artificiales. Teníamos que abandonar la mina ahora, antes de que nos enterraran vivos.

Un grito resonó más allá del túnel, pidiendo ayuda. Reconocí la voz de la novia de Kel, palabras cargadas de dolor. Mi hijo se giró en dirección a su llamado y pude imaginar la preocupación que se reflejaba en su rostro a través del casco opaco.

—¿Papá? Vuelvo enseguida, tú sigue. Kel salió corriendo antes de que pudiera intentar detenerlo.

El terror me recorrió las venas. —¿Kel? ¡Kel!

Sentí a lo lejos que una mano me agarraba el hombro y la mina se disolvió en la oscuridad. Abrí los ojos y volví a mirar la extraña nave. La criatura pálida, que se hacía llamar “Rykov”, estaba de pie junto a mí. Su expresión parecía preocupada.

Me froté el punto dolorido detrás de la oreja, donde me habían inyectado un implante de lenguaje. Las crestas de una fina cicatriz presionaron contra mis dedos. Me sirvió como confirmación de que los acontecimientos del día anterior eran reales y no un sueño febril.

—¿Estás bien? —preguntó Rykov—. Hablabas en sueños y parecías molesto.

Suspiré, las imágenes de mi hijo todavía revoloteaban en mi mente. “Estoy bien. Fue solo una pesadilla”.

Él asintió y se detuvo un momento. “¿Quién es Kel?”

—Kel... —cerré los ojos, intentando no llorar—. Es mi hijo. Está muerto.

—Lo siento. Sé cómo se siente —hizo una mueca, como si sintiera dolor—. Es lo peor que le puede pasar a un padre.

“¿Perdiste un hijo?”

—Sí. Mi hija menor, Alina. Tenía solo tres años cuando el cáncer se la llevó. Es una enfermedad terrible que pone a tu propio cuerpo en tu contra. Ella luchó con mucha fuerza, con mucho dolor. Hicimos todo lo que pudimos, pero ninguno de los tratamientos hizo nada. Sé que ella habría tenido mucho que ofrecer al mundo, si tan solo hubiera tenido la oportunidad.

—Lo siento, Rykov. Tan joven... —Una lágrima se deslizó por mi mejilla—. Kel era mi único hijo. Hubo un terremoto y nuestras minas se estaban derrumbando. Volvió corriendo para salvar a su novia. Tal vez si yo hubiera ido con él, las cosas habrían sido distintas, pero huí como un cobarde. Nunca logró salir. ¿Qué clase de padre soy?

Su ceño se profundizó. —No puedes culparte, Byem. Yo mismo caí en esa trampa. No es tu culpa. A veces, simplemente no hay nada que puedas hacer.

Oí un tintineo y sentí que sus manos se cerraban firmemente alrededor de mi muñeca izquierda. Observé cómo insertaba un pasador en el grillete y, con un clic, la banda se soltó. La piel donde había estado la atadura estaba irritada y aún quedaba un tono violeta oscuro en su lugar.

Rykov también me soltó el brazo derecho y luego dio unos pasos hacia atrás con cautela. Sus ojos no me dejaron ni un momento. También noté que tenía la mano suspendida sobre la cadera, donde parecía tener un arma escondida. ¿Pensaba que iba a abalanzarme sobre él, como si fuera una especie de animal salvaje? Nuestras interacciones no debieron haber aliviado todas sus sospechas.

Me estiré con movimientos lentos y deliberados y luego me puse de pie. No estaba claro por qué me habían soltado, pero tenía que haber algo que querían de mí.

Rykov se quedó mirándolo unos instantes más y luego relajó un poco su postura. —Sígueme.

Fue un paseo corto hasta nuestro destino, un hangar repleto de elegantes cruceros. Unos cuantos trabajadores inspeccionaban el estado de las naves y realizaban reparaciones, pero la mayoría del personal se paseaba sin ninguna tarea asignada. Los sonidos de charlas y risas zumbaban en el aire. El ambiente alegre me resultaba extraño; mi gente no había tenido ese espíritu en décadas.

Los miembros de la tripulación se quedaron callados al notar nuestra entrada y todos los ojos se volvieron hacia mí. Una oleada de susurros recorrió la sala. Agaché la cabeza, con la ansiedad burbujeando en mi pecho. Era probable que algunos de ellos albergaran sentimientos negativos hacia mi especie, así que dudaba de que mi presencia fuera bien recibida allí.

—¡Muy bien, quiero que todos ustedes escuchen! —gritó Rykov—. Estoy a punto de hacer una llamada al general Kilon y repasaremos los detalles de la misión.

Sacó un holopad de su bolsillo y hojeó algunas pantallas. Observé por encima de su hombro cómo aparecía el ser de tres ojos de mi primer interrogatorio.

—Hola, comandante —dijo el general—. Las naves furtivas deberían haber llegado, como usted solicitó.

Rykov hizo un gesto hacia atrás. “Sí, los tenemos aquí mismo, gracias. No estoy seguro de cómo convenció al Portavoz”.

—Años de práctica. —Las palabras estaban marcadas por una satisfacción petulante—. Pero no se equivoca al decir que nos estás espiando. ¿Te importaría explicarme eso?

El Comandante se movió torpemente. “Nosotros, eh, espiamos a todo el mundo. La Federación nunca ha sido… particularmente franca con nosotros. De todos modos, ¿recibiste el plan que te envié?”

—Sí, y sólo tengo una pregunta —suspiró el general Kilon, con una expresión exasperada en su rostro—. ¿Son todas tus ideas tan descabelladas?

“¿Qué? No veo el problema. ¿Estás diciendo que tienes algo mejor en mente?”

``Bueno no."

—Está elaborado en base a la información que Byem nos dio. Ahora tenemos un diseño aproximado del planeta. Hay doce asentamientos principales y el resto de la población está desplegada fuera del planeta. Hoy vamos a evacuar el más grande. Nuestros cazas se enfrentarán a las fuerzas de la IA en órbita, la mantendrán distraída, mientras las naves furtivas descienden a escondidas para rescatar a la gente. Eso tiene sentido, ¿no?

—Sí, pero ese no era todo el plan, comandante. Dejaste algunas cosas fuera. Como dejar que Byem volara una nave furtiva, las imposibles limitaciones de tiempo... Ah, sí, y la parte sobre las bombas de antimateria.

Mis ojos se abrieron de par en par al darme cuenta de lo que había dicho el general Kilon. ¿Su estrategia me involucraba como piloto? Después de años de reclutamiento, lo último que quería era volver a sumergirme en la guerra.

Rykov se encogió de hombros. “Hay centinelas mecánicos apostados por toda la ciudad, vigilando a la gente. Los humanos que caminan por ahí se notarían, pero Byem no se notaría. Si eliminamos de inmediato a los centinelas e intentamos evacuar a los civiles, podrían vernos como una amenaza y luchar. Necesitamos que Byem los convenza de que vengan con nosotros”.

—¿Y realmente crees que puedes completar todo esto en cuarenta minutos?

“La IA matará a la gente si sospecha que está perdiendo. Como tú y yo vimos, no permite la posibilidad de captura. Así que destruir todas sus fuerzas y reservas no es una opción, se trata más bien de ganar tiempo. Le daremos a Byem veinte minutos en tierra, luego eliminaremos a los centinelas. Tenemos unos veinte minutos más antes de que lleguen los drones de seguridad, y debe estar hecho para entonces”.

“Está bien, ¿qué tal si…”

“¿Las bombas de antimateria? La IA no puede darse cuenta de que la gente escapó. Si reducimos toda la ciudad a cenizas, con suerte pensará que todos están muertos”.

“Hay muchas cosas que podrían salir mal en esto. Todo tiene que ser perfecto”. El general dudó. “Tendrá mi apoyo, comandante, pero no haga que me arrepienta. Hablaré con usted después de la misión”.

Me quedé mirando fijamente al suelo cuando terminaron la llamada. Mi nombre se mencionaba en sus planes demasiado para mi gusto. No quería participar en el riesgo, en el peligro que implicaba todo aquello. ¿Podría realmente convencer a todo un asentamiento de que se marchara con soldados alienígenas, dentro de su plazo, de todos modos?

Rykov me miró y sonrió con confianza. “Bueno, ya has oído todo eso, Byem. ¿Qué dices? ¿Estás listo para salvar el día?”

Mil razones para no aceptar pasaron por mi mente. Podría haber consecuencias por rechazarlo, pero sabía que no era ningún héroe. Podían arrojarme a una celda y tirar la llave, era preferible a volver a casa. Todo lo que tenía que hacer era expresar una negativa rotunda.

Pero en lugar de eso, las palabras que salieron de mi boca fueron: “Cuenten conmigo”.