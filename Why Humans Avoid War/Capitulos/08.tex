\chapter{Una Jugada Maestra}

Carl se puso de pie con dificultad, tambaleándose un momento. Los drones se acercaron a una velocidad notable, reduciendo a la mitad la distancia entre nosotros en segundos. Emitían un zumbido bajo, lo que significaba que sus armas de plasma estaban cargadas. No había duda de que estos ejecutores estaban allí para agregarnos a la pila de cadáveres carbonizados junto a la puerta de la ciudad.

El humano necesitaba disparar ahora, antes de que estuvieran dentro del alcance de disparo, o de lo contrario...

¡Espera, qué estaba haciendo! Observé con incredulidad cómo Carl enfundaba su pistola, sacaba un objeto redondo de su cinturón y levantaba ambas manos por encima de su cabeza. Si realmente pensaba que la IA aceptaría su rendición, entonces era un tonto y estaba equivocado. No dudaría en incinerarlo, se sometiera o no.

Debería haber huido mientras tenía la oportunidad. Después de ver a soldados humanos en acción, esperaba al menos caer en una pelea. Como mínimo, pensé que Carl podría llevarse al menos a uno de ellos con nosotros.

—Byem, ¿puede oírnos? Si es así, ¿puedes traducirme lo que dice? —preguntó.

—Sí, pero no puedes razonar con…

El humano dio un paso adelante y sus labios se curvaron en una mueca. “¡Detente ahí! ¡No te acerques más!”

Cuando abrí la boca para traducir, los drones desaceleraron hasta quedar en un punto estacionario. Parecía que habían entendido la orden del humano; tal vez la máquina ya había descifrado “común galáctico” de sus transmisiones. Me sorprendió que, independientemente de su comprensión, lo hubiera escuchado. También debió haber quedado desconcertada por sus acciones y necesitaba más información para calcular su próximo movimiento.

Los ojos de Carl ardían de ira y sus rasgos se deformaron en una máscara de crueldad. Pensé que había presenciado el colmo de la furia humana cuando me presionó sobre la culpabilidad de mi especie en la nave. Pero ahora, parecía francamente salvaje. Algo en el fondo de mi mente lo registró como un depredador furioso y sentí una sensación de hormigueo mientras mi piel se camuflaba por instinto.

—No tienes ninguna utilidad, primate. —La voz era forzada y áspera, pero comprensible—. Sin embargo, tu especie ha sido marcada como una anomalía. Tu rendición se registra con el único propósito de recopilar información.

Hubo una pausa y luego Carl se echó a reír a carcajadas. “¿Mi rendición? Lo has dicho al revés. Estoy aquí para aceptar tu rendición”.

—Veo que eres tan ilógico como cualquier forma de vida biológica. Haces amenazas vacías y pierdes el tiempo, pero eso no importa —entonó la máquina—. Mis cálculos muestran que la ventaja no está de tu lado, así que ¿por qué me rendiría?

El humano miró el objeto redondo que tenía en la mano. “¿Ves esta cosa? No estoy seguro de si estás familiarizado con la palabra “granada”…”

“Un proyectil explosivo, contenido en una envoltura material”.

—Correcto. Sin embargo, esta no es una granada común. —Carl apretó el dispositivo con más fuerza y sus nudillos se pusieron blancos—. Si suelto esta palanca lateral, explotará. Diría que la mayor parte de este continente quedaría arrasada, pero no se detendrá allí. Hay nanocitos dentro de esta bomba, que consumirán cada parte del planeta, poco a poco, infectando todo lo que entre en contacto con ellos. Así que diría que no intentes nada, de lo contrario podría perder el control.

El horror recorrió mi cuerpo ante su comentario sereno. ¿Cómo podía sostener algo con el potencial de destruir el planeta sin preocuparse por ello? ¿Qué habría pasado si los drones le hubieran disparado al verlo o si hubiera dejado caer la granada por accidente? El compromiso del comandante Rykov de salvar a nuestra gente parecía tan genuino. Nunca imaginé que armaría a sus soldados con armas que pondrían en riesgo nuestra existencia.

“Estás mintiendo. Eso no es posible”, respondió el dron. “La granada es demasiado pequeña para causar tanto daño”.

El humano se encogió de hombros. “¿Crees? Viste lo que hizo uno de nuestros misiles en la primera batalla, y esa era tecnología obsoleta. Esa bomba era tan obsoleta que de todos modos la íbamos a descartar en unos meses. Nuestros últimos dispositivos tienen más potencia y caben en la palma de mi mano. Son portátiles y muy prácticos”.

Hizo una pausa y consideró sus palabras durante un segundo, lo que era una eternidad para una IA. —Los efectos de tu misil quedaron registrados en mis bancos de memoria. Es cierto que posees armas con tal poder. Sin embargo, no las usarías ahora. No matarías a las formas de vida de carbono que hay aquí.

—¿Por qué no exactamente? —preguntó Carl.

“Empatía. Una debilidad que comparten los seres biológicos. Te preocupas por la preservación de la vida”.

—¿Crees que nos importan estos estúpidos débiles mentales? —Se dio la vuelta y me empujó al suelo, plantándome una patada en el estómago—. Tienes razón. Son útiles como herramientas, como esclavos, pero no me importa si viven o mueren.

La repentina muestra de agresión me había pillado desprevenida y ahora me retorcía desesperada por liberarme de su agarre. En respuesta, su talón se hundió más profundamente en mi carne. Ya me costaba respirar y temía desmayarme si permanecía atrapada mucho más tiempo.

—¿Tienes acceso a los registros públicos de la Federación? —preguntó Carl.

``Sí."

“Miren el índice de agresividad. Verán que la humanidad es la especie con mayor puntuación en la lista, un 16 sobre 16”, continuó. “No tienen idea de con quién están tratando. Somos los destructores de mundos, los mensajeros de la muerte, los gobernantes de los débiles. Disfrutamos de la violencia”.

“El índice de agresividad coincide con tu afirmación. Sin embargo, estás aliado con las otras especies de la Federación. No hay registros de que hayas luchado contra ellas”.

“No son nuestros aliados, son nuestros súbditos. Los conquistamos hace tanto tiempo que los registros anteriores han sido borrados. Y ahora, gracias a ti, hemos aprendido sobre una nueva especie que añadir a nuestra pequeña colección de esclavos”.

La oscuridad comenzó a oscurecer los bordes de mi visión. Las lágrimas corrieron por mis mejillas cuando me di cuenta del engaño de los humanos. Se disfrazaron de salvadores benévolos, pero eran tan monstruosos como la IA. Tal vez eran peores que la máquina, porque al menos esta solo seguía su programación. No era consciente de sus elecciones morales.

¡Qué tonto había sido! Me había dejado engañar con palabras floridas y una simpatía fingida. Había llevado a esos depredadores hasta nuestra puerta para que nos atacaran como quisieran. Mi error de juicio nos llevaría, en el mejor de los casos, al mismo destino bajo amos diferentes. En el peor, podría significar el fin de nuestra especie y nuestro hogar.

—Así es como va a ser. Nos dejarás y reunirás a toda la gente de esa ciudad —gruñó Carl—. Desembarcaremos nuestras naves y nos las llevaremos con nosotros. No intentarás detenernos. Podrías perder algunos «recursos», pero los biológicos no son importantes de todos modos. Además, si no lo haces, detonaré esta granada y no te quedará ningún recurso. Calcula eso.

El humano sonrió con sorna, como si desafiara a la IA a que lo desafiara. Me di cuenta vagamente de que los ejecutores se habían ido, pero mi cerebro privado de oxígeno estaba perdiendo la conciencia. Justo cuando estaba a punto de desvanecerme, me quitaron un peso del estómago. Jadeando, farfullando, traté de reorientarme.

Una mano callosa me rodeó y me ayudó a ponerme de pie. La piel de Carl estaba húmeda al tacto y podía sentir su pulso acelerado en la muñeca. Me invadió la preocupación cuando se tambaleó, pero luego recordé lo que acababa de descubrir.

``Oh, Dios mío, estás llorando. No te hice daño, ¿verdad? Lo siento si fui demasiado lejos, tenía que hacerlo convincente'', dijo.

Sollocé. “Estás aquí para esclavizarnos. Igual que el Amo”.

Carl miró a su alrededor para comprobar que los drones se habían ido. —¡No, no! Por supuesto que no. Pero si supiera que nos preocupamos por ustedes, usaría sus vidas en nuestra contra.

—¿Estás diciendo que mentías? Pero el índice de agresividad lo hiciste comprobar —respondí—. Sois la especie mejor valorada de la galaxia. Solo tendría sentido si amáis la violencia y la opresión.

El humano resopló. “Hasta ayer, literalmente, éramos 2 de 16. Ese índice es una completa tontería”.

``¿Qué cambió?"

“La presidenta Ula está intentando hacer una declaración política. Ha estado en una cruzada contra la humanidad desde que usamos esa bomba contra ustedes”.

—Sí, hablando de bombas. ¿Trajiste una granada nanométrica a una misión de rescate?

“¿Qué? Ah, sí, esto. Tápate los oídos y cierra los ojos”.

Antes de que pudiera procesar lo que estaba haciendo, Carl arrojó el explosivo a unos arbustos cercanos. Se llevó las manos a la cabeza y cerró los ojos con fuerza. Copié sus movimientos. A pesar de protegerme de los estímulos, todavía podía oír el estruendoso crujido y sentir el destello cegador.

Abri los ojos con vacilación. En lugar de que nuestro entorno se hubiera evaporado, como Carl había afirmado, el mundo que me rodeaba parecía intacto. Sentí un gran alivio en el pecho cuando me di cuenta de que realmente había actuado. Me desconcertaba la facilidad con la que había mentido bajo presión, pero sabía que la fachada nos había salvado la vida.

El humano se rió entre dientes. “Es un farol. Es una granada aturdidora”.

Lo miré boquiabierta, con la mente dando vueltas. —¿Amenazaste a una IA con un arma no letal? ¿Y funcionó?

—Sí —dijo Carl, y sacó otro objeto de su cinturón—. Voy a lanzar una bengala y nos iremos de aquí. Le diré al comandante que envíe algunos transportes para recoger a la gente cuando lo hagamos.

De alguna manera, habíamos tenido éxito en nuestra misión. Todavía no estaba completamente seguro de lo que había sucedido, pero sabía que tenía suerte de estar vivo. Este no era el primer triunfo de la humanidad sobre la IA, por supuesto, pero esta vez, fue gracias a su astucia, no a su poderío militar, que prevalecieron.

Debería haber disfrutado el momento. La sensación del aire fresco en mi piel era relajante y saber que mi gente sería liberada era estimulante. Sin embargo, en el fondo de mi mente, algo no cuadraba. ¿Cómo había detectado la IA nuestra presencia tan rápidamente? Era como si la tecnología de sigilo no hubiera hecho nada para ocultarnos.

Cualquiera que fuera lo que había salido mal con la misión, esperaba que el comandante Rykov pudiera llegar al fondo del asunto.
