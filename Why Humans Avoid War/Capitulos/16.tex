\chapter{El Desafío del Suero}\label{sec:el-desafio-del-suero}

La llegada de la nave insignia fue anunciada por una ola de muerte, una combinación perfecta de brutal armamento cinético y nuevos rayos de plasma que arrasaban con todo lo que se cruzaba en su camino. No hubo pausas ni treguas, solo una corriente incesante de destrucción. Acorralados en las estrechas calles, los soldados de Xanik podían ser eliminados como peces en un barril. Había algo casi hermoso en su ejecución, de la misma manera que un espectador podía apreciar la habilidad de un cirujano al diseccionar a un paciente.

Estaba intentando olvidar la sangre verde que me empapaba las manos y el cuerpo alienígena sin vida que yacía a mi lado, lo suficiente para transmitir las coordenadas a través de mis auriculares. Pedir fuego tan cerca de la posición de mis soldados era, por supuesto, peligroso; una sola falla en la comunicación haría que las fuerzas terrestres fueran las receptoras de las descargas de los cañones. Yo era responsable de todas las vidas bajo mi mando; por el bien de mis hombres, no podía permitirme el lujo de dejar que las emociones me dominaran.

Sin embargo, los pensamientos no deseados seguían arrastrándose. El general Kilon tenía sus defectos, pero había demostrado ser un aliado leal en un momento en que estos escaseaban. Me agradaba y a menudo lo consideraba con el mismo respeto que a mi propia tripulación. Había sido idea mía traerlo conmigo y, aunque nunca podría haber imaginado el baño de sangre que se produjo, me sentía responsable de su destino.

No podía dejarlo morir. Lo más seguro era sentarme y esperar a que se calmara el polvo, pero para entonces ya sería demasiado tarde.

Una silueta apareció en el borde de mi visión y rápidamente levanté mi rifle.

—¡No dispares! —dijo Pavel, que tenía una venda roja en la pierna. En su camisa había una mezcla de sangre humana y no humana, lo que indicaba que también le habían dado en el torso—. Sé que tengo mala pinta, pero deberías ver al otro tipo.

Apreté los dientes. ``Vi al otro tipo. Eso estuvo fuera de lugar y ahora mi amigo se está muriendo por tu culpa''.

Me di cuenta demasiado tarde de que me había referido a Kilon como amigo. Los ojos de Pavel se posaron en el cuerpo del general y casi pude ver cómo giraban los engranajes en su cabeza. No dijo nada, pero sus pensamientos eran evidentes en su expresión. Consideró que mi apego a un oficial de la Federación de larga data era ingenuo y prematuro.

—Veamos —murmuró Pavel, agachándose junto al general con una mueca—. ¿Lo abatieron de un solo tiro en el estómago? Intentaré curarlo.

Mi hermano se quitó una mochila de los hombros y sacó de dentro un botiquín de primeros auxilios portátil. Volví a centrarme en el campo de batalla mientras él empezaba a suturar la herida. Un gran grupo de soldados de Xanik se refugiaban en el vestíbulo de un hotel de lujo que había al final de la calle. Tal vez pensaron que refugiarse en el interior los salvaría de la ira del buque insignia. Ordené que lanzaran una bomba sobre el edificio y, momentos después, vi cómo implosionaba sobre sí misma, tan fácilmente como un castillo de naipes. Tenía la terrible certeza de que la unidad estaba sepultada, o más bien, aplastada por los escombros.

Con la disminución de su número, esperaba que la docena de soldados Xanik restantes se rindieran a la Unión Terran en cualquier momento. Efectivamente, vi un grupo de ellos caminando hacia atrás, hacia nuestra línea del frente, sosteniendo sus rifles sobre sus cabezas. El equivalente a una bandera blanca en la Tierra.

—¡Alto el fuego! Desármenlos y conténtenlos, nada más —dije por los auriculares, lanzando una mirada intencionada a Pavel—. No quiero que los prisioneros de guerra sufran ningún daño, ¿me entienden?

Pensé que mi hermano estaba demasiado absorto en las heridas del general como para oírme, pero asintió en señal de reconocimiento. —Somos dos caras de la misma moneda, Mikhail. Hacemos lo que hay que hacer por el bien común y mantenemos tus manos limpias.

Suspiré. La Agencia tenía sus utilidades. Ayudó a presentar una imagen unificada de la Tierra a la comunidad galáctica, cuando en realidad, nuestros gobiernos regionales todavía se peleaban por cada política menor. Barría los incidentes diplomáticos bajo la alfombra, como la vez que los Hoda'al atraparon a un puñado de nuestros espías copiando documentos confidenciales. Proveía la base de nuestra inteligencia militar, manteniéndonos informados sobre las capacidades militares de la Federación (no es que estuviéramos particularmente impresionados).

Sin embargo, su actitud maquiavélica dejaba mucho que desear.

—Sigue diciéndote eso, si te ayuda a dormir por la noche —refunfuñé—. ¿Cómo está el general?

A mis ojos no se notaba si el Jatari aún respiraba. La decoloración de su piel lo había dejado pálido como un cadáver y temí que bien pudiera estar muerto.

—Tu amigo —Pavel se detuvo en la palabra amigo, con un tono de voz marcado por el desagrado— ha perdido mucha sangre. Necesita una transfusión y un milagro. Aquí sólo tengo sangre humana.

Fruncí el ceño. —Estás diciendo que va a morir.

``Yo diría que es probable. Se puede intentar llevarlo al buque insignia, administrarle muchos líquidos y ponerlo en soporte vital, pero es una apuesta arriesgada'', respondió.

Respiré profundamente. ``Está bien, está bien, algo es algo. Convocaré una lanzadera''.

``Pensé que era demasiado arriesgado volarlos. La Federación podría derribarlos''.

``Acabamos de salvarles el pellejo, más vale que no lo hagan. Y, de todos modos, tenemos que intentarlo''.

``Mikhail, dudo que eso haga alguna diferencia. No están hechos como nosotros. Su corazón es demasiado débil, ese es el problema''.

``¿Qué pasaría si le diéramos el suero? Mejora la fuerza, el tiempo de recuperación y, lo más importante, la función cardiovascular''.

``No puedes hablar en serio.''

El suero era la forma que tenía el profano de referirse a la nanotecnología genética, que se utilizaba para mejorar a nuestros soldados mediante la manipulación genética y la terapia regenerativa. Si bien los humanos eran más resistentes que la mayoría de los alienígenas, los nanobots que reparaban su tejido en tiempo real eran la razón por la que Pavel seguía en pie, a pesar de haber recibido dos disparos.

La ingeniería genética era un secreto de estado, ya que estaba prohibida por la ley galáctica. Honestamente, la prohibición existía por razones sensatas; la Federación estaba preocupada por los impactos a largo plazo en el acervo genético de una especie. Sin embargo, la humanidad no era exactamente una especie reacia al riesgo, y estaba lo suficientemente contenta de experimentar en el campo de manera discreta. Los asuntos de la Tierra no eran monitoreados de cerca, al menos mientras la Federación la había considerado un planeta pacifista.

—Lo digo en serio. Podría funcionar —dije.

``Piénsalo lógicamente por un segundo. No es humano y nunca se ha probado en especies alienígenas'', respondió Pavel. ``No tenemos idea de cómo lo afectará. Incluso podría matarlo''.

``De todos modos, va a morir. ¿Qué tenemos que perder?''

—Digamos que sobrevive. ¿Crees que no se dará cuenta de lo que hicimos? Entonces se lo contará a la Federación y volveremos a estar en su lista negra. ¿Qué pasará cuando les diga que asesinamos a Cazil?

``Kilon no es leal a la Federación. Podemos confiar en él''.

``¿Crees que podemos permitirnos correr ese riesgo? Tendríamos que mantenerlo en la Tierra, de forma permanente, para estar seguros''.

—No será necesario. Creo que puedo lograr que deserte, por su propia voluntad.

Las cejas de Pavel se alzaron con sorpresa. La idea de que la Unión Terrana reclutara a un oficial de alto perfil de la Federación para nuestras filas debió haber sido suficiente tentación, porque comenzó a hurgar en su mochila. Sacó un pequeño frasco de líquido transparente, llenó una jeringa e inyectó al general sin decir una palabra más.

No estaba convencido de poder llevar a Kilon a nuestro lado. Con lo apegada que estaba su especie al honor, no estaba claro si siquiera consideraría abandonar su mundo natal. Pero lo importante ahora era salvar su vida.

Caminé de un lado a otro mientras pedía por radio que me evacuaran. Mi mirada se desviaba una y otra vez hacia el general, estudiando su figura en busca de la más mínima señal de esperanza. No hubo ningún cambio inmediato en el estado del Jatari, pero esperaba que el suero pudiera darle una oportunidad de luchar.