\chapter{Descenso y Desesperación}\label{sec:descenso-y-desesperacion}

Me sentí agradecido de que mi compañero humano estuviera al mando de la nave furtiva. Con la gran variedad de botones y palancas que había en el interior, era poco probable que mi experiencia de vuelo se hubiera traducido en algo. Podía simplemente sentarme, admirar la vista e intentar calmar mis nervios.

Nuestro descenso a través de la atmósfera había sido lento y metódico, ya que los humanos querían explorar el paisaje en lugar de avanzar a ciegas. No estaba seguro de cómo podían distinguir algo desde esta altitud. Para mí, las estructuras de abajo eran poco más que contornos borrosos.

Debieron haber visto suficiente, porque unos minutos después, nuestros auriculares anunciaron una serie de coordenadas. Cuando las conectamos a nuestro sistema de navegación, marcaron un punto de aterrizaje justo en las afueras de la ciudad. Nos lanzamos hacia el suelo en un ángulo mucho más pronunciado que antes. El resto de nuestra formación nos siguió de cerca.

Había llegado el momento de la verdad. Sentí náuseas mientras pensaba en la posibilidad de que me detectaran. Sin la protección de las nubes para ocultarnos, me sentía vulnerable y expuesta.

“¿Humanos? ¿De verdad somos invisibles?”, susurré.

Resopló molesto. “Mi nombre es Carl, no soy humano, Devorador ” .

Fruncí el ceño, confundida por su respuesta. “¿Devorador?”

—Ah, eh… así es como llamamos a vuestra especie. Supongo que no es vuestro nombre real —respondió—. Ya sabéis, porque destruís todo lo que tocáis.

El nombre que nos habían dado confirmó mis sospechas sobre cómo nos veían los humanos. Las miradas de hostilidad que me lanzaron en el hangar fueron una buena pista, pero oír a uno de ellos expresar esos sentimientos con palabras me impactó de otra manera. Me dolió darme cuenta de que nos veían como poco más que una plaga en el universo.

—No te gusto, Carl —me aventuré a decir.

—Sí, tienes razón. No tengo idea de por qué te estamos ayudando. —El humano se giró para mirarme, con el ceño fruncido—. Ustedes fueron cómplices de todo lo que hizo la maldita IA. Miles de millones de personas inocentes han muerto debido a sus acciones. ¿Y ahora se hacen las víctimas?

Me encogí bajo la intensidad de su mirada. “No lo entiendes”.

“Entonces hazme entender”, dijo.

—Todos los que se opusieron murieron. Como mi padre —mi voz tembló al recordar ese fatídico día—. Era un oficial de policía y, cuando los drones llegaron a nuestra ciudad, se unió a su defensa. Poco después encontraron su cuerpo, quemado hasta quedar irreconocible por el fuego de plasma.

La expresión de Carl se suavizó. “Lo siento.”

—Yo tenía sólo siete años entonces. A los que sobrevivimos nos llevaron a campos de concentración. Nos llevaron al límite físico y, si no caías exhausto, podías morir de enfermedad —continué—. Cualquiera que desertara o se rebelara sufría una muerte horrible y se convertía en un ejemplo público. Al final, pierdes la esperanza y haces lo que te dé la gana. Si no lo haces tú, alguien más lo hará de todos modos.

El humano se quedó callado, lo que esperaba que fuera una señal de que mis palabras habían llegado a él. Si esta misión iba a ser un éxito, necesitaba la cooperación incondicional de mi compañero. No podíamos permitirnos el lujo de que se generaran hostilidades entre nosotros.

—De todos modos, no respondiste a mi pregunta. ¿Estás seguro de que somos invisibles? —pregunté.

Carl le ofreció una sonrisa tranquilizadora. “Deberíamos estarlo. No hay nada de qué preocuparse, relájese”.

Señalé un indicador parpadeante en la pantalla de armas. “Bueno, entonces, ¿qué es eso?”

Sus ojos se clavaron en las flechas rojas, que se acercaban rápidamente a nuestra posición. El color desapareció de su rostro, una visión que me hizo estremecer. La mayoría de los humanos estaban bastante pálidos en su estado normal, pero Carl se había vuelto tan pálido que parecía un cadáver. Temí que se desplomara frente a mí.

El humano encendió sus auriculares. “¡Misiles en camino, prepárense para el impacto! Nos han detectado”.

Unos momentos después, el barco se sacudió violentamente. Mi cuerpo se tambaleó hacia adelante, pero el arnés de seguridad me arrojó hacia atrás y me hizo volver a caer en la silla. Me quedé sin aire en los pulmones y sentí que el cerebro me temblaba en el cráneo. Sentí un mareo que se apoderó de mi mente y que se agravó cuando el barco entró en una espiral descontrolada.

Vi a Carl tirando desesperadamente de la palanca de control, pero no sirvió de nada para estabilizar nuestro vuelo. La necesidad de vomitar se hizo más fuerte a medida que aumentaba la aceleración. Era cuestión de segundos antes de que nos estrelláramos contra los campos de abajo.

Así fue como todo terminó. Me hubiera gustado decir que acepté mi muerte con calma, pero la verdad era que estaba aterrorizada. Mi último pensamiento antes del impacto fue maldecirme a mí misma por haber aceptado este plan descabellado y preguntarme por qué había actuado en contra de mi mejor criterio.

Hubo una sacudida cuando la nave se estrelló contra el suelo, seguida de un chirrido al romperse en múltiples pedazos. Objetos sueltos y escombros pasaron rodando junto a nosotros y, pensando rápidamente, me agaché para protegerme la cabeza. Nos deslizamos por el suelo durante lo que pareció una eternidad, antes de detenernos finalmente.

Aparte de algunos cortes y magulladuras menores, yo estaba ileso. Sin embargo, no se podía decir lo mismo del barco. Al mirar a mi alrededor y ver la devastación, pensé que alguien que pasara por allí podría haber confundido los restos con el paso de un ciclón. Fue un milagro que la cabina del piloto se hubiera mantenido intacta en su mayor parte.

Me sorprendió bastante seguir con vida, pero no parecía que fuera el momento de celebrar. El olor acre del humo me llegó a la nariz, lo que indicaba que era necesario evacuar rápidamente. Me fue bastante fácil desabrochar el arnés, a pesar de que me temblaban las manos. Ahora, lo único que me quedaba era salir al aire libre.

Antes de salir de la nave, pensé en ver cómo estaba Carl, solo para asegurarme de que estaba bien. Cuando mis ojos se posaron en el humano, mi alivio se convirtió en consternación. Estaba desplomado en su silla, sin reaccionar. Un líquido carmesí rezumaba de un corte en la parte posterior de su cabeza, manchando su cabello rubio escarchado. Supuse que era sangre, a pesar de la coloración inusual.

Corrí a su lado y lo sacudí por los hombros. —¡No, no, no, despierta!

Los ojos del humano se abrieron de golpe y gimió. Si mi especie hubiera sufrido ese tipo de herida en la cabeza, probablemente estaríamos muertos; recuperar la conciencia habría sido imposible. Pero, claramente, los humanos eran más resistentes. La pregunta era cuánto lo afectarían sus heridas y si sería capaz de caminar por sus propios medios.

Carl me miró mientras yo le quitaba el arnés. —¿Puedes ayudarme a salir de aquí? No te estoy pidiendo que me lleves como a una princesa, pero...

—Sí, claro. No te dejaría aquí —respondí.

Me pasé el brazo por el cuello y me preparé para soportar su peso. Logramos salir tambaleándonos de entre los escombros, pero Carl se desplomó de rodillas unos pasos más allá del campo. Era evidente que no estaba en condiciones de andar por ahí. Con suerte, el resto de nuestro séquito todavía estaba en condiciones de volar. Sería un consuelo saber que estaban allí, preparando un grupo de rescate.

El humano se llevó una mano a la herida y puso una mueca. —¿Qué tal si descansamos un poco aquí? Necesito un momento.

—Está bien. Está claro que el Maestro… la IA sabe que estamos aquí ahora. No creo que fuéramos invisibles. ¿Qué haremos ahora exactamente? —pregunté.

“Improvisamos”, gruñó. “Nuestro mayor error fue confiar en la tecnología de la Federación, pero era un plan terrible desde el principio. Algo iba a salir mal”.

La alarma recorrió mis venas cuando Carl sacó una pistola de su funda y caí hacia atrás en mi prisa por escapar. No había sido mi intención provocarlo, pero pensé que mi crítica a su mando no sería bien recibida. Sin embargo, en lugar de apuntarme a la cabeza, extendió un brazo para ofrecerme el arma.

—Por favor, dime que sabes disparar uno de estos, Byem —dijo.

Empujé el arma hacia él. “Bueno, no exactamente. Solo nos entrenan en combate aéreo”.

Soltó un suspiro de exasperación. “Está bien, entonces estamos jodidos. Hay tres drones acercándose a tu izquierda y supongo que no son amistosos”.

Efectivamente, un trío de drones de seguridad se acercaba volando desde la ciudad. El instinto de huir era abrumador, pero logré mantenerme firme. Carl no merecía morir solo. Había abandonado a mi propio hijo para salvar mi pellejo, pero no estaba dispuesta a cometer el mismo error dos veces. Lidiar con esa culpa una vez más sería demasiado para soportar.

Mi única esperanza era que un humano herido pudiera vencer a un escuadrón de ejecutores mecánicos. Su especie no había tenido problemas para derrotar a la IA en encuentros anteriores, pero estas circunstancias eran muy diferentes.

Quizás era pedirle demasiado a Carl, pero incluso en su estado de debilidad, no estaba listo para descartarlo todavía.