\chapter{Fuegos Artificiales y Traiciones}\label{sec:fuegos-artificiales-y-traiciones}

Ninguna misión salió según lo planeado cuando había humanos involucrados.

No es que el plan fuera particularmente sensato, ya que la distracción de Pavel implicaba disparar cohetes al cielo. No estaba seguro de que encender explosivos fuera una buena idea, a menos que el objetivo fuera que los humanos hicieran estallar la embajada por sí mismos. Pero el comandante Rykov ordenó a sus hombres que prepararan el lanzamiento, aparentemente ajeno a lo extraña que era la sugerencia de su hermano.

``¿En qué sentido exactamente va a servir de algo lanzar bombas al aire?'', pregunté, sin poder contener más mis dudas.

El comandante se rió entre dientes. ``No son bombas, son fuegos artificiales''.

Lo miré con cara de pocos amigos, sin entender qué le parecía divertido. ``Así que estos cohetes tienen un nombre. No veo cómo eso le resta valor a mi argumento''.

``Son explosivos recreativos. Casi inofensivos''.

``¿Explosivos recreativos?'', repetí. ``¿Qué significa eso?''


``La gente los dispara en días festivos para celebrar. Son ruidosos, explotan en un destello de color brillante, pero eso es todo. Ven a la Tierra el Día de la Unificación y verás muchos de ellos''.

``¿Eso se supone que es divertido?''


``Sí.''


``Yo…nuestra especie claramente tiene ideas muy diferentes sobre la diversión''.

—Bueno, sea divertido o no, debería hacer que los terroristas salgan a investigar. Supongo que estarán tan confundidos como tú.

Un rastro de brasas anaranjadas se arqueaba a través del cielo nocturno, como un grupo de estrellas fugaces. Se escucharon algunos aplausos de los soldados terranos, la mayoría de los cuales admiraban el espectáculo de luces que había en lo alto. Hice una mueca ante el volumen de la explosión; no estaba del todo convencido de que el fuego no fuera a caer sobre nuestra posición. Los patrulleros de Xanik parecían completamente asustados, y muchos se agacharon para cubrirse.

El comandante Rykov sonrió burlonamente, intentando claramente no reírse del aterrorizado embajador Cazil. ``Tranquilo, es perfectamente seguro''.

Cazil se frotó el pico nerviosamente. ``Bueno, me alegro de que hayan decidido unirse al bombardeo, pero desearía que hubieran esperado hasta que no estuviéramos al descubierto''.

``¿Bombardeo?'', preguntamos Rykov y yo al unísono.

—¿No te has enterado? Muy pronto, esta roca será propiedad conjunta de la Unión Terran y la República Xanik. —Cazil miró en mi dirección y observó la expresión de asombro en mi rostro—. No te preocupes, general, estoy seguro de que los Jatari serán lo suficientemente sabios como para abandonar el barco.

Parecía que la insinuación del embajador era que los terranos y los xanik se estaban uniendo para invadir la capital. Por muy mal que me pareciera abandonar la Federación que había jurado proteger, ponerse del lado de la Tierra en una guerra equivalía a un suicidio. Los jatari probablemente optarían por la autopreservación, al igual que yo, por lo que no tenía sentido que el comandante mintiera sobre sus intenciones.

Miré a mi amigo humano con el ceño fruncido. —¿Entonces estás atacando a la Federación? Tu plan era tomar el control de la capital desde el principio.

—Absolutamente no —gruñó el Comandante—. No hay ninguna posibilidad de que la Tierra haya firmado para esto.

—Todavía no, pero esperamos el apoyo total de la Unión Terran. ¿Quién necesita escuchar a esos estúpidos débiles mentales de la Federación por más tiempo? —El embajador Cazil se estremeció cuando otro fuego artificial detonó arriba, pero no vaciló—. Su riqueza, su poder, es nuestro para reclamarlo. Seguramente ves la sabiduría.

El humano levantó el rifle y entrecerró los ojos. —¿Sabiduría? Solo veo codicia. Sea lo que sea lo que estés planeando, cancela todo.

Los soldados de Xanik que estaban cerca apuntaron sus armas al Comandante en respuesta, y las tropas terranas también prepararon sus armas. Dudé en unirme al enfrentamiento, pero pensé que de todos modos quedaría atrapado en el fuego cruzado. A pesar de que los humanos estaban en inferioridad numérica diez a uno, apunté con mi rifle a Cazil. No me había gustado desde el momento en que lo conocí, y no iba a perder la oportunidad de eliminarlo.

``Es demasiado tarde. Hemos transmitido nuestras demandas a la Federación y, si no las cumplimos, será una vergüenza para nuestro pueblo'', afirmó el embajador.

—Si sigues adelante, vivirás lo suficiente para ver a tus soldados morir a manos nuestras. —Una nueva voz se escuchó detrás de mí; era Pavel, que parecía haberse materializado de la nada—. Derrocaremos a tus líderes, arruinaremos tu economía y financiaremos a los insurgentes. Luego afirmaremos que no tuvimos nada que ver con eso, y tus súplicas a la Federación caerán en oídos sordos. Nunca tendrás la menor prueba de que estuvimos involucrados de alguna manera.

—¡Qué palabras más feroces, viniendo de un diplomático! ¿No se suponía que usted debía estar en la embajada, liberando a los rehenes?

Pavel se acercó al embajador. ``Han llegado nuevos pedidos''.

A primera vista, el hermano de Rykov parecía desarmado, pero algo en su postura me pareció extraño. Al observarlo más de cerca, noté que sostenía un brazo detrás de la espalda, ocultando un objeto de la vista de Cazil.

El comandante miró su reloj de pulsera y escuchó el timbre de una notificación. ``Yo también acabo de recibir nuevas órdenes. Esta es principalmente una población civil y el gobierno terrano no tolerará que los bombardees hasta someterlos. O te rindes o te tomaré como prisionero de guerra''.

—¡Lo hicimos por ustedes! ¡Por la humanidad! Fuimos los primeros en estar a su lado, ¿y esta es nuestra recompensa? —gritó Cazil—. ¡La Federación los traicionó! No son dignos de su lealtad.

—Última oportunidad. Ríndete —susurró Rykov.

``No puedo hacer eso.''


—Muy bien. Soldados, a mi orden, eliminen todas las fuerzas hostiles. Quiero que capturen con vida al embajador... —El comandante Rykov hizo una pausa al ver a su hermano escabullirse hacia la línea Xanik—. Pavel, ¿qué estás haciendo?

—Lo siento, Mikhail. Tenemos órdenes diferentes.

En un movimiento fluido, Pavel se abalanzó sobre el embajador, sacando una daga más rápido de lo que mi vista podía seguirlo. Cortó las arterias vitales del cuello de Cazil con un corte limpio, dejando al político escupiendo sangre. El fuego de plasma estalló a mi alrededor y apreté el gatillo de mi rifle por instinto. Podía escuchar al comandante pidiendo apoyo aéreo, presumiblemente desde la nave insignia, y a las tropas terranas transmitiendo órdenes.

Una mano humana me tiró al suelo y jadeé cuando un rayo de plasma pasó zumbando justo donde había estado mi cabeza. Miré hacia atrás para ver a Mac, que estaba eliminando a los soldados de Xanik con serena precisión. No me olvidé de darme cuenta de que probablemente me había salvado la vida y le hice un pequeño gesto de reconocimiento. Respiré profundamente y traté de orientarme. Adquirir un objetivo, eliminarlo, repetir y repetir. Sencillo.

Vi a un soldado de Xanik agachado detrás de una barricada policial, sosteniendo una granada propulsada por cohetes. El tiempo se ralentizó mientras miraba a través de mi mira, alineándola con mi punto de mira. Estabilicé mi mano y apreté el gatillo. La bala le quemó la carne de la frente y ella se desplomó en el suelo.

A unos cuantos pasos de mi posición se produjo una explosión que acabó con la vida de dos humanos. Era evidente que el combatiente que había eliminado no era el único que tenía un lanzacohetes. Tal vez lo mejor sería salir del camino abierto antes de que me bombardearan hasta matarme.

Vi a un pequeño grupo de soldados terranos a mi izquierda, que se refugiaban detrás de un puesto de mercado. Unirnos a ellos probablemente sería la apuesta más segura; podríamos agazaparnos allí hasta que llegara el apoyo aéreo. Me puse de pie y corrí en su dirección.

Había dado apenas unos pasos cuando un dolor agudo me atravesó el costado y caí en cascada al suelo. Miré hacia abajo y vi un pequeño agujero en mi estómago; la carne todavía chisporroteaba en los bordes. La sangre brotaba de la herida, tiñendo mi uniforme de un tono verde enfermizo. Intenté moverme, pero mi cuerpo se negó a obedecer.

—¡General! —El comandante Rykov salió a toda prisa de un callejón y en un instante estuvo a mi lado. Me pasó los brazos por debajo de los hombros y me arrastró hasta un lugar seguro—. Todo va a salir bien. Estoy aquí.

La preocupación en sus ojos contaba una historia diferente.

—Tus planes… son todos… estúpidos —tosí, sonriendo débilmente.

El humano se rió, secándose una lágrima. ``Sí, sí, lo son''.

Miré las estrellas que brillaban en el cielo nocturno y admiré su belleza estática. El caos del campo de batalla se estaba volviendo cada vez más tenue, como si mis oídos ya no registraran el sonido. Todo estaba tranquilo, silencioso, quieto. Tal vez la muerte no era el demonio que temía. Me sentí en paz por primera vez en mucho tiempo, lista para dejarme llevar por la noche.

La oscuridad se apoderó de mi visión y luego el mundo se desvaneció.
