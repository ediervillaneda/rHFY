\chapter{El Dilema del General}\label{sec:el-dilema-del-general}

Una mano humana me agarró del hombro y me despertó de un salto.

—Despierte, general —la voz era firme e insistente—. Tenemos que hablar.

Gemí y miré el reloj. Había pasado menos de una hora desde que me quedé dormida y necesitaba dormir desesperadamente. Por mucho que me agradara el comandante Rykov, más le valía tener una buena razón para despertarme.

``¿Puede esperar hasta más tarde? Es…''


—No. Tenemos un problema. Y uno muy grave.

¿Era miedo lo que había en su voz? Mi cerebro se puso alerta al instante y me levanté de la cama sin protestar más. Si eso asustaba al Comandante, solo podía significar una de dos cosas: o los humanos habían hecho algo terrible o había una amenaza apocalíptica en el horizonte. No estaba seguro de cuál de las dos cosas me preocupaba más.

Rykov me guió hasta la salida de mis aposentos, abriéndose paso a través de los sinuosos pasillos de la nave insignia. Su silencio era inquietante y bastante fuera de lugar. Por más tentador que fuera exigir respuestas, tenía la sensación de que las iba a obtener bastante pronto.

Entramos en una sala de conferencias, donde nos esperaban dos personas. Reconocí al refugiado Devorador de nuestras aventuras anteriores, pero no al humano rubio que estaba sentado a su lado. El humano estaba hundido en su silla, luciendo derrotado y exhausto. Byem tenía una mirada distante en su rostro, sin siquiera reaccionar a nuestra presencia.

El comandante se aclaró la garganta y frunció el ceño con desagrado. —Capitán Larsson, por favor repita lo que me dijo. Incluida la parte sobre abandonar un campo de refugiados para morir.

El capitán Larsson se estremeció como si le hubieran dado una bofetada. ``Con todo respeto, señor, nuestro caza estaba averiado y sin municiones. Hubiera sido un ejercicio inútil…''

``No quiero excusas. Simplemente empiecen desde el principio'', dijo Rykov.

—Bueno, para resumir, montamos un campamento de refugiados en una base militar, que los Devoradores atacaron sin previo aviso. Sus escudos recibieron una mejora. Ninguna de nuestras armas principales funcionó y la mayor parte de nuestra flota murió en acción. —El capitán rubio hizo una pausa, evaluando mi reacción. Me las arreglé para mantener mi expresión neutral, pero un nudo de miedo se estaba formando en la boca de mi estómago—. Nuestro CO nos ordenó que nos retiráramos del sistema. Creemos que estarán en camino a la Tierra en breve.

Habían pasado apenas unos días desde la decisiva victoria de los terranos sobre los Devoradores, y sus armas ya se habían vuelto ineficaces. Me estremecí al recordar con qué facilidad los humanos habían atravesado las defensas de la capital; la artillería de la Federación parecía un juguete en comparación. Si ninguna de sus armas de fuego funcionaba... no teníamos ninguna posibilidad.

No pude evitar sentirme responsable, ya que fui yo quien convenció a Rykov de no cristalizar su mundo. En retrospectiva, tal vez los Devoradores deberían haber sido eliminados, mientras la suerte lo permitió. Habría sido la opción pragmática, aunque no la moral.

Según recuerdo, la embajadora Johnson dijo algo así como: ``Prefiero que ellos mueran antes que nosotros'', en su infame discurso en el Senado. Ahora, cuando toda la galaxia se enfrenta a la extinción, tal vez la Federación entienda esas palabras.

``¿Esta base tenía armas nanométricas?'', pregunté.

El capitán Larsson suspiró. ``Por supuesto. Todos nuestros puestos de avanzada los tienen''.

Ni siquiera quería abordar las implicaciones de ese comentario, ya que significaba que los humanos poseían miles de esos misiles que alteraban la realidad. Un arsenal de ese tamaño... ¿los terrícolas planeaban arrasar una pequeña galaxia o algo así?

``Tiene que haber algún tipo de debilidad, alguna grieta en su armadura'', reflexioné.

—Armas más pequeñas. Pero no son precisamente muy potentes. —Larsson tamborileó con los dedos sobre la mesa, con los ojos en blanco, pensativo—. Son vulnerables a las tácticas de embestida. No es que yo lo intentara primero.

—Esas opciones no son ideales —convine—. ¿Cuándo llegarán a la Tierra?

``Mi mejor estimación es unas cuantas horas''.

Cierto. Estábamos condenados.

Miré al comandante Rykov. —Por favor, dígame que tiene uno de sus terribles planes.

—Bueno... —Ni siquiera una sonrisa burlona del humano de cabello oscuro. Evitaba el contacto visual, lo que no era una buena señal—. Creo que tenemos que sacar las armas del fin del mundo.

Casi me desplomé de la impresión mientras intentaba procesar sus palabras. ¿Esas bombas nanométricas, que podían desintegrar una flota entera, no eran sus armas de último recurso? Según sus simulaciones, tenían un cinco por ciento de posibilidades de destruir el universo; fuera lo que fuese a lo que se refería Rykov, tenía que ser realmente espantoso.

—Estoy segura de que me va a encantar —refunfuñé—. ¿Qué son exactamente?

El Comandante miró al silencioso Byem, frunciendo aún más el ceño. —Las llamamos bombas gravitacionales. Pueden usarse para provocar una explosión estelar. Algunos en la Federación nos verán como monstruos, ya lo hacen. Pero tenemos que terminar el trabajo.

¿La humanidad tenía el poder de apagar estrellas? Podía dejar sistemas enteros inhabitables en un instante, condenando especies con un movimiento de la mano. A estas alturas, ya sabía que era una especie peligrosa con la que entablar amistad, pero su capacidad para la violencia nunca dejaba de sorprenderme. ¿Por qué se les ocurriría siquiera algo así durante siglos de paz?

Sin duda, los líderes de la Federación se harían la misma pregunta si este plan funcionara, y tal vez con razón. Al menos estarían ahí para plantear esas preocupaciones. No veía otra manera de proceder; si no se eliminaba a la IA Devoradora, nuestros mundos quedarían reducidos a polvo.

El capitán Larsson se inclinó y le dio un codazo a Byem en el hombro. ``¿Qué piensas?''

—Hazlo. —El refugiado Devorador se movió ligeramente, con el rostro desprovisto de emoción—. Hay destinos peores que la muerte. Ya hemos sufrido bastante.

—¿Y usted, general? ¿Tengo su bendición? —preguntó Rykov.

—Siempre. Pero hay algo que debemos tener en cuenta: ¿los Devoradores no nos verán venir? ¿Intentarán detenernos?

``Estoy seguro de que han dejado algunos barcos atrás. Tendremos que abrirnos paso luchando''.

``¿Qué pasa con los que se dirigen a la Tierra?''

``Según Byem, están programados con un 'interruptor de seguridad' si pierden contacto con el mundo de origen. Si eliminamos a la IA, mataremos dos pájaros de un tiro''.

Hice una mueca. Lo que estábamos a punto de hacer equivalía a un genocidio, y no puedo decir que me pesara la conciencia. Millones de Devoradores esclavizados morirían junto con la IA, y no tendríamos oportunidad de rescatar a ninguno de ellos. A pesar de sus palabras tranquilas, la tensión en el rostro del Comandante me dijo que sentía la misma carga.

``Hay que hacerlo. Debemos partir lo antes posible'', dije.

Rykov asintió. —Muy bien, Larsson, Byem, desembarquen de inmediato. Mis guardias los escoltarán hasta la salida.

—Pero señor, queremos unirnos a usted —protestó el capitán.

—Esa fue una orden. Ambos tienen un aspecto horrible. —El comandante se volvió hacia mí y noté las ojeras bajo sus ojos. Él tampoco tenía muy buena pinta—. General, sígame al puente. Tengo un asunto privado que tratar en el camino.

Corrí tras él y me resultó mucho más fácil seguir su ritmo rápido que antes. La terapia con nanocitos debió haber tenido un impacto positivo en mi estado físico; sin duda me sentía más fuerte, incluso más joven. Eso me recordó que necesitaba llegar a un acuerdo para traer la tecnología a casa. Podría mejorar la calidad de vida, revolucionar el campo de la medicina y reforzar nuestras fuerzas. Con las innovaciones humanas en nuestro bolsillo, podríamos dar inicio a la nueva era dorada de los Jatari.

—Quizás no sea el momento adecuado, pero… —dudé, tratando de encontrar las palabras adecuadas—. Si sobrevivimos a todo esto, estaba pensando que los Jatari podrían ayudar con tu investigación sobre nanocitos. Si quieres socios, claro está.

—Sí, eso es lo que también quería hablar contigo. —El Comandante se pasó una mano por el pelo, suspirando—. Te prometí que no volvería a decir tonterías, así que te digo la verdad. Entiendes lo peligrosa que puede ser la nanotecnología. Lo último que quiere la Tierra es que caiga en las manos equivocadas, así que no quieren que salga de nuestro territorio bajo ninguna circunstancia.

—Ya veo. Eso no me incluiría a mí, ¿verdad? —El humano guardó silencio—. Bueno, mierda. ¿Soy un prisionero?

—Yo no te haría eso, pero… otros sí lo harán. Me ha gustado tenerte a bordo y pensé que tal vez podrías quedarte. Como mi primer oficial.

``Estás sugiriendo que me una al ejército terrestre''.

``Sí.''

Un puesto de mando sonaba mejor que una celda oscura, eso era seguro. Una estancia permanente con los humanos no era exactamente lo que tenía en mente, pero no parecía que tuviera otra opción en el asunto. Rykov claramente sabía lo que implicaría exponerme a los nanocitos, pero no podía culparlo, ya que la alternativa era que me dejara morir.

Sin embargo, la idea de no volver a ver mi hogar era casi insoportable. Y algo me decía que los humanos harían una procesión pública de mi traslado, utilizándolo con fines de relaciones públicas. Mi legado sería destruido, mi nombre sería maldecido y calumniado entre mi gente. Sería recordado como el general que abandonó el barco a la primera señal de problemas, y eso era algo con lo que realmente no podía vivir.

—No es mi primera opción —dije.

``Lo sé, lo siento.''

—¿Sabes lo que mi gente hace con los traidores, comandante? Porque así es como verán la deserción. A los que capturan, los destripan de pies a cabeza y luego vierten ácido dentro de sus cuerpos. Y tu linaje se considera manchado por cien generaciones, tu familia es ridiculizada y golpeada. Cualquier hazaña que hayas logrado alguna vez se quema en las páginas de la historia, tu nombre solo se pronuncia con las maldiciones más viles. Sería mejor morir.

``Entonces les diremos que moriste. Con honor. No saben que estás vivo''.

``Eso… podría ser aceptable. Lo pensaré''.

Mi principal lealtad siempre estaría con los Jatari, pero tal vez podría encontrar una manera de proteger sus intereses desde lejos, de la misma manera que los humanos vigilaban a sus vecinos. Si fuera por mí, nuestra especie trabajaría en conjunto. Pero si la Unión Terran estaba tan decidida a mantener su progreso en secreto, era necesario un enfoque diferente.

Por supuesto, ninguna de esas preocupaciones sería relevante si fracasáramos en nuestra misión. El destino de la galaxia dependía de las próximas horas y sería bueno que, aunque fuera por esta vez, todo saliera según lo planeado.