\chapter{Byem POV: El Último Asalto}

El ejército terrano había desplegado varios cientos de cazas en la órbita superior, que se desplegaron en una alineación defensiva. Nuestras comunicaciones comenzaron a fallar mientras ascendíamos a través de la atmósfera, lo que explicaba el silencio del escuadrón inicial. No habría conexión con la base en tierra; éramos solo nosotros y ellos.

Finalmente supe cómo era estar en el otro lado de nuestras invasiones. Solo hizo falta una mirada a las fuerzas enemigas para reconocer nuestras naves, las mismas que había pilotado durante años. Parecían estar invadiendo el planeta desde todos los lados, como insectos descendiendo sobre un cadáver fresco. Algo había cambiado claramente desde nuestro último encuentro. La flota terrana lanzaba misiles con furia implacable, pero nuestros oponentes los apartaban con facilidad. Las armas de nanobots, antes omnipotentes, eran desactivadas con pulsos, mientras que las armas de antimateria y plasma rebotaban inofensivamente contra sus escudos.

Estábamos atrapados, superados en número, y nuestra única opción era abrirnos camino a la fuerza. Las probabilidades no parecían estar a nuestro favor.

Si eso no fuera suficiente motivo de preocupación, un destructor masivo acechaba en la retaguardia de la formación enemiga. Sabía que estaba lleno de armas diseñadas para hacer un planeta inhabitable, a través de radiación y envenenamiento atmosférico. Lo único positivo que pude encontrar en la situación era que la IA no parecía haber replicado las peores armas humanas, pero no las necesitaba. Podía neutralizar el aluvión de proyectiles y simplemente soportar la tormenta. Los humanos eventualmente se quedarían sin explosivos.

“¿Cuál es el plan?” Pasé por la pantalla holográfica de disparo, tratando de detener el temblor de mis manos. “Tenemos un plan, ¿verdad?”

Carl me miró por encima del hombro. “Improvisamos. Igual que la última vez, ¿recuerdas?”

“La última vez casi morimos,” le espeté.

“Casi es la palabra clave.” El humano ofreció una sonrisa tranquilizadora, pero algo me decía que se estaba convenciendo tanto a sí mismo como a mí. “Mira, los escudos claramente han recibido una mejora. Pero lo que tienen los escudos es que no son buenos para detener proyectiles más pequeños. Yo digo que probemos con mini explosivos.”

Lo miré incrédulo. “Tus misiles no están funcionando, así que solo quieres disparar con misiles más pequeños.”

“Podemos dispararles con balas también. Ya sabes, cinéticas.”

“¿Realmente crees que eso funcionará?”

“¿Honestamente? No lo sé. Pero vale la pena intentarlo.” Carl se pasó una mano por su cabello canoso. “Si quieren tomar este planeta, tendrán que pasar por encima de nosotros. Su próximo objetivo podría ser la Tierra.”

En el lado positivo, no podría funcionar menos que lo que los humanos ya habían intentado. Carl no se equivocaba al decir que el armamento más pequeño podía penetrar los escudos, pero requeriría múltiples impactos. Tan pronto como nos moviéramos, no tenía dudas de que recibiríamos fuego pesado; el enemigo probablemente contaba con una acción agresiva. Tendríamos que desgastar su armadura, apuntando a los compartimentos más vulnerables de las naves. Sabía exactamente dónde apuntar, pero si podría alcanzar esos objetivos en una fracción de segundo, en el calor de la batalla, era otra cuestión.

Cambié a los misiles miniatura; según la computadora, normalmente se usaban como interceptores, pero podían usarse con fines ofensivos si era necesario. Aceleramos hacia la flota enemiga, y nuestros aliados se alinearon detrás de nosotros, inspirados por nuestro avance. Cinco naves de la IA se cerraron a nuestro alrededor, listas para eliminarnos. Apunté a la nave más cercana y ofrecí una oración silenciosa al universo, esperando que nuestra jugada desesperada y suicida funcionara.

Una salva de explosivos, lo suficientemente pequeños como para caber en la palma de mi mano, navegó a través de la negrura. Se dirigieron hacia el flanco derecho del navío, deslizándose a través de los escudos. La armadura repelió el ataque, con solo una abolladura y algunas marcas de quemaduras como resultado. Pero había encontrado el punto dulce, justo al lado de los sistemas de propulsión. La nave comenzó a desviarse, perdiendo su maniobrabilidad.

Carl se desvió hacia la derecha, esquivando una ráfaga de disparos de plasma disparados por los compañeros de la nave dañada. Estaba listo para centrar mi atención en una nave diferente, pero el humano quería ir a matar. Giró de nuevo hacia la nave paralizada, evitando por poco una colisión con ella. Nos posicionó justo encima, dándome una vista clara del objetivo.

“Mismo lugar, Byem, ¡dispara ahora!” gritó.

Lancé otra andanada sin dudarlo. Esta vez, la nave perdió todo control, cayendo en una vertiginosa espiral. El pozo de gravedad del planeta solo alentó su trayectoria hacia adelante, atrayéndola hacia la superficie. Las naves terranas se apartaron alrededor de la nave condenada y luego avanzaron con renovado espíritu. Emularon nuestras tácticas, lanzando pequeños misiles a la flota opositora. ¿Quién habría pensado en atacar una fortaleza con guijarros, sino los humanos?

Las naves enemigas concentraron toda su potencia de fuego en nuestro avance. Al principio, habían sido tomados por sorpresa, pero compensarían rápidamente. Algunos de nuestros aliados encontraron un final prematuro, ya que las potentes rondas de plasma dieron en el blanco. Estas naves humanas no estaban diseñadas para resistir golpes. Tuvimos que confiar en sus marcos rápidos y ágiles para evadir los proyectiles entrantes, porque un solo disparo significaba la muerte para nosotros.

Carl se agachó para evitar la ráfaga de fuego, aprovechando nuestro campo de batalla tridimensional. “Cambia a balas. Vamos a golpear el blando vientre de la bestia.”

Tácticas inteligentes. Típicamente, en la guerra espacial, la “altitud” (en relación con tu oponente, por supuesto) se consideraba ventajosa, por lo que nuestras naves se enfrentaban en el mismo plano. También era por eso que no esperaban que lleváramos a cabo un asalto desde abajo, especialmente mientras estaban ocupados con las naves cargando desde arriba.

Aunque las armas cinéticas eran bastante débiles, las encontré mucho más fáciles de apuntar que sus sucesoras más poderosas. Además, era posible disparar docenas de rondas en rápida sucesión, sin una pausa. Desaté una ráfaga de balas en el vientre de las naves, esperando bombardearlas hasta la sumisión. La visión de la atmósfera ventilándose indicó que, de hecho, habíamos hecho algunos agujeros en su armadura.

Desafortunadamente, ninguna de las naves parecía incapacitada. Éramos más una molestia que una amenaza. Una advertencia apareció en la pantalla de armas, indicando que un misil teledirigido por calor se había fijado en nosotros. De repente, deseé no haber llamado su atención.

Aclaré mi garganta. “¿Carl? Hay…”

“Lo veo. Trata de no vomitar, ¿de acuerdo?” respondió.

“No desperdicio comida, pero cómo...” Me detuve cuando nuestra nave cambió de dirección en un instante, comenzando un ascenso a toda velocidad. El ácido burbujeaba en mi garganta y traté de tragarlo. “Oh. Eso significa vomitar, ¿verdad? No vas a dejar atrás a un misil, lo sabes.”

“No lo planeaba.”

El misil estaba a segundos de alcanzarnos cuando Carl enderezó el caza, nuevamente en la misma dirección que el enemigo. El humano nos dirigió directamente hacia una nave rival, sin mostrar señales de desacelerar. Mientras nos precipitábamos hacia una colisión, no pude evitar preguntarme si su plan era un suicidio. Seguramente, iba a girar y tomar algunas medidas defensivas.

Rozamos justo sobre el casco del enemigo, tan cerca que juré que nuestros esqueletos de metal se rozaron. Nuestra nave se deslizó alrededor de la suya, agachándose del otro lado. El misil, que estaba destinado a nosotros, en su lugar se estrelló contra ellos, y su nave absorbió lo peor del golpe de antimateria. Sus escudos resistieron, pero apenas. El parpadeo de la electricidad alrededor de su caparazón insinuaba que estaban fallando.

No necesitaba instrucciones de Carl para aprovechar esta oportunidad. Si sus defensas realmente estaban bajas, solo teníamos una breve ventana antes de que se recuperara. Cambié a nuestras rondas de plasma, esperando que finalmente funcionaran. Los disparos quemaron su carne vulnerable, convirtiéndola en un caparazón marchito en segundos. Sabía que sus ocupantes estaban muertos cuando la nave no se desvió de su deriva sin rumbo. Si hubiera algún sobreviviente, habrían disparado de vuelta.

Nuestro escaramuzador serpenteó a través de la formación enemiga, zigzagueando para evitar una ráfaga de fuego de plasma. Nuestra pequeña nave se estaba convirtiendo en una espina en su costado,

y sin duda estaban ansiosos por eliminarnos. Por un momento, sentí que estábamos danzando a través del cielo nocturno. Había una gracia sin esfuerzo en las maniobras evasivas de Carl, y se me ocurrió que nada podría tocarnos.

Debo haberlo gafado, porque fue entonces cuando una ronda golpeó nuestra ala izquierda.

El armazón del caza vibró debajo de mí, y me pregunté si la nave se desintegraría. Por algún milagro, llegamos tambaleándonos a su flanco trasero, todavía en una sola pieza. La condición de nuestra nave era preocupante: estaba ralentizada, inclinándose hacia un lado y con municiones peligrosamente bajas.

No estaba seguro de cuánta lucha le quedaba, pero teníamos que seguir adelante. Si esta era nuestra última resistencia, que así sea.

Eché un vistazo al campo de batalla, tratando de pensar en nuestro próximo movimiento. Un mar de metralla y naves heridas había nacido entre los vivos, tal vez mejor descrito como un cementerio. El enemigo parecía haber sufrido más pérdidas que los terranos, pero no importaba. A medida que una de sus naves caía, otra surgía en su lugar. Los humanos luchaban con una intensidad cegadora, pero nuestras fuerzas disminuían demasiado rápido. No podríamos mantener la línea por mucho más tiempo.

El destructor aniquilador de planetas también se acercaba cada vez más. Si se le permitía llegar al rango de disparo, diezmaría las formas de vida abajo, y eso sería el fin del programa de refugiados. Nuestros esfuerzos desesperados necesitaban centrarse en detenerlo. Algunos de los humanos parecían haber llegado a la misma conclusión. Unas pocas docenas de naves terranas estaban rodeando a la bestia, golpeándola con todo tipo de municiones, desde todos los ángulos. Nada parecía dañarla, ni siquiera frenarla.

Deseaba poder sugerir un punto débil para atacar, pero hasta donde sabía, no tenía ninguno.

“Carl, ¿ves el destructor?” pregunté.

Frunció el ceño, entrecerrando los ojos a través del visor. “¿Ese gran y feo trapecio?”

“Sí. Si vamos a perder... necesitamos eliminar esa nave a toda costa.”

“Entiendo.” El humano cayó en silencio. Esperaba que discutiera que no íbamos a perder, pero solo suspiró. “Byem, ha sido un honor.”

“Igualmente.”

Los motores de la nave aceleraron quizás por última vez, y nos dirigimos hacia una muerte segura.