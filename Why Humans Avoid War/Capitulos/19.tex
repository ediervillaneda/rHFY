\chapter{Batalla en el Espacio}\label{sec:batalla-en-el-espacio}

El ejército terrestre había enviado varios cientos de cazas a la órbita superior, que se desplegaron en una alineación defensiva. Nuestras comunicaciones comenzaron a fallar mientras ascendíamos a través de la atmósfera, lo que explicaba el silencio del escuadrón inicial. No habría comunicación con el cuartel general en tierra; solo éramos ellos y nosotros.

Finalmente supe lo que era estar del otro lado de nuestras invasiones. Bastaba con echar un vistazo a las fuerzas enemigas para reconocer nuestras naves, las mismas que había pilotado durante años. Parecían estar invadiendo el planeta por todos lados, como insectos que se lanzan sobre un cadáver fresco. Algo había cambiado claramente desde nuestro último encuentro. La flota terrestre estaba lanzando misiles con una furia implacable, pero nuestros oponentes los aplastaban con facilidad. Las armas nanométricas, que alguna vez fueron omnipotentes, se desactivaban con pulsos, mientras que las armas de antimateria y plasma rebotaban inofensivamente en sus escudos.

Estábamos encerrados, en inferioridad numérica, y nuestra única opción era salir a zarpazos. Las probabilidades no parecían estar a nuestro favor.

Si eso no fuera motivo suficiente para preocuparse, un destructor enorme acechaba en la retaguardia de la formación enemiga. Sabía que estaba repleto de armas diseñadas para volver inhabitable un planeta mediante la radiación y el envenenamiento atmosférico. Lo único positivo que pude encontrar en la situación fue que la IA no parecía haber replicado el peor armamento humano, pero no lo necesitaba. Podía neutralizar la avalancha de proyectiles y simplemente capear el temporal. Los humanos acabarían quedándose sin explosivos.

—¿Cuál es el plan? —Pasé la pantalla holográfica de disparos, tratando de evitar que mis manos temblaran—. Tenemos un plan, ¿verdad?

Carl miró por encima del hombro. ``Improvisamos. Igual que la última vez, ¿recuerdas?''


—La última vez casi morimos —espeté.

—Casi es la palabra clave. —El humano ofreció una sonrisa tranquilizadora, pero algo me decía que se estaba convenciendo a sí mismo tanto como a mí—. Mira, los escudos claramente recibieron una mejora. Pero el problema con los escudos es que no son muy buenos para detener proyectiles más pequeños. Yo digo que probemos con mini explosivos.

Lo miré con incredulidad. ``Tus misiles no funcionan, así que solo quieres dispararle con misiles más pequeños''.

``También podemos dispararle con balas. ¿Sabes? ¿Cinética?''


—¿De verdad crees que algo de eso funcionará?

—¿En serio? No lo sé, pero vale la pena intentarlo. —Carl se pasó una mano por el pelo helado—. Si quieren apoderarse de este planeta, tendrán que hacerlo a través de nosotros. Su próximo objetivo podría ser la Tierra.

El lado positivo era que no podía funcionar de forma menos eficaz que lo que los humanos ya habían intentado. Carl no se equivocaba al afirmar que las armas más pequeñas podían penetrar los escudos, pero se necesitarían varios impactos. Sin embargo, en cuanto nos acercáramos, no tenía ninguna duda de que soportaríamos un fuego intenso; el enemigo probablemente estaba contando con una acción agresiva. Tendríamos que desgastar su blindaje, atacando los compartimentos más vulnerables de las naves. Sabía exactamente a dónde apuntar, pero si podía alcanzar esos objetivos en una fracción de segundo, en el calor de la batalla, era otra cuestión.

Cambié a los misiles en miniatura; según la computadora, normalmente se usaban como interceptores, pero podían usarse con fines ofensivos si era necesario. Aceleramos hacia la flota enemiga y nuestros aliados nos siguieron, inspirados por nuestro avance. Cinco naves de IA se cerraron a nuestro alrededor, listas para derribarnos. Apunté a la nave más cercana y ofrecí una oración silenciosa al universo para que nuestra jugada desesperada y suicida funcionara.

Una salva de explosivos, tan diminuta que cabía en la palma de mi mano, atravesó la oscuridad total. Se dirigieron hacia el flanco derecho de la nave, atravesando los escudos. El blindaje repelió el ataque y solo quedó una abolladura y algunas marcas de quemaduras. Pero había encontrado el punto ideal, justo al lado de los sistemas de propulsión. La nave comenzó a desviarse de su rumbo, ya que perdió su capacidad de dirección.

Carl se desvió hacia la derecha, esquivando una serie de proyectiles de plasma disparados por los compañeros de la nave dañada. Estaba listo para centrar mi atención en otra nave, pero el humano quería ir a por todas. Se desvió hacia la nave dañada, evitando por poco chocar con ella. Nos colocó justo encima de ella, lo que me permitió ver claramente el objetivo.

—¡En el mismo lugar, Byem, dispara ahora! —gritó.

Lancé otra andanada sin dudarlo. Esta vez, la nave perdió todo rastro de control y se precipitó en picada vertiginosamente. La gravedad del planeta solo alentó su trayectoria hacia adelante, arrastrándola hacia la superficie. Las naves terrestres se separaron alrededor de la nave condenada y luego avanzaron con espíritu renovado. Emularon nuestras tácticas, lanzando pequeños misiles a la flota enemiga. ¿Quién habría pensado en atacar una fortaleza con guijarros, excepto los humanos?

Las naves enemigas concentraron toda su potencia de fuego en nuestro avance. Al principio los habían pillado desprevenidos, pero pronto se recuperarían. Algunos de nuestros aliados sufrieron un final prematuro, ya que potentes proyectiles de plasma dieron en el blanco. Estas naves humanas no estaban diseñadas para recibir golpes. Tuvimos que confiar en sus rápidos y ágiles armazones para evadir los proyectiles que se aproximaban, porque un solo disparo significaba la muerte para nosotros.

Carl se agachó para evitar el chorro de fuego, aprovechando nuestro campo de batalla tridimensional. ``Cambia a balas. Vamos a alcanzar la parte más blanda del cuerpo de la bestia''.

Tácticas inteligentes. Normalmente, en la guerra espacial, el ''
terreno elevado''
 (en relación con el oponente, por supuesto) se consideraba ventajoso, por lo que nuestras naves se enfrentaban en el mismo plano. También era la razón por la que no esperaban que lleváramos a cabo un asalto desde abajo, especialmente mientras estaban preocupados por las naves que atacaban desde arriba.

Aunque los sistemas cinéticos eran bastante débiles, me resultó mucho más fácil apuntarlos que sus sucesores más poderosos. Además, era posible disparar docenas de rondas en rápida sucesión, sin siquiera una pausa. Disparé una lluvia de balas hacia la parte inferior de las naves, con la esperanza de bombardearlas hasta que se rindieran. La visión de la atmósfera saliendo indicó que, efectivamente, habíamos hecho algunos agujeros en su blindaje.

Desafortunadamente, ninguna de las naves parecía incapacitada. Éramos más una molestia que una amenaza. Una advertencia brilló en la pantalla de armas, indicando que un misil termoguiado nos apuntaba. De repente deseé no haber llamado su atención.

Me aclaré la garganta. —¿Carl? Hay...

—Ya lo veo. Intenta no perder el almuerzo, ¿vale? —respondió.

—No desperdicio comida, pero ¿cómo...? —Me detuve cuando nuestra nave cambió de dirección en un instante y comenzó a ascender a toda velocidad. El ácido gorgoteó en mi garganta y traté de tragarlo—. Oh, eso significa vómito, ¿no? No vas a superar a un misil, ¿lo sabes?

``No lo tenía planeado.''


El misil estaba a unos segundos de alcanzarnos cuando Carl enderezó el caza, una vez más en el mismo rumbo que el enemigo. El humano nos dirigió directamente hacia una nave rival, sin mostrar signos de desaceleración. Mientras nos dirigíamos a toda velocidad hacia una colisión, no pude evitar preguntarme si su plan era suicida. Seguramente, se daría la vuelta y tomaría algunas medidas defensivas.

Pasamos justo por encima del casco del enemigo, tan cerca que juré que nuestros esqueletos de metal se rozaron entre sí. Nuestra nave serpenteó alrededor de la de ellos, agachándose por el otro lado. El misil, que estaba destinado a nosotros, en cambio se estrelló contra ellos, y su nave absorbió la peor parte del impacto de antimateria. Sus escudos resistieron, pero apenas. El destello de electricidad alrededor de su caparazón insinuó que estaban vacilando.

No necesité instrucciones de Carl para aprovechar esta oportunidad. Si sus defensas realmente estaban bajas, solo teníamos una breve ventana antes de que se recuperara. Cambié a nuestras balas de plasma, esperando que funcionaran por fin. Los disparos quemaron su carne vulnerable, convirtiéndola en una cáscara marchita en segundos. Supe que sus ocupantes estaban muertos cuando la nave no se desvió de su deriva sin rumbo. Si hubiera habido sobrevivientes, habrían respondido al fuego.

Nuestro escaramuzador se abrió paso entre la formación enemiga, zigzagueando para evitar una serie de disparos de plasma. Nuestra pequeña nave se estaba convirtiendo en una verdadera espina para ellos, y sin duda estaban ansiosos por eliminarnos. Por un momento, sentí que estábamos bailando en el cielo nocturno. Había una gracia natural en las maniobras evasivas de Carl, y se me pasó por la cabeza la idea de que nada podría tocarnos.

Debí haberlo gafe, porque fue entonces cuando una bala nos rozó el ala izquierda.

El armazón del caza se sacudió debajo de mí y me pregunté si la nave se rompería en pedazos. Por algún milagro, llegamos a su flanco trasero, todavía en una sola pieza. La condición de nuestra nave era preocupante: estaba ralentizada, se inclinaba hacia un lado y se estaba quedando peligrosamente sin munición.

No estaba seguro de cuántas fuerzas le quedaban, pero teníamos que seguir adelante. Si esta era nuestra última batalla, que así fuera.

Eché un vistazo al campo de batalla, tratando de pensar en nuestro próximo movimiento. Un mar de metralla y naves heridas se había formado entre los vivos, tal vez la mejor descripción sería como un cementerio. El enemigo parecía haber sufrido más pérdidas que los terranos, pero no importaba. Cuando una de sus naves cayó, otra apareció en su lugar. Los humanos lucharon con una intensidad cegadora, pero nuestras fuerzas estaban disminuyendo demasiado rápido. No podríamos mantener la línea por mucho más tiempo.

El destructor destructor de planetas también se acercaba cada vez más. Si se le permitía acercarse a su alcance, diezmaría las formas de vida que había debajo, y eso sería el fin del programa de refugiados. Nuestros últimos esfuerzos debían centrarse en detenerlo. Algunos de los humanos parecían haber llegado a la misma conclusión. Unas cuantas docenas de naves terrestres rodeaban a la bestia, golpeándola con todo tipo de munición, desde todos los ángulos. Nada parecía dañarla, o incluso frenarla.

Desearía poder sugerir una debilidad a la cual apuntar, pero hasta donde yo sé, no tenía ninguna.

—Carl, ¿ves al destructor? —pregunté.

Frunció el ceño y miró a través de la ventana con los ojos entrecerrados. —¿Ese trapezoide grande y feo?

—Sí. Si vamos a perder… tenemos que derribar esa nave a toda costa.

—Lo entiendo. —El humano se quedó callado. Esperaba que dijera que no íbamos a perder, pero se limitó a suspiró—. Byem, ha sido un honor.

``Asimismo.''


Los motores del barco aceleraron quizás por última vez y navegamos hacia una muerte segura.
