\chapter{La Nueva Carrera Armamentística}\label{sec:la-nueva-carrera-armamentistica}

El Comando Terran nos había proporcionado a regañadientes una nueva nave, tras determinar que la nave insignia estaba dañada sin posibilidad de reparación. Esta tenía mejores comodidades, ya que su función principal era la diplomática. Su viaje inaugural sería esa noche, ya que los humanos estaban hospedando a un grupo de dignatarios y oficiales de la Federación.

El plan era dar una breve descripción general de la historia militar de la Tierra, así como de su arsenal actual. El nuevo Portavoz estaba haciendo un esfuerzo genuino por suavizar las cosas, pero yo sabía que sería difícil. Muchos de los representantes todavía luchaban por entender la verdad, y la muerte de Ula había levantado nuevas sospechas. Especialmente porque el Embajador Johnson había estado en la sala cuando la asesinaron.

Aunque las imágenes de seguridad del asesinato mostraban a Byem apretando el gatillo, eso no impidió que los teóricos de la conspiración afirmaran que los terranos orquestaron todo el asunto y que el altercado entre el Embajador y el Devorador fue una farsa. Para ser honesto, no estaba convencido de que estuvieran equivocados.

No es que me molestara que los humanos estuvieran detrás de todo esto. Lo que me molestaba era presenciar mi propio funeral por televisión. Me sentía como un traidor a los Jatari, y ese pensamiento me revolvía el estómago. Todo lo que quería era volver a casa, recuperar mi antigua vida, estar detrás del timón de mi propia nave una última vez.

—Kilon, no te ves bien.

Levanté la cabeza y vi a Rykov en la entrada de mis aposentos. Debía de haber regresado de la capital, antes de la conferencia de esa noche. Entre sus ojos inyectados en sangre y la tablilla en su nariz, pensé que tenía peor aspecto que yo.

Forcé una sonrisa. —Bienvenido de nuevo, Mikhail. ¿O ahora eres el general Rykov?

Hizo una mueca. ``Lo siento. Nunca quise robarte el trabajo. Aún no lo he aceptado, puedo…''


``No seas tonto. Te lo mereces''.

—No sé si lo sé, pero gracias. De todos modos, solo quería avisarte que deberías estar fuera de la nave en una hora. No sería bueno para ninguno de nosotros responder preguntas sobre cómo regresaste de entre los muertos.

``Me imaginé tanto.''


—Bien. Hay una lanzadera esperándote en el hangar. He hecho arreglos para que pases un par de días en el planeta. Puede que te ayude a adaptarte a la cultura humana y, en el peor de los casos, será tiempo libre remunerado.

``Cualquier cosa es mejor que tratar con políticos. Es divertido''.

``Al menos estos no son… ah, no deberían hablar mal de los muertos''.

La mención de Ula despertó mi curiosidad, pero pensé que debía abstenerme de preguntar por ella. Si los terrícolas habían organizado su asesinato, entonces no era un tema del que quisieran hablar. En cualquier caso, descubrir sus secretos nunca terminaba bien; así fue como me quedé atrapado aquí en primer lugar. Cuando se trataba de los humanos, algunas preguntas era mejor dejarlas sin respuesta.

Me quedé mirando al suelo, intentando apaciguar el resentimiento que se estaba gestando en mi mente. ``Probablemente no deberíamos hablar de ella en absoluto''.

Rykov debió haber leído algo en mi expresión, porque entrecerró los ojos. —Quieres saber si matamos a Ula, ¿no? Hasta donde yo sé, no estuvimos directamente involucrados.

—Eso implica que usted estuvo involucrado indirectamente —señalé.

—Bueno… lo siento por Byem. —Una expresión de remordimiento se dibujó en su rostro y su voz se volvió apagada—. Sentí que algo no iba bien la última vez que estuvo aquí, después de que destruyeran el campo de refugiados. Pero nunca pensé que este sería el resultado.

``Nadie podría haber esperado eso. Quiero decir, ¿cómo pudo siquiera introducir un arma en el Salón?''


``Esa es la parte más loca. Acompañó a la prensa y metió la pistola en el bolso de un ingeniero de sonido. Los agentes de seguridad apenas prestan atención a los medios. Una vez que pasó los detectores de metales, simplemente se la robó''.

—¡Qué cabrón tan listo! ¿Quizás tenga una oportunidad mientras está prófugo?

—Tal vez. Por si sirve de algo, espero que nunca lo encontremos.

``Algo me dice que no buscarás con tanta atención.''


—Creo que ya he dicho demasiado, Kilon.

Me resistí a la respuesta mordaz que tenía en la cabeza: que no importaba. No había ninguna buena relación entre Ula y los terrícolas; por supuesto, no estarían ansiosos por llevar a su asesino ante la justicia. De todos modos, ¿a quién le iba a decir que no era humano?

—Muy bien. Bueno, supongo que esto es una despedida —dije.

—Por ahora, cuídate, ¿de acuerdo? —Y con eso, Rykov se despidió con la mano y desapareció por el pasillo.

Cuando la soledad volvió a reinar, mi deseo de acercarme a mi gente se volvió insaciable. En nombre de la amistad, ¿podría ver cómo el ejército terrano, en el mejor de los casos, volvía obsoleta a mi especie? ¿Preservar mi imagen era realmente más importante que asegurar la supervivencia de mi raza?

Los humanos dormitaban ahora, pero su historia demostraba de lo que eran capaces, en las circunstancias adecuadas. Parecía poco probable que pudieran detonar una supernova en mitad de la noche... pero el problema era que podían hacerlo.

La única forma de contrarrestar esa posibilidad era ponerse al día con su tecnología. Como la Tierra, que tenía armas poderosas escondidas, por si acaso. Si ocurría lo peor, los Jatari merecían una oportunidad de luchar.

Al diablo con los riesgos, este fue mi último servicio a mi planeta.

Saqué una navaja de mi mochila y me corté la palma de la mano. Después, con los dedos destapé una botella de agua vacía y dejé que mi sangre goteara en el recipiente. Los nanobots sellaron la herida a toda prisa, pero no antes de que una muestra utilizable se filtrara al fondo.

Saqué un trozo de papel de mi bloc de notas y lo coloqué sobre el escritorio. Mientras buscaba un bolígrafo, las palabras parecían fluir de mi mano por sí solas.

\textit{La nueva carrera armamentística se nos viene encima. Toda la Federación se está esforzando por imitar la tecnología humana, pero con esta muestra de sangre podemos ser los primeros. Esto es sólo una muestra de su proyecto de ingeniería genética (clasificado).}

\textit{La investigación en este campo debería seguir siendo nuestro pequeño secreto. La construcción de un arsenal debería ser el objetivo principal, pero como pueden ver, también hay usos civiles para los nanobots: medicina, construcción... pero no dejen que los humanos vean que los hemos alcanzado. No reaccionarían bien si les niveláramos el campo de juego.}

\textit{¡Diablos!, podría incentivarlos a construir algo peor. Créanme, no incluyan esto en los libros.}

\textit{~Un amigo}

Doblé la nota y la sujeté a la botella con una banda elástica. Una mirada a mi holopad confirmó la ruta a las habitaciones de invitados. Como oficial terrano recién nombrado, también tenía autorización para revisar la lista de invitados de esa noche. El embajador jatari Pallum estaba reservado para la habitación C14, que, por lo tanto, era mi destino.

La misión era bastante sencilla: dejar el paquete y dirigirme al hangar para disfrutar de unas merecidas vacaciones. Lo único que sabía cuando salí de mi alojamiento era que aquello era lo que necesitaba.

Varios humanos se cruzaron en mi camino, pero no me preocupaban. Mientras actuara con normalidad, sabía que no me mirarían dos veces. Mi desvío a la habitación C14 fue breve de todos modos; entré y salí, antes de que los espectadores pudieran sospechar. Solo me tomó unos segundos deslizar el paquete debajo de la almohada de Pallum, y luego continué hacia el hangar como estaba planeado. Con un poco de suerte, el embajador Jatari notaría mi correspondencia cuando se retirara a sus aposentos.

¿El comandante Rykov lo entendería si lo que yo había hecho saliera a la luz? Si nuestros papeles se hubieran invertido, dudaba que abandonara la Tierra. Tal vez, en mi posición, hubiera tomado medidas similares. Era demasiado pedirle a un soldado que le diera la espalda a quienes juró proteger.

Las posibles consecuencias de mi decisión no debían subestimarse; lo sabía. Estábamos lidiando con la guerra de los humanos, una guerra sin honor y sin ganadores. Esta brecha no solo corría el riesgo de provocar la ira de los terranos, sino que también aumentaba la posibilidad de destrucción galáctica. Cuantos más grupos poseyeran armas nanométricas, más probable era que alguien las usara.

Pero esos peligros podrían ser abordados en una fecha posterior. No importaba que estuviéramos viviendo en un polvorín a menos que alguien creara una chispa. Yo iba a apartar de mi mente las cosas terribles que había presenciado, con la esperanza de que un día, fueran verdaderamente olvidadas.

Hoy, mi intención era vivir la vida al máximo y esperar que los humanos siguieran siendo amigos por un poco más de tiempo.

