\chapter{El Plan del Embajador}\label{sec:el-plan-del-embajador}

Las órdenes del comandante Rykov habían sido no atacar a los soldados de Xanik a menos que ellos dispararan primero, pero yo esperaba que abrieran fuego tan pronto como saliéramos de la nave insignia. En cambio, estaban deambulando por la espaciosa terminal, con las armas apuntando al suelo de baldosas en lugar de a nosotros. Un individuo me llamó la atención de inmediato; sus llamativas plumas de color azul oscuro y su pico pronunciado sugerían que era de linaje noble. A diferencia de los demás, vestía su uniforme de gala en lugar de equipo de combate. Podría haber jurado que lo había visto en los medios, aunque no podía identificarlo. ¿Era algún político o general? Si era así, ¿qué demonios estaba haciendo en el campo de batalla?

El comandante humano siguió mi mirada y, al ver al noble xanik, se le reflejó en los ojos un destello de reconocimiento. —Tranquilo. Embajador Cazil, ¿qué es esto?

Los soldados terranos que nos flanqueaban se retiraron al oír las nuevas órdenes y relajaron su postura. Miré con incredulidad al embajador Cazil. Los embajadores planetarios solo estaban presentes en eventos de gran importancia y, para la mayoría de las especies de la Federación, se los consideraba los dignatarios de mayor rango en su gobierno. ¿Un embajador que acompañaba a un equipo de seguridad a una confrontación hostil? Eso era simplemente inaudito.

Cazil se rió entre dientes, con un sonido sordo y retumbante. —Quieres llegar a la embajada de Terran, ¿no? Creo que mi equipo de seguridad personal será una escolta suficiente.

—¿No estás aquí para detenernos? —preguntó Rykov, levantando una ceja con escepticismo.

``Por supuesto que no. La Tierra es nuestro mayor acreedor extranjero y también un importante socio comercial. Declarar la guerra con ustedes paralizaría nuestra economía durante décadas'', respondió el embajador.

El humano sacudió la cabeza y sonrió con sorna. —Déjame aclarar esto. Nos estás ayudando, no porque seamos aliados o porque el comportamiento de la Federación sea injusto, sino por dinero.

``Exactamente.''


—Veo por qué nuestros gobiernos se llevan tan bien —dijo el comandante con un dejo de diversión—. ¿Te das cuenta de que la Federación podría ir a la guerra contigo por ayudarnos, no?

—¿Guerra, con qué ejército? Si nos vamos, las otras especies militares nos seguirán. —Cazil estiró una garra hacia mí—. Parece que ya te has ganado a los Jatari de todos modos. ¡Has convertido al general de más alto rango de la Federación en tu chico de los recados!

Me ardía la sangre por el insulto y levanté mi rifle de plasma hacia la cabeza del embajador. —¿Chico de los recados? Te reto a que lo repitas otra vez.

El comandante Rykov se acercó y me arrebató el arma. —General, le agradecería que no le disparara a nuestro único aliado en el Senado.

Apreté los dientes y sentí que las venas del cuello se me hinchaban. ¿De verdad el Comandante estaba de su lado? Atacar al embajador de Xanik no sería la decisión más inteligente, especialmente estando rodeado de sus soldados, pero su actitud altiva era insoportable.

—Sí, deberías escuchar al humano —dijo Cazil, con una mirada triunfante en sus ojos.

Rykov hizo un gesto con el dedo al embajador. —No te pongas tan petulante. Tú también te estás pasando de la raya, intentando meterte bajo la piel de un Jatari. El general y yo tenemos una cuenta pendiente con cierta persona. No es mi subordinado.

``Tranquilo, solo me estaba divirtiendo un poco. Los Jatari están demasiado tensos'', respondió. ``¿Quién es ese tal ‘alguien’?''


—Bueno, nuestro trato con ella no es oficial, si entiendes lo que quiero decir. —Rykov pasó la mano por el cañón de mi rifle confiscado, con una expresión sombría en su rostro—. Queremos localizar a la portavoz Ula.

—¿Esa duzei? —No estaba muy familiarizada con las blasfemias de Xanik, pero creo que duzei se traduce libremente como intestino-cerebro—. No debería ser difícil encontrarla. Ula está en el Senado mientras hablamos, planteando una moción para la eliminación de la Unión Terrana de la Federación.

—El Senado está en sesión ahora, ¿y tú no estás allí? —Me quejé.

—Tenía mejores cosas que hacer. —Cazil me miró un momento y luego volvió a mirar al comandante—. Francamente, no sé por qué asistió la embajadora Johnson. Supongo que le gusta escuchar a idiotas fanfarrones.

Rykov se rió. ``Probablemente le recuerda a su casa. Pero, hablando en serio, ¿crees que la moción será aprobada? ¿Tiene Ula los votos necesarios?''


El embajador dudó. ``No lo sé. Hay muchos representantes indecisos, pero su ataque al Capitolio probablemente incline la balanza a favor del Portavoz''.

``¿Nuestro ataque? La Federación disparó los primeros tiros'', protestó el Comandante.

—No importa. Ula no lo contará así. —Cazil miró su holopad, evitando la mirada del humano—. Basta de charlas. Deberíamos partir hacia la embajada.

Rykov asintió y me devolvió mi arma de fuego. Nuestro grupo siguió al equipo de seguridad de Xanik mientras salíamos del puerto espacial a paso rápido. Unos pocos humanos se quedaron en la nave insignia para protegerla y, si era necesario, despegar para evitar que la capturaran, pero la mayoría de la tripulación había desembarcado para esta misión. Supuse que para la mayoría de los humanos, esta sería la primera vez que veían la capital en persona. Incluso después de cientos de visitas, todavía encontraba la ciudad-estado digna de contemplar.

Bajo el resplandor esmeralda del sol poniente, la arquitectura de la capital adquirió una cualidad etérea. En el horizonte se alzaba el Salón de Gobierno, un intrincado esferoide azul que albergaba el Senado y el centro de mando militar. El resto de los edificios rodeaban el Salón; a todas las especies de la Federación se les concedió una franja de tierra, y el territorio de los miembros fundadores se encontraba en el anillo interior. Las embajadas solían estar encajadas entre tiendas y lugares culturales, lo que daba a cada región de la ciudad un estilo distinto.

El rincón de los terrícolas era famoso por sus vendedores ambulantes y su vida nocturna, pero hoy los puestos del mercado estaban abandonados. Una multitud de manifestantes no humanos, que se contaban por miles, llenaba la calle. La turba parecía agitada; había barricadas alineadas a lo largo de la avenida para frustrar su avance hacia la embajada. Un muro de ``policías''
 humanas acampaba detrás de las barreras, haciendo retroceder a cualquier manifestante que intentara cruzar el umbral. Digo policías entre comillas porque iban vestidos de pies a cabeza con un equipo de combate negro; un atuendo idéntico al de un soldado terrícola.

El único camino hacia la embajada era a través de la multitud, y algunos de los manifestantes ya habían notado nuestra presencia. Un grupo de ellos se dispersó y cargó contra nosotros, blandiendo armas contundentes y proyectiles improvisados. Era evidente que teníamos que sacar a los manifestantes de nuestro camino, o nos abrumarían con su gran número. El mismo pensamiento debe haber cruzado por la mente del embajador Cazil, porque con un silbido, hizo una señal a sus hombres para que se pusieran en posición de disparar. Los soldados de Xanik encontraron un objetivo, con las garras suspendidas sobre el gatillo...

—¡Alto! ¿Qué les pasa, gente? —gritó Rykov, casi histérico.

Le ofrecí una sonrisa comprensiva. ``No hay otra manera''.

El embajador Cazil hizo un gesto de aprobación, desconcertado por el arrebato del humano. El comandante quitó el silenciador de su rifle sin decir palabra, sin prestar atención a los civiles que se acercaban. Apuntó el cañón al cielo y disparó tres veces en rápida sucesión. Hice una mueca ante los inconfundibles y ensordecedores estallidos.

Los manifestantes que descendieron sobre nuestra posición retrocedieron a toda prisa y oí gritos de la multitud.

—¡Los humanos nos están disparando! —gritó una voz.

En cuestión de segundos, los manifestantes se dispersaron y corrieron para salvar sus vidas. Se dispersaron por callejones y tiendas, despejando el camino para nuestra unidad.

El comandante Rykov suspiró y bajó el arma. ``Siempre hay otra manera''.

Un nudo de vergüenza se instaló en mi estómago. Si no fuera por el Comandante, me habría quedado de brazos cruzados contemplando una masacre innecesaria de civiles. El Portavoz Ula no podía haber estado más equivocado sobre su especie, eso era evidente. Los humanos no tenían ningún deseo de morir y su primer pensamiento siempre era limitar las bajas. Fuera cual fuera el derramamiento de sangre que había ensuciado su historia, habían cambiado.

Uno de los agentes de policía se separó de su formación y marchó por la calle. Se abrió paso entre los soldados de Xanik y señaló a Rykov con un dedo acusador. —¿Por qué estás aquí? Le dije al Comando Terran que no te dejara venir.

El comandante jadeó y miró boquiabierto al extraño. —¿Pavel? Creí que eras un rehén.

—La Agencia tiene formas de entrar y salir de la Embajada. —Pavel se quitó el casco y dejó al descubierto un rostro que parecía una versión más joven de Rykov—. No sería bueno que alguno de nosotros fuera capturado.

—¿Qué es «La Agencia»? —pregunté—. ¿Éste es tu hermano?

El comandante Rykov hizo un gesto con la mano con desdén. ''
El Departamento de Estado. Ya te lo dije. Y sí, este es mi hermano. Pavel, este es...''


Pavel sonrió con sorna. —Lo sé, general Kilon. Es un honor. Y, a menos que esté alucinando, el embajador de Xanik también está en su pequeño grupo.

``Estamos a sus órdenes. En nombre de mi gobierno, pedimos disculpas por las recientes acciones de la Federación''
, dijo Cazil.

—No hace falta. Ya que están aquí, podrían ayudar. —El hermano de Rykov señaló la embajada—. Tengo un plan para sacar a algunos rehenes de un túnel de mantenimiento, pero necesito una distracción.

No podía quitarme de la cabeza la sensación de que Pavel era algo más que un simple empleado del Departamento de Estado, aunque decidí no expresar mis dudas. ¿Qué diplomático común se pondría en contacto con una fuerza policial militarizada o elaboraría planes tácticos para rescatar rehenes?

Fue un alivio que el hermano del Comandante estuviera a salvo, pero algo me decía que debía vigilarlo muy de cerca.

Forcé una sonrisa, intentando actuar con normalidad. ''
Podemos hacerlo. Solo dinos qué tienes en mente''
.

