\chapter{El Ataque y la Decisión}\label{sec:el-ataque-y-la-decision}

``¿Por qué entonces exactamente se instaló el ‘campo de refugiados’ en una base militar?'', pregunté.

El rostro de Carl no reveló nada mientras levantaba la vista de sus cartas. —No todos creen tu pequeña historia, Byem. Quieren vigilarte, asegurarse de que no seas una amenaza. No querrían que intentaras nada en la Tierra.

Después de la misión de rescate en nuestro mundo natal, los humanos nos trasladaron a un puesto militar en una colonia terraformada. Por lo que entendí, su función principal era la de estación de escucha, aunque también podría servir como punto de lanzamiento para un ataque preventivo si fuera necesario. Instalaron una ciudad de tiendas de campaña en las instalaciones con relativa facilidad y tenían abundantes reservas de alimentos para alojarnos, al menos durante unas semanas.

Carl había solicitado un traslado al campo de refugiados mientras se recuperaba de sus heridas, algo por lo que me sentí agradecido. Me di cuenta de que a muchos de los soldados terranos no les agradábamos por su lenguaje corporal tenso y sus respuestas cortantes. Los otros tres humanos en la mesa ni siquiera me habían mirado durante toda la partida.

Ayer me habían enseñado este juego de póquer, pero las reglas eran bastante sencillas. Intentar detectar las señales de engaño de los humanos era otra historia. Sinceramente, este juego parecía diseñado para sociópatas. ¿Su especie obtenía placer mintiendo?

Con un suspiro de frustración, junté las manos. ``Obviamente, ustedes los humanos son mucho mejores mintiendo que yo''.

Carl sonrió con sorna, mientras se llevaba mis fichas. ``No puedes dejar que te intimiden. A veces, tienes que pagar. ¿Puedo ver tus cartas?''

``Adelante'', me quejé.

—¡Byem, por qué te retiraste! —La mirada que me dirigió parecía casi enojada y me encogí en mi asiento—. Tenías ases en la mano. Sabes que eso es bueno, ¿verdad?

-Sí, pero no sé qué tenías.

``No importa. Esa es literalmente la mejor mano inicial''.

``¿Que tenías?''

Carl dio vuelta sus cartas y reveló un dos y un siete. Escuché algunas maldiciones del hombre que estaba a mi lado y estuve tentado de soltar también algunas palabras extravagantes. ¿Por qué haría una apuesta tan grande con una mano tan débil? Desafiaba toda lógica.

``Tal vez simplemente no estoy preparado para esto'', dije.

``Oye, oye, no te rindas ahora. ¿Qué tal si en lugar de preocuparte por mentir, lo miras como un juego de matemáticas? Intenta averiguar las probabilidades de que alguien tenga…''

Una alarma aullante ahogó el final de la explicación de Carl, sobresaltándome casi hasta la muerte. Con su tono estridente, era imposible no oírla. Me tapé los oídos, pero no sirvió de mucho para amortiguar el sonido. El miedo empezó a apoderarse de mí; nada tan fuerte podía significar nada bueno. Miré a los humanos, esperando algo que me tranquilizara. Parecían desconcertados, pero tuve la clara impresión de que estaban esperando órdenes.

—Se acerca un ataque orbital. —Una voz baja y mecánica confirmó mis preocupaciones—. Todos los soldados a sus puestos. Esto no es un simulacro.

Todos los juegos y el ocio se olvidaron en un instante. El mensaje automático ni siquiera había terminado cuando mi amigo me tiró del brazo y me guió hacia la salida del salón. Lo seguí aturdido; nuestra estancia pacífica en el campamento me había llevado a una falsa sensación de seguridad. ¿Quién sería lo suficientemente tonto como para atacar al invencible ejército terrano, después de todo? Estuve tentado de preguntar, pero Carl parecía tan sorprendido como yo.

Salimos del edificio a toda prisa y eché un vistazo al caos que nos rodeaba. Los soldados terranos casi se tropezaban mientras corrían hacia sus destinos, mientras que el personal médico trasladaba a los refugiados a un búnker. Un escuadrón de aviones de combate ya estaba alineado en la pista, preparándose para despegar. La respuesta al ataque fue casi inmediata; los humanos podían interceptar al enemigo antes de que llegara al planeta.

Disminuí el ritmo, tratando de recuperar el aliento. ``¿A dónde vamos?''

—Te llevaré al refugio antiaéreo con los demás —respondió Carl.

Fruncí el ceño. ``¿Y tú qué?''

—Bueno, también querrán que esté allí, porque aún no tengo autorización para volar. —Me apretó la muñeca con más fuerza y me dijo sin palabras que siguiera su ritmo—. Pero me subiré a un barco, de una forma u otra.

—Voy contigo. —A pesar de las protestas de mis instintos de supervivencia, algo no me sentó bien en la idea de dejar que Carl se las arreglara solo. Tal vez solo estaba tratando de convencerme de que ya no era un cobarde—. Necesitas un artillero. Yo soy tu hombre.

—No lo sé... pero pareces estar seguro. Está bien, pongámonos manos a la obra. —Se desvió hacia el hangar más cercano, con pasos rápidos y confiados—. Si alguien nos detiene, déjame hablar.

Algunas personas me miraron con sorpresa cuando entramos al hangar, pero yo no hice caso. La estructura estaba repleta de naves espaciales, todas ellas con un aspecto impecable y bien conservadas. La mayoría de ellas tenían diseños esbeltos y angulares, lo que indicaba que estaban construidas para la velocidad. Había algunas naves más grandes, que probablemente eran las más pesadas. No había nada del calibre de la nave insignia en stock, pero aun así era una armada considerable.

Carl nos dirigió hacia un escaramuzador. Me ayudó a sentarme en el asiento trasero antes de subirme detrás de la columna de dirección; era un asiento mucho más cómodo que la nave en la que volamos antes. Mientras me abrochaba el arnés de seguridad, escuché el interruptor humano en el relé de comunicaciones y una transmisión crepitó en los altavoces.

—Contesta, maldita sea. ¿Qué demonios estás haciendo? —preguntó una voz masculina enfadada.

Carl se aclaró la garganta. —Vamos a matar a unos bastardos alienígenas, señor.

``No irás con esa cosa a cuestas'', fue la respuesta.

La expresión de mi amigo se ensombreció. `` Esa cosa tiene un nombre. Byem voló honorablemente bajo el mando del comandante Rykov…''

``Rykov es un hombre de corazón sensible. Y todo el mundo lo trata como a un maldito héroe popular''.

``Es un héroe y, lo que es más importante, es un buen hombre, pero eso no viene al caso. Señor, confío en Byem con mi vida. ¿Qué daño puede hacer un par de manos extra?''

``Tenemos mucha gente ansiosa por empezar. El primer escuadrón se enfrenta ahora al enemigo, de todos modos esto terminará pronto''.

Una voz femenina se comunicó con nuestro canal, que supuse que era la frecuencia de emergencia. ``Señor, hemos perdido contacto con nuestros cazas''.

—¿Qué? —Hubo una pausa y luego una respuesta mesurada—. ¿Cuántos son?

``Todos'', respondió ella.

El hombre se quedó en silencio, estupefacto, y el silencio se prolongó durante varios segundos. —Rubia, ¿sigues ahí?

Carl frunció el ceño, pero no corrigió su nombre. —Sigo aquí, señor.

``Tienes autorización para despegar. Haz que ese pájaro despegue antes de que cambie de opinión''.

``Sí, señor.''

Me moví en mi asiento mientras nuestra nave salía del hangar y se ponía al final de la cola. Si ninguna de las naves enviadas en la primera oleada regresaba, ¿qué significaba eso para nosotros? ¿Estaban muertas? ¿Estaríamos muertos pronto? No creía que hubiera un ejército en el cúmulo de galaxias que pudiera intercambiar golpes con los humanos, y mucho menos aniquilar a un escuadrón entero.

Se me pasó por la cabeza que era la IA. Tal vez había aprendido de sus derrotas y, tras estudiar lo suficiente, era capaz de replicar o anular la tecnología terrestre. Eso también explicaría por qué habían atacado un puesto de avanzada poco conocido en un sistema periférico. Estaba allí para demostrar que no era posible escapar o, en sus propias palabras, para recuperar sus recursos.

Intenté quitarme esa idea de la cabeza, pero cuanto más pensaba en ella, más seguro estaba de que era verdad. Mis nervios dieron paso a una fría resolución. Después de haber probado la libertad, aunque sólo fuera por unos días, no podía volver a vivir encadenado. La muerte sería preferible.